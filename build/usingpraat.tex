\documentclass[11pt]{article}
\usepackage[margin=1in]{geometry}
\geometry{letterpaper}
\usepackage{graphicx}
\DeclareGraphicsRule{.tif}{png}{.png}{`convert #1 `dirname #1`/`basename #1 .tif`.png}
\usepackage{fontspec,xltxtra,xunicode}
\defaultfontfeatures{Mapping=tex-text}
\setromanfont[Mapping=tex-text]{Charis SIL}
\usepackage[parfill]{parskip}
\usepackage{url}
\usepackage{setspace}
\setstretch{1.1}
%\doublespacing
\usepackage{color}
\usepackage{titlesec}
\usepackage{natbib}
\usepackage{hyperref}
\usepackage{tikz}
\usepackage{pdflscape}
\usepackage{ntheorem}
\usepackage{longtable}
\usepackage{listings}

\lstset{%
  language=perl,
  backgroundcolor=\color{white},
  basicstyle=\ttfamily,
  breaklines=true,
  columns=fullflexible
}

\usepackage{fancyhdr}
\pagestyle{fancy}
\fancyhf{}

\cfoot{\thepage}
\renewcommand{\headrulewidth}{0pt}

\title{\textbf{Using Praat for Linguistic Research}}
\author{\textbf{Dr. Will Styler}}
\date{}                                           % Activate to display a given date or no date
\def\tightlist{}


\begin{document}

\rhead{\color{gray}Using Praat for Linguistic Research - Version 1.9.1.1\color{black}}

\maketitle

\begin{center}
\vspace{0.5in}
\textbf{\small Document Version}: 1.9.1.1

\textbf{Last Update}: \today

\vspace{2.5in}

\textbf{Important!}

This document will be continually updated and improved.  Download the latest version at:

\url{http://savethevowels.org/praat}

You can also contribute changes, fix typoes, or make additions by making pull requests using github:

\url{https://github.com/stylerw/usingpraat}

\vspace{0.5in}

  \centerline{
    \mbox{\includegraphics[width=1.00in]{cc.png}}
  }

Using Praat for Linguistic Research by Will Styler is licensed under a Creative Commons Attribution-ShareAlike 3.0 Unported License.  For more information on the specifics of this license, go to \url{http://creativecommons.org/licenses/by-sa/3.0/}.

\end{center}

\pagebreak

\tableofcontents

\hypertarget{version-history}{%
\section{Version History}\label{version-history}}

\begin{itemize}
\item
  1.9.1.1 - October 26th, 2023 - Fixed some errant markup (thanks to
  Kevin McGowan for the report!) and revised the build script to be more
  linux-friendly
\item
  1.9.1 -- March 19th, 2022 - Added a few notes to Source Filter
  Resynthesis section.
\item
  1.9 - March 4th, 2022 - Moved to open contribution model on GitHub,
  squished some bugs, fixed and separated the bibliography, paid down
  some technical debt.
\item
  1.8.3 - March 10th, 2021 - Added a brief discussion of one bit
  requantization under formula modification.
\item
  1.8.2 - October 5th, 2020 - Added a `Playing Sounds' section, with
  mention of the `Add Silence Before' option which specifically helps
  devices that disable speakers until signal is detected and recent
  Macs.
\item
  1.8.1 - December 25th, 2017 - Fixed a few minor typoes in the text,
  added a note about stereo and sound localization, added details about
  sampling rate and acoustic nasality measurement to the relevant
  sections, and added reference to Styler 2017 to the Nasality Section,
  which more explicitly describes the foibles of nasality measurement.
\item
  1.8 - December 18th, 2017 - Now in Stereo! Added discussion of
  recording stereo signals, manipulating mono vs.~stereo, and a few
  other notes (See Sections \ref{sec:stereo}, \ref{sec:stereomanip}).
  Also added some minor details to LongSound, recording, and squished a
  few typoes.
\item
  1.7.2 - October 31st, 2017 - Added another example of time in a
  formula thanks to some clarification from Paul Boersma (see Section
  \ref{sub:formulas}).
\item
  1.7.1 - October 25th, 2017 - Added an example demonstrating how to use
  time as a compontent of a formula manpulation to \ref{sub:formulas}.
  Niche, but useful! Clarified further on October 31st.
\item
  1.7 - January 2017 - Revisions for the LSA ``Praat Beyond the Basics''
  course, including major revisions to amplitude scaling/matching in
  Section \ref{sub:matchingintensity}, adding Section
  \ref{sub:concatenation} discussing concatenating sounds, some serious
  and minor formatting and typo fixes, and a few other tweaks and
  warnings (particularly about Praat becoming unresponsive) within the
  scripting section.
\item
  1.6.4 - October 5th, 2016 - A few small clarifications based on
  feedback from Marcin Włodarczak. Also, updated list of free software
  for accomplishing batch tasks by removing the (now moribund) Miro
  Video Converter, adding Sox and OpenSesame, and mentioning Forced
  Alignment.
\item
  1.6.3 - May 26th, 2016 - Added words of caution about A1-P1 nasality
  measurement to Section \ref{a1p0}. Also fixed a typo in the ``Create
  Sound from Formula'' section (Thanks to Penelope Howe for the catch!)
\item
  1.6.2 - January 14, 2016 - Discussed how to exit a for loop in Section
  \ref{forloops} and added a Section (and stern warning) about
  \texttt{goto} in Section \ref{goto}.
\item
  1.6.1 - October 29th, 2015 - Added some dire-sounding warnings about
  the need to update Praat regularly, in part reflecting the fact that
  we're now up to Version 6. Also added Section \ref{scriptwontrun},
  which discusses some common ways of fixing things when a Praat script
  you've downloaded from the internet simply won't run.
\item
  1.6 - July 5th, 2015 - Added Section \ref{sec:advanced}: ``Advanced
  Techniques'', including (initially) Amplitude Envelopes, AM
  Demodulation, and using Praat Sounds for non-acoustic data. Also added
  a needlessly florid conclusion, to appease my sense of symmetry.
\item
  1.5.1 - April 30th, 2015 - Added a quick tip about closing Editor
  Windows generated by script to Section \ref{scriptingtips}.
\item
  1.5 - February 2015 - Added ``Script tips!'' to various parts of the
  document to aid in automating these tasks, and added sample code for
  modifying duration. Completely re-typeset for easier maintenance.
  Modified some notes about A1-P0, added some scripting detail, and made
  lots of small revisions for easier understanding.
\item
  1.4.9 - January 22nd, 2015 - Added a brief description of modifying
  Spectral Tilt in Praat
\item
  1.4.8 - November 28th, 2014 - Added a section discussing tasks best
  done outside of Praat and the best free alternative tools (Section
  \ref{scriptingalternatives}), gave stern warnings regarding the use of
  H1-H2 for creak/breathiness (Section \ref{creakybreathy}), the use of
  LongSound files (Section \ref{longsounds}). Added a shameless plug for
  the author's dissertation research to Section \ref{a1p0}, and revised
  the Code Cannibalism section to include a link to the author's Github
  page containing many tasty scripts.
\item
  1.4.5 - March 6, 2014 - Made a number of clarifications, improvements,
  fixes, and updates to the scripting section (\ref{sec:scripting})
  based on (wonderful) feedback from José Joaquín Atria. Thanks, José!
  Also, added brief discussion and URLs for various forced-alignment
  tools in the intro to Section \ref{sec:scripting}.
\item
  1.4.2 - January 16, 2014 - Added Section \ref{scaling}, discussing
  nowarn, noprogress, and other ways to speed up your scripting on large
  datasets.
\item
  1.4.1 - January 10, 2014 - Fixed a few typoes and updates based on
  Gareth Walker's feedback. Thanks Gareth!
\item
  1.4 - December 9, 2013 - Added discussion of the shift in command
  syntax in newer versions of Praat (Section \ref{newcommands}). Added
  discussion of units of intensity, addressing the Pascal vs.~dB issues
  some of my students were having (Section \ref{unitsofintensity}).
  Added further warning, admonition, doom and gloom to Section
  \ref{a1p0} about A1-P0 as a measure of nasality, which, sadly, is
  complicated. Updated the ``Reasonably Recent'' Praat version number to
  encourage people to upgrade. Added quick code example for finding odd
  and even numbers in a Praat script to Section \ref{scriptingtips}.
  Added a brief discussion of ``while'' loops (Section \ref{while}).
\item
  1.3.6 - October 2, 2013 - Added discussion of the smoothing of the
  intensity line and its relation to the pitch track in Praat.
\item
  1.3.5 - March 22, 2013 - Updated License for the Manual. A few typoes
  squashed.
\item
  1.3 - September 28th, 2012 - Added Section 7.7, Manipulating Duration
  with explanations of how to slow and speed sound files
\item
  1.2.5 - August 10th, 2012 - Added discussion of Praat Picture for more
  complex displays
\item
  1.2 - May 19th, 2012 - A few small tweaks, added a section on
  measuring Voice Onset Time
\item
  1.1.1 - January 2nd, 2012 - A few other small corrections based on
  Paul Boersma's feedback.
\item
  1.1 - January 1st, 2012 - Updated with Paul Boersma's valuable
  feedback and a variety of small corrections.
\item
  1.0.2 - July 14th, 2011 - Removed some typoes and fixed other small
  issues.
\item
  1.0.1 - July 10th, 2011 - Updated to include instructions on removing
  scripts from Buttons.
\item
  1.0 - July 10th, 2011 - Version created for LSA Institute Workshop on
  Praat.
\end{itemize}

\hypertarget{contributors}{%
\section{Contributors}\label{contributors}}

\begin{itemize}
\tightlist
\item
  \textbf{Will Styler} - The author (and editor) of the document, and
  sole maintainer from 2011-2022.
\item
  \textbf{Your name here!} - Go to
  \url{https://github.com/stylerw/usingpraat} to contribute text or
  other documents.
\end{itemize}

\pagebreak

\hypertarget{introduction}{%
\section{Introduction}\label{introduction}}

Praat is a wonderful software package written and maintained by Paul
Boersma and David Weenink of the University of Amsterdam. Available for
free, with open source code, there is simply no better package for
linguists to use in analyzing speech.

Unfortunately, much of the existing documentation for the software is
just that, software documentation, and is not designed to help linguists
(who may not necessarily consider themselves to be ``phoneticians'' or
have a strong phonetics background) get the measurements and make the
changes that they need and desire for their research.

As such, rather than introducing each menu item and function as such,
I've instead chosen to describe how to do some of the tasks that
linguists want to do without assuming a strong phonetics or programming
background. Then, eventually, we'll discuss some of the more complicated
measures and tricks one can perform with Praat.

Of course, no one workshop can discuss the myriad features present in
Praat, nor cover all of the quirks of the package, but this workshop
will hopefully leave you feeling more at home in Praat, and give you an
opportunity to go forth and explore further on your own.

\hypertarget{versions}{%
\subsection{Versions}\label{versions}}

This guide is kept up to date, so that all code and commands will work
in the latest version of Praat (6.0, at the time of writing). Download
the latest version right now before getting started with this guide, and
in the future, you should start any given project by downloading the
latest version from \url{http://praat.org}. Paul Boersma is
\emph{exceptionally} good about not breaking old functionality, and
\textbf{not using the latest version of Praat is the most common cause
of issues when trying to run scripts or following online tutorials, so
please do stay up to date!}

All screenshots here are from Praat 5.3.60 running on Mac OS X, but your
copy on your platform should not differ significantly. Unless otherwise
specified, workflows for making measurements and manipulations do not
differ significantly across platforms and versions, and any
version-specific issues are mentioned there.

\hypertarget{other-resources}{%
\subsection{Other Resources}\label{other-resources}}

\label{sec:otherresources}

Although this guide aims to be painfully comprehensive, there are many
other resources available for helping with Praat. The first step for
dealing with any issue is Praat's built in help guide, accessible from
the upper right corner of most windows in the program. You'll be best
served by starting with ``Intro'' and moving from there.

There are also a variety of tutorials for Praat available online, and
the Yahoo! Groups ``Praat-users'' group, whose archives can be searched
at the below link:

\url{http://uk.groups.yahoo.com/group/praat-users/}

You will want to search the archives before posting, as there are likely
a great many people who have had your question before in the history of
the software.

\hypertarget{getting-and-installing-praat}{%
\subsection{Getting and Installing
Praat}\label{getting-and-installing-praat}}

Praat can be downloaded from \url{http://www.praat.org}, and its
installation will vary depending on your platform.

\begin{itemize}
\item
  Mac OS X - Just drag Praat into your Applications folder.
\item
  Windows - Download the installer and run it, and a link to the program
  will be placed on your desktop.
\item
  Linux - If you're running Linux, you'll be able to figure out the
  install on your own. Many distributions have Praat as an installable
  package in their repositories, but check the version numbers, as you
  won't want anything older than 5.2.x, and many scripts will want
  versions newer than 5.4.
\end{itemize}

\hypertarget{using-this-guide}{%
\subsection{Using this guide}\label{using-this-guide}}

This guide is meant to be useful both to people just starting with the
Praat program, and to experienced users who are getting deeper into
Praat scripting.

\textbf{If this is your first exposure to Praat, you can and should
ignore all of the ``Scripting'' sections and ``Script tips!'' bubbles}.
That's meant for people who want to automate what they're doing over
large datasets. You may want to go there someday, but you'll get more
from understanding the interface now that you will from getting behind
the scenes with scripting.

\hypertarget{about-praat}{%
\section{About Praat}\label{about-praat}}

\hypertarget{praat-windows}{%
\subsection{Praat Windows}\label{praat-windows}}

Once you've opened Praat, a variety of windows will open automatically,
and there are many other windows which will pop up when using the
software. It's best to discuss these now so we can refer to them by name
later when discussing the path to certain commands.

The \textbf{Praat Objects window} (Figure \ref{objects}) is where you'll
start most workflows, using this menu to open, create and save files, as
well as to open the various editors and queries which you'll need to
work with sound files.

\begin{figure}
  \centerline{
    \mbox{\includegraphics[width=4.00in]{objects.png}}
  }
    \caption{The Praat Objects Window \label{objects}}
  
  \end{figure}

The \textbf{Editor window} (Figure \ref{editor}) is where you'll spend
most of your time, and can be accessed by selecting a sound and choosing
``View \& Edit''. When examining a sound file, the editor window will
show the sound's waveform on the top and a spectrogram on the bottom,
and the cursor will allow you to take selections and measurements. The
menus along the top will allow you to show and hide different bits of
information (formants, pitch, intensity), as well as to make more
detailed queries. When working with other types of Praat objects
(e.g.~spectra), the editor window will allow you to query those objects
as well.

When you make a query, either in the editor window or from the objects
window, the \textbf{Info Window} will pop up with your results. You can
also print to this window when scripting in Praat (see Section
\ref{sec:scripting}). Note that information printed here will not
necessarily be saved, and running a new query will overwrite it by
default.

The \textbf{Praat Picture window} (shown towards the end of the document
in Figure \ref{picture}) is used to create and display
publication-quality images, and is open by default when you start the
program. For detailed information about using the Pictures window and
why it exists at all, see Section \ref{sec:pictures}.

Knowing the names of all these commands allow us to more easily describe
the commands to use when working with Praat. For instance, if this guide
says that you get the duration of a sound by using:

\begin{quote}
Objects -\textgreater{} Query -\textgreater{} Query Time Domain
-\textgreater{} Get Total Duration
\end{quote}

It means, roughly, ``Go to the Objects window, Choose ``Query'', then
from that submenu choose ``Query Time Domain'', then ``Get Total
Duration''.

\hypertarget{recording-sounds}{%
\section{Recording Sounds}\label{recording-sounds}}

To record sound using Praat, you'll want to plug in your microphone,
sound card, or external ADC (Analog-Digital Conversion) box to your
computer \emph{before starting Praat}, and then\ldots{}

\begin{quote}
\emph{Objects -\textgreater{} New -\textgreater{} Record Mono}
\end{quote}

This will pull up a recording menu which allows you to choose a sampling
frequency (the default, 44100 Hz, is fine for most purposes), as well as
the microphone or other sound source. Press \emph{Record} to record, and
\emph{Stop} to stop, being careful that the sound level bar stays within
the ``green'' range to avoid clipping. Once you've made a recording,
name it and choose \emph{Save to list}, and it will now show up in the
Praat objects window where it's ready for editing.

If you don't see a green bar (indicating that Praat hears you) while you
are recording, try changing the \emph{Input source} on the left side of
the SoundRecorder window. If this doesn't help, go to the computer's
sound control panel to ensure the proper microphone is selected, and
that the input volume is not turned way down.

Praat is limited in its recording length by a pre-set recording buffer.
To record longer sounds, you can either change the buffer size in
\emph{Praat -\textgreater{} Preferences -\textgreater{} Sound Recording
Preferences} or you can use Audacity (free, available from
\url{http://audacity.sourceforge.net/}) to record the session and then
import the sounds into Praat (see Section \ref{opening}) afterwards for
analysis and manipulation. I tend to recommend Audacity, as it's a bit
more robust for this purpose, and the quality is identical.

\hypertarget{mono-vs.-stereo-recording}{%
\subsection{Mono vs.~Stereo Recording}\label{mono-vs.-stereo-recording}}

\label{sec:stereo}

You do have the option of recording a ``Stereo'' signal (as opposed to
`Mono' or Monaural), which records not one but two signals, using two
`channels', by using:

\begin{quote}
\emph{Objects -\textgreater{} New -\textgreater{} Record Stereo}
\end{quote}

Although stereo sound allows us the ability to localize sounds in
side-to-side space, in general, phonetic research works on mono signals,
as most microphones only capture one channel, and the human vocal tract
for a single speaker contains only one sound source. The exceptions are
if you require the sound localization (e.g.~being able to distinguish a
talker on the right from one on the left) that stereo signals provide,
or if you need to record two data sources in a synchronized way, for
instance, recording two speakers (each wearing a different headset
microphone), or recording one speaker's audio from a microphone with an
electroglottograph (EGG) signal capturing vocal fold movement on the
second channel.

In these cases, where you're specifically capturing two sources, you'll
likely want to set up your recording so that each data source is on a
single `channel' (`L' and `R'). This has the advantage of automatically
synchronizing the two signals, such that the start and end of recording
for both is identical, and keeping them together in a single file. The
tracks can be split and combined later (see Section
\ref{sec:stereomanip}).

But unless you have two distinct sources, or , even if your
microphone/computer/setup is capable of stereo recording, \emph{you
should likely record as mono}. If you don't need two independent
signals, by recording as stereo, you're just doubling your sound files'
size with no particular benefit.

\hypertarget{opening-and-saving-files}{%
\section{Opening and Saving Files\}}\label{opening-and-saving-files}}

\label{opening}

\hypertarget{opening-files}{%
\subsection{Opening Files}\label{opening-files}}

If you already have a sound file recorded that you'd like to open
(recorded in .aiff, .wav or .flac format), there are two ways to open it
in Praat. If you're using Praat on OS X, you can drag supported files
onto the Praat icon in the dock. However, if that doesn't work, or if
you're on a different platform:

\begin{quote}
\emph{Objects -\textgreater{} Open -\textgreater{} Read from
File\ldots{}}
\end{quote}

Then use the next dialog to find the files you're interested in on your
hard disk. Once you've loaded the files, they'll appear in your objects
window for further use. Note that other files created by Praat can be
opened in the same way.

\textbf{Tip:} Praat can't open .wma, .mp3 or .m4a audio files. To
convert these easily to .wav files \emph{en masse}, download iTunes, set
it to ``import'' files into the .wav format in Preferences, and use
\emph{iTunes -\textgreater{} Advanced -\textgreater{} Create .wav
version}.

\hypertarget{working-with-longer-sound-files}{%
\subsubsection{Working with longer sound
files}\label{working-with-longer-sound-files}}

\label{longsounds}

Praat has historically had trouble working on sound files more than 20
minutes or so long, and if you're using a 32-bit version or have little
available memory, you may experience frequent out-of-memory errors
working with large files unless you use the \emph{Objects
-\textgreater{} Open -\textgreater{} Open long sound file\ldots{}}
option.

However, \emph{if you can possibly avoid using LongSound objects, avoid
using LongSound objects}. Certain analyses, menu commands, and other
miscellaneous functions cannot be applied to LongSound objects, and as
computers evolve, they're less necessary. There are still times,
particularly when you experience slowdowns when trying to seek to or
play a specific area of a long file, where using LongSound is
advantageous or necessary, but in general, you'll have a nicer
experience importing as conventional sound files until and unless you
experience problems.

To avoid these issues, it's recommended to cut your files into chunks
shorter than an hour, either using Audacity or by editing the Long Sound
object as described in Section \ref{sec:cropcopy}.

For Mac users, this is a mostly a non-issue if you have a 64-bit Mac and
download a recent 64-bit build for OS X. You may still see slowdowns
using the editor working with Sound files which are very large, but
you'll be able to do what you need.

\hypertarget{playing-files}{%
\subsection{Playing Files}\label{playing-files}}

Playing back files is generally straightforward, using either\ldots{}

\begin{quote}
\emph{Objects -\textgreater{} Play}
\end{quote}

\ldots{} or the gray bars at the bottom of the editor window as
described below. Note, however, that if you play from the Objects
window, the sound will continue to play its entire duration. To stop the
sound playing, you'll want to open the sound in the Editor Window, and
then immediately close it.

\vspace{0.5cm}
\begin{tabular}[h]{ p{0.6in} p{12cm}}
\includegraphics[width=0.5in]{danger.png} \newline\textbf{Danger!} & \raisebox{2mm}{\parbox{13cm}{\textit{On recent Apple Hardware, as well as on systems where the speakers are 'turned off' until a signal is sent, you may find the first part of the sound is cut off.  To fix this, use Praat -> Preferences -> Sound Playing Preferences and adjust the value of 'Silence before'.}}}
\end{tabular}

\vspace{0.5cm}

\hypertarget{saving-files}{%
\subsection{Saving Files}\label{saving-files}}

Praat does not save \emph{anything} by default, and until you save files
explicitly, opened or edited versions of the files will exist fleetingly
in the objects window. For emphasis, \textbf{unless you save files
explicitly, they will disappear completely and unrecoverably when Praat
is closed}.

To save a file, select the file in the Objects window, then\ldots{}

\begin{quote}
\emph{Objects -\textgreater{} Save -\textgreater{} Save as \_\_\_\_\_\_
file}
\end{quote}

For sound files, you'll likely choose \emph{Save as WAV file}, but for
the other types of files (TextGrids, formant objects, pitch objects,
etc), you'll save them as text files.

As this can be very tedious, you might consider downloading and
installing a Praat script (see Section \ref{sec:scripting}) which saves
all the objects in the objects window at once.

\hypertarget{phonetic-measurement-and-analysis-in-praat}{%
\section{Phonetic Measurement and Analysis in
Praat}\label{phonetic-measurement-and-analysis-in-praat}}

\hypertarget{working-with-praat-waveforms-and-spectrograms}{%
\subsection{Working with Praat Waveforms and
Spectrograms}\label{working-with-praat-waveforms-and-spectrograms}}

Once a sound has been recorded or opened, you'll spend much of your time
interacting with the sound by means of the Editor window. To open a
sound in the editor window, select the sound and then\ldots{}

\begin{quote}
\emph{Objects -\textgreater{} View \& Edit}
\end{quote}

\begin{figure}
  \centerline{
    \mbox{\includegraphics[width=5.0in]{editor.png}}
  }
  \caption{The Praat Editor Window \label{editor}}

  \end{figure}

You're immediately presented with an editor window (like that in Figure
\ref{editor}), showing the waveform of the sound, and if the sound is
sufficiently short, a broadband spectrogram showing the spectral energy
of the sound over time. In addition, you might also be presented with a
series of red dots (representing formants), blue lines (representing the
speaker's pitch), and a yellow line (representing intensity). These can
be enabled and disabled in the \emph{Editor -\textgreater{} View
-\textgreater{} Show Analyses} menu.

If your window is showing two waveforms, then you've opened a stereo
sound file. The spectrogram displayed will show you the two channels
combined. Everything below will remain true, and there's no fear to be
had, but particularly if the two channels show two types of vastly
different data, the resulting data could be a bit odd, and you might see
Sections \ref{sec:stereo} and \ref{sec:stereomanip} for instructions on
separating the channels.

Clicking within this window will place the cursor on the waveform and
spectrogram. If you click within the editor window, the cursor will
spawn two dotted lines. A vertical bar shows the time within the sound
where you clicked (labeled at the top in seconds) and, if you clicked
within the spectrogram, a horizontal bar shows the frequency at the
cursor (labeled on the left in red). If the pitch or intensity tracks
are displayed where the cursor is placed, values at the time the cursor
represents are given on the left side of the editor window.

In addition, you can click and drag (or use the \emph{Select} menu) to
select portions of the sound. The time of the start and finish of the
selection will be displayed in red, and the duration of the selection
(in seconds) will be displayed in the top of the bar.

To play a sound in the editor window, use the three gray bars at the
bottom of the editor window. The bottom-most bar (\emph{Total Duration})
will play the entire sound. The middle bar (\emph{Visible Part}) will
play only the visible portion of the sound. The different sections of
the top bar (split by the cursor or selection), when clicked, will play
the corresponding pieces of the visible portions of the sound file.
Hitting also plays the visible portion of the file.

Obviously, to view some analyses and to get a closer look at your data,
you'll need to use the five buttons in the bottom left corner of the
window. As you can imagine, \textbf{\emph{all}} shows the entire file,
\textbf{\emph{in}} and \textbf{\emph{out}} zoom in and out,
\textbf{\emph{sel}} zooms to make the current selection fill the window,
and \textbf{\emph{bak}} zooms back to the previous zoom level. For
longer sound files, in order to view analyses like the spectrogram and
formants, you'll need to zoom in to show only a pre-defined amount of
time.\footnote{This amount of time can be changed in \textit{Editor -> View -> Show Analyses -> Longest analysis}.  20 seconds is a sane value for most modern computers, much higher will cause your system to lag when viewing files.}

The \textbf{\emph{Group}} setting in the bottom right corner of the
window will ensure that if two sounds are open in two Editor windows at
once, they'll share the same zoom characteristics. This is best used to
compare two versions of the same file, say, an original versus one with
an acoustic modification made.

All of the measures discussed in this section will use the Editor
window, and you will spend much of your time working with Praat here, so
any time spent gaining familiarity will be repaid tenfold.

\hypertarget{pulling-out-a-smaller-section-of-the-file-for-analysis}{%
\subsubsection{Pulling out a smaller section of the file for
analysis}\label{pulling-out-a-smaller-section-of-the-file-for-analysis}}

Although zooming in and out will get you most of the way there, it's
often useful to isolate a section of a sound (usually a single word or
vowel) into a different Sound object. To do this, select a portion of a
sound, say, a vowel, and then:

\begin{quote}
\emph{Editor -\textgreater{} File -\textgreater{} Extract Selected Sound
(time from 0)}
\end{quote}

This will create a new sound in the Objects window, containing just the
selected part of the original sound. The \emph{(preserve times)} option
(in the same \emph{Editor -\textgreater{} File} menu) just keeps the
timecode on the extracted sound the same as in context (so, if the vowel
starts at 0.245 s, the extracted sound file will start at 0.245 s).

This can also be done from the objects window using \emph{Objects
-\textgreater{} Convert -\textgreater{} Extract part\ldots{}}, if you
know the start and end time of the portion you'd like to extract.

\hypertarget{adjusting-the-spectrogram-settings}{%
\subsection{Adjusting the Spectrogram
settings}\label{adjusting-the-spectrogram-settings}}

\label{subsec:spectrogramsettings}

Although the basic 0-5000 Hz broadband spectrogram will suffice for many
uses, it's useful to be able to change those settings. To make changes
to the spectrogram settings\ldots{}

\begin{quote}
\emph{Editor -\textgreater{} Spectrum -\textgreater{} Spectrogram
Settings}
\end{quote}

This will pull up the Spectrogram settings window (like that in Figure
\ref{spectrogramsettings})

\begin{figure}
  \centerline{
    \mbox{\includegraphics[width=4.0in]{spectrogramsettings.png}}
  }
  \caption{Praat's Spectrogram Settings \label{spectrogramsettings}}

  \end{figure}

The most important settings here are the \textbf{window length} and
\textbf{view range}.

View range controls how much of the spectrum is visible. For speech,
you'll likely be interested in the range from 0 to 5000 or 6000 Hz, but
if you're examining fricatives, you might want to look as high as 15,000
Hz. If you're looking at music, you may focus on the area from 100 to
2000 Hz. Either way, this is how you set which part of the spectrum you
care about.

If your sound files have a relatively small or large \textbf{dynamic
range} (the difference in volume between the loudest and quietest
frequencies or times), or if your spectrograms seems too light or too
dark, you may want to adjust the dynamic range setting, but 50 dB is
usually fine for most
purposes\footnote{Like so many things, the key to understanding the usefulness of these settings is sitting down and playing around with them a bit.  "Fiddling around with settings" is one of the best things a novice Praat user can do with 20 minutes}.

Window length (given in seconds) controls how large of a chunk of the
sound Praat will examine when trying to find the frequencies present at
a given moment in the signal.

This reveals a fundamental tradeoff in signal processing: You can either
have precise information about frequency, or precise information about
time. If you examine large chunks of the signal, you'll be able to see
patterns repeating over and over, and will have a \emph{very} firm
understanding of their frequency. However, if you're looking at a large
chunk, sudden changes (say, the end of a stop consonant, or a
vowel-to-nasal transition) will seem fuzzy. On the other hand, if you
only examine a very small chunk of data for each moment in the
spectrogram, you'll have a very good sense of time, able to make precise
claims about starts and stops, but you'll have very little accuracy in
discussing
frequencies\footnote{Another analogy is a camera.  If you open the shutter for a longer period of time, moving objects are blurry, so you don't know what any of them were doing at the moment you hit the shutter.  If you only open the shutter for a moment, even fast motion is stopped, and you know precisely what was happening when the shutter was tripped.}.

This choice, of ``wide window, good frequency, bad timing'' vs.~``small
window, bad frequency, good timing'' is the difference between Broadband
and Narrowband spectrograms, and switching between them is a useful
technique for analyzing a novel linguistic signal.

\hypertarget{narrowband-vs.-broadband-spectrograms}{%
\subsubsection{Narrowband vs.~Broadband
Spectrograms}\label{narrowband-vs.-broadband-spectrograms}}

\label{subsec:broadnarrow} Praat defaults to showing a Broadband
spectrogram, which is excellent for viewing the temporal structure of
the sound and for seeing vowel formants, but sometimes, you'll want to
look at harmonics and F$_{0}$ instead. To do this, you'll ask Praat to
provide you with a narrowband spectrogram. To do this:

\begin{enumerate}
\def\labelenumi{\arabic{enumi}.}
\tightlist
\item
  \emph{Editor -\textgreater{} Spectrum -\textgreater{} Spectrogram
  Settings}
\item
  Set the \emph{Window Length} to 0.025 (or the narrowband window length
  of your choosing)
\item
  Click OK
\end{enumerate}

Now, harmonics should clearly be visible in the spectrogram. Note that
this is often the most reliable way to quickly visualize pitch
variation, as it's immune to pitch-tracking issues.

To return to a broadband spectrogram:

\begin{enumerate}
\def\labelenumi{\arabic{enumi}.}
\tightlist
\item
  \emph{Editor -\textgreater{} Spectrum -\textgreater{} Spectrogram
  Settings}
\item
  Set the \emph{Window Length} to 0.005 (or the broadband window length
  of your choosing)
\item
  Click OK
\end{enumerate}

\hypertarget{measuring-duration}{%
\subsection{Measuring Duration}\label{measuring-duration}}

As you might expect, measuring duration is quite easy. Once the sound
file is open in the Editor window:

\begin{enumerate}
\def\labelenumi{\arabic{enumi}.}
\tightlist
\item
  Select the portion of the file you'd like to measure (e.g.~the vowel)
  with the cursor
\item
  Read the duration of the selection (in seconds) from the duration bar
  along the bottom of the Editor window OR
\item
  \emph{Editor -\textgreater{} Query -\textgreater{} Get selection
  length} and read your selection in the info window
\end{enumerate}

If you'd like the duration \emph{of an entire file}, just select the
file in the Objects window and:

\begin{quote}
\emph{Objects -\textgreater{} Query -\textgreater{} Query Time Domain
-\textgreater{} Get Total Duration}
\end{quote}

\hypertarget{measuring-voice-onset-time-vot}{%
\subsubsection{Measuring Voice Onset Time
(VOT)}\label{measuring-voice-onset-time-vot}}

``Voice Onset Time'' (VOT) is the time between when the stop is released
and when the voicing of the following vowel begins. Measuring this time,
which can be positive (say, for the English voiceless aspirated stop
FIXMEta), around zero (for the English ``voiced'' stop /d/, or, more
commonly, the voiceless unaspirated {[}ta{]} around the the world), or
negative (for fully voiced stops, where voicing starts before the stop
is released, as found in most non-English languages). Many languages
classify their stops largely based on Voice Onset Time, and it's often
an excellent, more gradient empirical measure of the
``voiced/voiceless'' phonological distinction.

Measuring Voice Onset Time (VOT) is very easy to do in Praat, as it's
just a duration measurement between two set points, the release of the
stop and the start of voicing.

\begin{enumerate}
\def\labelenumi{\arabic{enumi}.}
\tightlist
\item
  Find the stop release
\item
  Find the start of voicing
\item
  Select the span between these two points
\item
  Read the duration of the selection (in seconds) from the duration bar
  along the bottom of the Editor window OR
\item
  \emph{Editor -\textgreater{} Query -\textgreater{} Get selection
  length} and read your selection in the info window
\item
  If the start of voicing came before the stop release, the VOT is
  negative. Otherwise, the VOT is positive.
\end{enumerate}

In general, voiced sounds (in languages other than English) will have a
VOT which is negative, voiceless unaspirated sounds will have a VOT
which is around 0, and aspirated sounds will have a positive VOT.

\hypertarget{examining-and-measuring-f0pitch}{%
\subsection{Examining and measuring
F$_{0}$/Pitch}\label{examining-and-measuring-f0pitch}}

F$_{0}$ and Pitch can be measured in a number of ways in Praat, more and less
reliably.

\hypertarget{measuring-f0-from-a-single-cycle}{%
\subsubsection{Measuring F$_{0}$ from a single
cycle}\label{measuring-f0-from-a-single-cycle}}

The surest way to get an accurate F$_{0}$ for a single cycle is to open the
file in the Editor window, then:

\begin{enumerate}
\def\labelenumi{\arabic{enumi}.}
\tightlist
\item
  Zoom in to the point where you can see individual cycles in the sound
  file
\item
  Select one complete cycle, as accurately as possible, thus, giving
  Praat the period in seconds (t)
\item
  Praat will calculate the frequency of the sound in Hertz in the top
  bar, giving it in the format ( \_\_\_/ s). Use the zoom \emph{sel}
  button to zoom in if you can't see the frequency
  readout.\footnote{This is going to be an accurate number (so long as you gave an accurate period), but you’re welcome to calculate it yourself to make sure.  f = $\frac{1}{t}$, where *t* is the period in seconds}
\end{enumerate}

\hypertarget{viewing-pitch-via-a-narrowband-spectrogram}{%
\subsubsection{Viewing Pitch via a narrowband
spectrogram}\label{viewing-pitch-via-a-narrowband-spectrogram}}

The most reliable way of getting a sense of the pitch through the course
of the word in Praat is by examining a narrowband spectrogram with a
reduced visible range (0 - 400 Hz for speech). This can be done by
editing the spectrogram settings as described in Section
\ref{subsec:spectrogramsettings}. The contours of the harmonics will
accurately represent the pitch contours of the voice during the word,
and doing this will give you a sense of the contour before using the
Praat pitch tracker for more precise measurement.

\hypertarget{using-praats-pitch-tracking}{%
\subsubsection{Using Praat's Pitch
Tracking}\label{using-praats-pitch-tracking}}

\label{pitchtracking} Praat does have the ability to provide a pitch
track in the editor window. To enable the pitch track in the Editor
window:

\begin{quote}
\emph{Editor -\textgreater{} Pitch -\textgreater{} Show Pitch}
\end{quote}

At this point, a blue line will be placed on top of the spectrogram
representing the pitch, where Praat can find it. Once the pitch track is
placed, you can use the cursor to check the pitch at any given point in
the word. Just place the cursor and look for the middle blue number on
the right side of the window. You can also place your cursor at a given
point in the file and \emph{Editor -\textgreater{} Pitch -\textgreater{}
Get Pitch}. Running \emph{Editor -\textgreater{} Pitch -\textgreater{}
Get Pitch} when a chunk of the sound is collected will return the
average pitch during that selection.

\hypertarget{improving-pitch-tracking-by-changing-the-pitch-settings}{%
\subsubsection{Improving Pitch tracking by changing the Pitch
Settings}\label{improving-pitch-tracking-by-changing-the-pitch-settings}}

It's worth noting, though, that Praat's pitch tracking can be quite
finicky. You will often see it jump up and down, doubling and halving
the actual F$_{0}$, and in many cases, especially where the speaker is at all
creaky, the pitch track will drop out altogether. This does not
represent any specific failing of the software, but instead, comes from
the variability and noise inherent in actual phonetic data. Part of the
strength of Praat's approach is that you as a user can help Praat
improve its pitch tracking for a given file or speaker by changing some
of the Pitch settings.

So, in order to do any serious research using the pitch track, and to
avoid some of the problems discussed above, you may need to adjust some
of the pitch settings, to help Praat's pitch tracker better reflect the
speaker's voice. To do so:

\emph{Editor -\textgreater{} Pitch -\textgreater{} Pitch
Settings\ldots{}}

Then adjust the settings as follows:

\begin{itemize}
\tightlist
\item
  \emph{Pitch Range (Hz)}

  \begin{itemize}
  \tightlist
  \item
    Set the pitch range to a reasonable range (50 - 400 for general
    usage, going much higher for song or children's voices). If you have
    a good idea what the speaker's actual range is (taken from a
    narrowband spectrogram, for instance), set the minimum to just under
    the speaker's lowest F$_{0}$ and the maximum to just over their highest
    pitch excursion. Changing the minimum pitch will also change the
    smoothing of the intensity line, when displayed.
  \end{itemize}
\item
  \emph{Unit}

  \begin{itemize}
  \tightlist
  \item
    Here you can choose your unit of choice for the display of the
    speaker's F$_{0}$, ranging from Hertz to Semitones to auditory-scaled
    measures like logHz or mel.
  \end{itemize}
\item
  \emph{Method}

  \begin{itemize}
  \tightlist
  \item
    You'll usually keep this parameter set to ``autocorrelation'', but
    you can switch to ``cross-correlation'' to see if it improves your
    pitch track.
  \end{itemize}
\end{itemize}

You may want to tweak the advanced settings as well. To do so:

\begin{quote}
\emph{Editor -\textgreater{} Pitch -\textgreater{} Advanced Pitch
Settings\ldots{}}
\end{quote}

Then adjust the settings as follows:

\begin{itemize}
\tightlist
\item
  \emph{Silence threshold}

  \begin{itemize}
  \tightlist
  \item
    This is the amplitude threshold for what Praat considers to be
    speech, relative to the peak amplitude in a file. If Praat can't
    find ANY pitch in a quiet file, adjust this setting.
  \end{itemize}
\item
  \emph{Voicing threshold}

  \begin{itemize}
  \tightlist
  \item
    Praat uses this value in its algorithm to help it decide whether
    voicing is present. If Praat is finding voiceless portions of the
    word as voiced, raise this number. If Praat isn't detecting voicing
    that's there in the signal, lower it. This is mostly important when
    working with data with either very large or very small amounts of
    background noise, and can often be left alone.
  \end{itemize}
\item
  \emph{Octave Jump cost}

  \begin{itemize}
  \tightlist
  \item
    Changing this value affects the algorithm's decision about whether a
    jump in F$_{0}$ is reasonable. Larger values discourage abrupt changes in
    F$_{0}$. Increase this number if you are getting pitch-doubling, decrease
    it if you are failing to track actual rapid changes in F$_{0}$.
  \end{itemize}
\item
  \emph{Voiced / unvoiced cost}

  \begin{itemize}
  \tightlist
  \item
    Increasing this number will make Praat more reluctant to claim a
    transition between voicing and voicelessness. Turn this up if your
    pitch track is cutting in and out more than is reasonable.
  \end{itemize}
\end{itemize}

\vspace{0.5cm}
\begin{tabular}[h]{ p{0.6in} p{12cm}}
\includegraphics[width=0.5in]{danger.png} \newline \textbf{Danger!} & \raisebox{2mm}{\parbox{13cm}{\textit{Praat’s pitch tracking is a good way to get a rough idea of what’s going on with the speaker’s $F_{0}$, but relying on it to give you sane measures is not wise, especially in scripts.  Make sure you sanity-check any measures which seem unreasonable against single-cycle $F_{0}$ measurements or against harmonic frequencies, and that you throw out anything completely ridiculous.}}}
\end{tabular}
\vspace{0.5cm}

\hypertarget{scripting-creating-a-pitch-object}{%
\subsubsection{Scripting: Creating a Pitch
Object}\label{scripting-creating-a-pitch-object}}

\emph{Again, if you're just learning Praat for the first time, skip
right past these ``Scripting'' sections. This is down-the-road stuff for
automating these measurements, not where you want to focus today!}

When scripting, you may want to create a Pitch object (select the sound,
then \emph{Objects -\textgreater{} Analyze Periodicity -\textgreater{}
To Pitch}, specifying the proper range) so that you don't need to open
the editor to measure pitch. Once a pitch object is created, you can
instead select the Pitch object and run \emph{Objects -\textgreater{}
Query -\textgreater{} Get value at time\ldots{}} to find the pitch at
whatever time you'd like. Pitch in a pitch object is calculated in the
same way as in the Editor window, so the same disclaimers apply.

\vspace{0.5cm}
\begin{tabular}[c c]{ p{0.7in} p{12cm}}
\includegraphics[width=0.7in]{hal.png} \newline \textbf{Script Tip!} & \raisebox{5mm}{\parbox{12cm}{\textit{Build in sanity-checks for pitch!  Specify the highest and lowest reasonable pitches at the start of the script, use them in the creation of pitch objects and elsewhere, and specify that any values out of this range are wrong, and should be measured again. 10 or 600 Hz F$_{0}$ measurements won't be a problem anymore! }}}
\end{tabular}
\vspace{0.5cm}

\hypertarget{getting-maximum-minimum-and-average-pitch-for-a-section-of-speech}{%
\subsubsection{Getting Maximum, Minimum, and Average pitch for a section
of
speech}\label{getting-maximum-minimum-and-average-pitch-for-a-section-of-speech}}

This is easy.

\begin{enumerate}
\def\labelenumi{\arabic{enumi}.}
\tightlist
\item
  Select the portion of the sound for which you'd like the Maximum,
  Minimum or Average Pitch
\item
  Select the proper command for your task from the \emph{Editor
  -\textgreater{} Pitch} menu.
\end{enumerate}

Note that \emph{Editor -\textgreater{} Pulses -\textgreater{} Voice
Report} will give this information as well.

\hypertarget{measuring-pulses-jitter-shimmer-and-harmonics-to-noise-ratio}{%
\subsection{Measuring Pulses, Jitter, Shimmer, and Harmonics-to-noise
ratio}\label{measuring-pulses-jitter-shimmer-and-harmonics-to-noise-ratio}}

\label{pulsesjittershimmerhnr}

As a part of its pitch-handling system, Praat includes the ability to
find individual glottal pulses in a signal and to analyze the pulses as
part of more complex analyses. To view these pulses, \emph{Editor
-\textgreater{} Pulses -\textgreater{} Show Pulses}, and they'll then
display on top of the waveform in your file. Although the pulses
themselves are mostly only useful in scripting, the \emph{Editor
-\textgreater{} Pulses} menu contains one of the more useful commands in
the program:

\begin{quote}
\emph{Editor -\textgreater{} Pulses -\textgreater{} Voice Report}
\end{quote}

To use this command, simply select a voiced section of the sound, then
\emph{Editor -\textgreater{} Pulses -\textgreater{} Voice Report}. An
information window will then pop up providing you with a variety of
useful measures. In addition to maximum and minimum pitch (with
additional statistics), you will also be given the \textbf{jitter},
\textbf{shimmer}, \textbf{harmonics-to-noise ratio (HNR)}, and the
\textbf{noise-to-harmonics ratios} for the selected portion of the
sound.

Jitter is a measure of the periodic deviation in the voice signal, or
the \textbf{pitch perturbation} of the signal. Put differently, each
cycle of speech with a given F$_{0}$ should, in a perfect world, have the
same period. The jitter in a person's voice is how much one period
differs from the next in the speech signal. This is a useful measure in
speech pathology, as pathological voices will often have a higher jitter
than healthy
voices\footnote{It’s worth noting that \textit{all} voices will have some jitter, as the vocal folds are imperfect, and it will be far more in older speakers, smokers, etc.  The presence of *any* jitter isn't a pathology, just a massive degree, and low-level jitter doesn't particularly affect speech perception.}.

Shimmer (\textbf{amplitude perturbation}) is similar to jitter, but
instead of looking at periodicity, it measures the difference in
amplitude from cycle to cycle. Once again, this is a useful measure in
speech pathology, as pathological voices will often have a higher
shimmer than healthy voices (although again, both healthy and unhealthy
voices will have \emph{some} shimmer).

Harmonics-to-noise ratio (HNR) and Noise-to-harmonics ratio are both
measures of the amount of periodic noise compared to the amount of
irregular, aperiodic noise in the voicing signal. Because the aperiodic
noise often represents frication in the vocal tract, the HNR will go
down significantly with hoarse or breathy speech, and other laryngeal
pathologies will lower the HNR further still.

\hypertarget{measuring-formants}{%
\subsection{Measuring Formants}\label{measuring-formants}}

\label{sub:measuringformants}

Praat has several methods of built in formant measurement. Of course,
the easiest way to examine formant heights is by simply looking at a
broad-band spectrogram and using the cursor to find, roughly their
frequencies. However, ``eyeballing it'' won't pass scientific muster,
and is more time consuming than using Praat's built-in Linear Predictive
Coding (LPC) algorithms as a tool to help to find them.

\hypertarget{using-the-formant-tools-in-the-editor-window}{%
\subsubsection{Using the Formant tools in the Editor
window}\label{using-the-formant-tools-in-the-editor-window}}

When you open a sound file in the Editor window, you can choose to have
Praat calculate and display where it thinks that the vowel formants are
(\emph{Editor -\textgreater{} Formants -\textgreater{} Show Formants}).
This will overlay a series of red dots onto the image which represent
peaks in the series of LPCs which Praat has run at each moment in the
word.

This formant track can be queried at any time in a variety of ways, all
accessed through the \emph{Editor -\textgreater{} Formants} menu. If
you're interested in a single formant's height, you can place the cursor
where you want a measurement and choose \emph{Editor -\textgreater{}
Formants -\textgreater{} Get formant\ldots{}}, but it's often more
efficient to use \emph{Editor -\textgreater{} Formants -\textgreater{}
Formant Listing}, which will give you heights for F1, F2, F3, F4, along
with the timepoint at which the measures were taken.

In addition, if you're interested in formant bandwidth, bandwidth for
the first four formants can be taken using the \emph{Editor
-\textgreater{} Formants -\textgreater{} Get \_\_\_ Bandwidth} commands.

For hand measurement, using \emph{Editor -\textgreater{} Formants
-\textgreater{} Formant Listing} and sanity-checking by visually
inspecting the formants on the Spectrogram will usually produce
reasonable results, but there are ways to improve Praat's
formant-picking performance for a given speaker.

\hypertarget{improving-formant-finding-results}{%
\subsubsection{Improving Formant Finding
results}\label{improving-formant-finding-results}}

For most speakers, the default settings will suffice, but if you find
Praat to be struggling with ``missing'' or the addition of extra
formants, you'll likely find that that particular speaker's formants are
more effectively measured if you make some tweaks to Praat's Formant
Settings, helping the computer with its task.

To apply any of these changes, you'll want to open the Formant Settings
window (see Figure \ref{formantsettings}):

\begin{quote}
\emph{Editor -\textgreater{} Formants -\textgreater{} Formant
Settings\ldots{}}
\end{quote}

\begin{figure}
  \centerline{
    \mbox{\includegraphics[width=4.0in]{formantsettings.png}}
  }
  \caption{Praat's Formant Settings Window \label{formantsettings}}

  \end{figure}

Finding formants is a tricky thing. When you set out looking for areas
of the spectrum where there's a bit of extra energy, you \emph{will}
find them, so the problem isn't finding peaks, but finding \emph{the
correct peaks}. To help in this task, Praat has settings dictating how
many formants it will find, and how spread out those formants will be.

We typically will assume that speakers will have one formant per 1000
Hz, and thus, that there will be 5 formants in the 5000 Hz we usually
worry about for speech research. Thus, ``5'' is the default setting for
\emph{Number of Formants}, and the highest we'll look for formants (the
\emph{Maximum Formant}) is 5000 Hz by default.

Usually, you'll only need to adjust the \emph{Number of Formants}.
Although 5 formants is a good baseline, if Praat is finding formants
where there are none (latching onto a small bump between two actual
formants, usually), you should lower this value down to 4 or 3. If Praat
is finding too few formants (missing F2 and labeling F3 as F2, for
instance), you'll want to raise this number up to 6.

If you're working with a child, gnome, hobbit, or a creature of
otherwise unusually small vocal tract length, you may find that the
Praat is finding non-existent formants between the speaker's F1 and F2,
and missing the speaker's higher formants (F3 and F4) altogether. In
this situation, you'd want to increase the \emph{Maximum Formant (Hz)}
value to tell Praat to search a bit higher up in the spectrum for
formants, and perhaps lower the number of formants it's searching for.
Again, this is a somewhat speaker-specific process, and the first set of
measurements for a new speaker may require some ``dialing in'' to get
accurate measures.

Realize, though, that Praat can always find more peaks, and there are
often small peaks not perceptible to humans which may still have an
acoustical relevance. When there are ``too many formants'', Praat is not
necessarily finding formants which ``aren't there'', but is finding
additional peaks which, although present, aren't the F1, F2 and F3 peaks
which we as linguists are chiefly interested in. When there are ``too
few'', Praat is just giving you only the most prominent peaks that
you've asked for. The results of Praat's formant tracker, in reality,
are largely determined by what you're asking it to find, and this
settings adjustment be done with a mind to what you're actually
interested in.

\emph{Dot size (mm)} simply controls how large the red dots in the
formant display are. Although changing this can be useful if the track
obscures the spectrogram, this will have no effect on your measurements.

These settings persist even once you've closed Praat, so if you make
adjustments here, you'll want to return these settings to the defaults
when you've finished with your odd speaker.

\vspace{0.5cm}
\begin{tabular}[h]{ p{0.6in} p{12cm}}
\includegraphics[width=0.5in]{danger.png} \newline\textbf{Danger!} & \raisebox{2mm}{\parbox{13cm}{\textit{No matter your settings, Praat will happily find you formants even in fricative noise or silence, and because it doesn't know how many formants it should be searching for where, it commonly merges F1 and F2 for high back vowels (where they're close together).  In addition, Praat will often have issues finding a single timepoint, so if you're getting an unusual measurement, a timepoint shortly before or after may be more accurate.  Always sanity-check your measurements, make sure you know what you're measuring, and during automated measurement, always run the results by a trained human first!}}}
\end{tabular}
\vspace{0.5cm}

\hypertarget{scripting-only-formant-objects}{%
\subsubsection{Scripting Only: Formant
Objects}\label{scripting-only-formant-objects}}

When scripting, you may want to create a Formant object (select the
sound, then \emph{Objects -\textgreater{} Formants \& LPC
-\textgreater{} To Formant (burg)\ldots{}}, specifying the proper
settings) so that you don't need to open the editor to measure formants.
Once a formant object is created, you can instead select the Formant
object and run any of the commands in the \emph{Objects -\textgreater{}
Query} menu to get information.

\vspace{0.5cm}
\begin{tabular}[c c]{ p{0.7in} p{12cm}}
\includegraphics[width=0.7in]{hal.png} \newline \textbf{Script Tip!} & \raisebox{5mm}{\parbox{12cm}{\textit{All of the parameters discussed above are manipulable when scripting, so build in some sanity checks to capture the common failure modes.  "If F2 > 3000Hz, increase the number of formants and try again".  This little step can save a lot of pain down the road.}}}
\end{tabular}
\vspace{0.5cm}

Oddly, formant measures taken at the same timepoint and with the same
settings from a formant object and from the editor window do not always
agree, and in fact, can differ significantly. If measuring formants
automatically, you may consider using both methods (the formant object
and the editor window's formant track). It often happens that if one of
the measurement tactics misses or adds a formant, the other will not, so
a measure where both are in agreement is often more trustworthy than one
where they disagree significantly. The author cannot explain the
discrepancy, but is quite happy to leverage it extensively in his
scripting.

\hypertarget{measuring-intensityamplitude}{%
\subsection{Measuring
Intensity/Amplitude}\label{measuring-intensityamplitude}}

Measuring intensity in Praat is relatively straightforward, albeit with
a major disclaimer.

To get the overall intensity of a sound, select the desired sound and
run \emph{Objects -\textgreater{} Query -\textgreater{} Get Intensity
(dB)}. To get the intensity at a specific point in the sound, open it in
an editor window, \emph{Editor -\textgreater{} Intensity -\textgreater{}
Show Intensity}, and then use the various commands available in the
\emph{Editor -\textgreater{} Intensity} menu to get whatever information
you desire.

By default, Praat's display of the intensity of a word is smoothed to
avoid showing individual pulses in the amplitude lines, both in the
editor window and in amplitude objects (when drawn or viewed). This
smoothing is based on the minimum F$_{0}$ of the sound.

If you want to see something closer the amplitude envelope of the sound
in Praat (where pulses show up individually as amplitude peaks), or if
you want the amplitude curve to be smoother than it normally would be,
you must simply adjust the minimum pitch expected by Praat. This can be
done in \emph{Editor -\textgreater{} Pitch -\textgreater{} Pitch
Settings\ldots{}}, as described more fully in Section
\ref{pitchtracking}. Similar smoothing/desmoothing can be accomplished
when creating Intensity objects by altering the minimum pitch value in
the \emph{Objects -\textgreater{} To Intensity\ldots{}} dialog box.

This decrease in amplitude smoothing is particularly useful for
measuring or counting quick, amplitude-based phenomena like taps and
flaps.

That said, \textbf{in most recordings made for phonetic research,
absolute intensity measures as given by Praat are largely meaningless}.
To accurately measure the absolute intensity of a speaker's voice, a
sound-attenuated booth with a calibrated sound level meter or calibrated
microphone with specialized software should be used.

Relative intensity (say, between two segments or words) can be measured
with an uncalibrated microphone, but is only accurate if the recording
is made in a consistently quiet area, and the speaker stayed in the same
general position relative to the microphone throughout the recording(s)
(and wouldn't have changed much during the time between the two points
of comparison). This issue is discussed in depth in Praat's user manual.

\hypertarget{units-of-intensity-db-vs.-pascal}{%
\subsubsection{Units of Intensity (dB
vs.~Pascal)}\label{units-of-intensity-db-vs.-pascal}}

\label{unitsofintensity}

Praat uses two measures of intensity: Pascal and dB. Pascal tend to be
very small numbers (like ``0.00033082594541105064'') whereas dB
measurements are far larger yielding numbers like ``59.23328336655995''.
Often, when scripting or making measurements of intensity of a section
through the interface, we want information in dB, but selecting
\emph{Objects -\textgreater{} Query -\textgreater{} Get mean\ldots{}} to
get Mean intensity gives us the information in Pascal.

In order to get minimum, maximum, or mean intensity in dB, we must first
convert the sound to an intensity object:

\begin{quote}
\emph{Objects -\textgreater{} To Intensity}
\end{quote}

Then \textbf{select the intensity object} and run \emph{Objects
-\textgreater{} Query -\textgreater{} Get mean\ldots{}}. This will
return values in dB, as desired.

Similarly, if scripting this process:

\begin{verbatim}
    select Sound soundname$
    min = do ("Get minimum...", 0, 0, "Sinc70")
    max = do ("Get maximum...", 0, 0, "Sinc70")
    mean = do ("Get mean...", 0, 0, 0)
\end{verbatim}

\ldots{} will yield min, max, and mean intensity measurements in Pascal,
whereas \ldots{}

\begin{verbatim}
    select Sound soundname$
    do ("To Intensity...", 100, 0, "yes")
    min = do ("Get minimum...", 0, 0, "Parabolic")
    max = do ("Get maximum...", 0, 0, "Parabolic")
    mean = do ("Get mean...", 0, 0, "energy")
\end{verbatim}

\ldots{} will yield measurements in dB.

\hypertarget{working-with-spectra}{%
\subsection{Working with Spectra}\label{working-with-spectra}}

Sometimes, you need specific details about the frequencies and
individual harmonics in a sound at a given moment in time, and examining
a narrowband spectrogram alone does not provide sufficient information.
In these cases, you'll need to take a spectral slice for analysis.
Spectral slices (also referred to as FFTs or spectra) are the result of
a fast fourier transform done on a very small portion of the sound,
providing you with very specific information about the frequencies
present in the sound and their relative amplitudes.

Spectral slices are useful for a variety of measures of \(F_0\),
nasality, creak, breathiness, and spectral tilt, and are a crucial part
of many measurement workflows.

\hypertarget{taking-a-spectral-slice}{%
\subsection{Taking a spectral slice}\label{taking-a-spectral-slice}}

\label{sub:gettingaslice} To take a spectral slice, you'll need to do
the following:

\begin{enumerate}
\def\labelenumi{\arabic{enumi}.}
\tightlist
\item
  \emph{Editor -\textgreater{} Spectrum -\textgreater{} Spectrogram
  Settings}
\item
  Set \emph{Window Length} to ``0.025'' (effectively producing a
  narrow-band spectrogram)
\item
  \emph{Editor -\textgreater{} Spectrum -\textgreater{} Advanced
  Spectrogram Settings}
\item
  Set \emph{Window Shape} to ``hamming''
\item
  Select the point at which you'd like to see the slice taken
\item
  \emph{Editor -\textgreater{} Spectrum -\textgreater{} View Spectral
  Slice}
\end{enumerate}

This will create a new Spectrum object, and pull up a window like that
in figure \ref{spectralslice}, showing amplitude on the Y axis, and
frequency (from 0 up to the Nyquist frequency) on the X axis. You can
zoom in and out using the buttons in the bottom left corner, as you
wish.

\begin{figure}
  \centerline{
    \mbox{\includegraphics[width=5.0in]{spectralslice.png}}
  }
  \caption{A spectral slice Editor window \label{spectralslice}}

  \end{figure}

If you've selected a portion of the sound (rather than a single
timepoint) when you use \emph{Editor -\textgreater{} Spectrum
-\textgreater{} View Spectral Slice}, Praat will create a spectrum
representing the average characteristics across the entire selection,
which isn't useful for most of the spectral measures discussed below.

\hypertarget{measuring-harmonic-amplitude-frequency}{%
\subsection{Measuring Harmonic Amplitude,
Frequency}\label{measuring-harmonic-amplitude-frequency}}

Getting harmonic frequency and amplitude in a Spectrum Editor window is
fairly straightforward, as clicking anywhere within the spectrum editor
window will give you the frequency and amplitude measurements at the
cursor. So, to find the amplitude and frequency of a given point in the
spectrum, click that point and read off the amplitude (on the left) and
the frequency (at the top), as shown in figure \ref{spectralslice}.

This can be done even more easily and efficiently by script, allowing
you to find the highest point on a given harmonic without clicking
guesswork.

\hypertarget{measuring-creakiness-and-breathiness-using-spectral-tilt}{%
\subsection{Measuring Creakiness and Breathiness using Spectral
Tilt}\label{measuring-creakiness-and-breathiness-using-spectral-tilt}}

\label{creakybreathy}

\textbf{Spectral tilt} is often used in phonetic research as a measure
of creak. As discussed in Gordon and Ladefoged 2001
\cite{Gordon:2001um}:

\begin{quote}
One of the major acoustic parameters that reliably differentiates
phonation types in many languages is spectral tilt, i.e., the degree to
which intensity drops off as frequency increases. Spectral tilt can be
quantified by comparing the amplitude of the fundamental to that of
higher frequency harmonics, e.g., the second harmonic, the harmonic
closest to the first formant, or the harmonic closest to the second
formant. Spectral tilt is characteristically most steeply positive for
creaky vowels and most steeply negative for breathy vowels.
\end{quote}

Spectral tilt is easily measured by finding H1 and H2, measuring their
amplitudes as described above, and comparing the two numbers.

\textbf{However}, H1-H2 is subject to very strong interference from
nasality, as described in A.P. Simpson's sternly named paper \emph{The
first and second harmonics should not be used to measure breathiness in
male and female voices.} \cite{simpson2012first}, and investigators of
these phenomena would do well to read that paper, and focus on other
measures like Harmonics-to-Noise ratio (see Section
\ref{pulsesjittershimmerhnr}).

\hypertarget{measuring-nasality-using-a1-p0}{%
\subsection{Measuring Nasality using
A1-P0}\label{measuring-nasality-using-a1-p0}}

\label{a1p0}

\textbf{A1-P0} is an acoustical measure of nasality first described by
Marilyn Chen in \emph{Acoustic correlates of English and French
nasalized vowels} (\cite{Chen:1997vr}), and later discussed in Styler
2017 (\cite{Styler:2017}). Like spectral tilt, it's a ratio measure of
the amplitudes of two harmonics: A1, which is the highest harmonic peak
near the first formant, and P0, which is a low frequency harmonic
(usually H1 or H2) which corresponds to a low resonance in the nasal
passages. See Figure \ref{chennasalnon} for an illustration of these
peaks in spectra. To compute A1-P0, you need to take three main
steps\footnote{Many thanks to Dr. Rebecca Scarborough, on whose handout “Measuring Nasality (using A1-P0)” this section is loosely based}

\begin{enumerate}
\def\labelenumi{\arabic{enumi}.}
\tightlist
\item
  Find A1 and measure its amplitude
\item
  A1 is the highest harmonic near the frequency of the first formant.

  \begin{itemize}
  \tightlist
  \item
    F1 can be located as described in Section
    \ref{sub:measuringformants}
  \item
    The frequency of F1 will vary from vowel to vowel, tending to be
    lower for high vowels and higher for low vowels. The amplitude of
    A1, though, will not vary by vowel quality
  \end{itemize}
\item
  Find P0 and measure its amplitude

  \begin{itemize}
  \tightlist
  \item
    P0 is a specific harmonic peak which is reinforced by resonances
    within the nasal passages.
  \item
    The frequency of P0 will be specific to each speaker, and won't
    change from word to word (as the speaker's nasal passages are
    unlikely to change resonant characteristics) or from vowel to vowel.
  \item
    The amplitude of P0 will change from word to word, depending on the
    degree of openness of the velopharyngeal port.
  \item
    Although each talker is different, P0 is likely to correspond to
    either H1 or H2 for a given speaker
  \item
    In general, for speakers with lower pitched voices, P0 will be H2,
    and for speakers with higher pitched voices, P0 will likely be H1
  \item
    The best way to identify this peak is by examining a known nasal
    vowel and a known oral vowel to determine which peak is reinforced
    by the nasality. You may need to examine several words before a
    clear winner emerges.
  \end{itemize}
\item
  Subtract P0 from A1 to get the measurement
\end{enumerate}

\begin{figure}
  \centerline{
    \mbox{\includegraphics[width=5.0in]{chennasalnon.png}}
  }
  \caption{Figure 2 from Chen 1997 (\cite{Chen:1997vr}), showing A1 and P0 in oral vs. nasal vowels \label{chennasalnon}}

  \end{figure}

Because the nasal peak (P0) will be reinforced by the resonances in the
nose during nasal vowel production, nasal vowels will tend to have
\emph{lower} A1-P0 values than non-nasal vowels. Across many tokens,
A1-P0 can be a very good predictor of vowel nasality, however, there are
several important things to keep in mind when using A1-P0 to measure
vowel nasality:

\begin{enumerate}
\def\labelenumi{\arabic{enumi}.}
\tightlist
\item
  A1-P0 only works in situations where F1 is higher in frequency than H1
  and H2. Most of the time, this is true, but \textbf{for high vowels,
  A1 and P0 often occur at the same place in the spectrum, leaving the
  measurement unreliable}.

  \begin{itemize}
  \tightlist
  \item
    In these situations, A1-P1 (a second nasal resonance at
    \textasciitilde950 Hz, described in \cite{Chen:1997vr}) is an
    oft-used measurement, but identifying P1 can be very difficult for
    some speakers, and even when found, the A1-P1 is rather unreliable
    (c.f. Styler 2017). A1-P0 is probably just as relable here.
  \end{itemize}
\item
  A low A1-P0 relative to known oral tokens is a good predictor of
  nasality \emph{across a large number of tokens}, but individual
  nasal/oral vowel pairs may or may not demonstrate a strong effect. It
  should not be used to measure the nasality of one individual token
  compared to another.
\item
  A1-P0 should only be examined in comparison with other tokens from the
  same speaker. An A1-P0 of -2 may be normal for an oral vowel for some
  speakers, but indicate extreme nasality for others. The absolute value
  of A1-P0 is not interpretable across speakers.

  \begin{itemize}
  \tightlist
  \item
    Different speakers have not only different baseline values, but
    different amounts of A1-P0 change from oral to nasal vowels.
    Comparison of raw or Z-Scored A1-P0 values across speakers is not
    wise, and will lead to untrustworthy measurements.
  \end{itemize}
\item
  Remember, here you're relying on Praat to give an accurate measurement
  of the first formant, as well as F$_{0}$. Feel free to tweak the formant
  settings, and don't blindly trust Praat's formant tracker to find what
  linguists call the first formant and not some other spectral
  prominence.
\end{enumerate}

This is a complex measure, and I (personally) have spent a great deal of
time working with it and its
measurements\footnote{For more on nasality, please see the description of my dissertation work, \cite{Styler:2015aa} and \cite{Styler:2017}, posted at \url{http://savethevowels.org/will/publications.html}. I did not write 84 pages worth of Praat manual only to shy away from a completely self-serving and shameless plug in the sole earthly context in which nasality research is actually relevant.}.

Although it is among the best acoustical approaches to nasality
presently available, it is also remarkably noisy, capricious, and
complex. A1-P0 nasality should be one element of your successful
analysis, not the sole element, and as I have discovered, no nasality
experiment is simple. Proceed with caution.

\hypertarget{measuring-spectral-center-of-gravity}{%
\subsection{Measuring Spectral Center of
Gravity}\label{measuring-spectral-center-of-gravity}}

\textbf{Spectral center of gravity (Spectral COG)} is useful for
measuring the frequency characteristics of aperiodic sounds in speech
(release bursts and fricatives, usually). It is most often used to
describe the production of fricatives (with the understanding that
sounds with a higher spectral COG are often produced more towards the
front of the mouth), and simply measures the overall weighting of the
noise in the spectrum by reporting its ``center of gravity''. See Figure
\ref{spectralcog} for a more visual demonstration. This measure can be
useful in sociophonetic work, as well as in fieldwork, for determining
the articulatory positions of different fricatives.\\

\begin{figure}
  \centerline{
    \mbox{\includegraphics[width=6.0in]{spectralcog.pdf}}
  }
  \caption{Two spectra showing the higher spectral COG for a token of /s/ (here, 8202 Hz) compared with the lower spectral COG for a token of [ʃ] (here, 3243 Hz)\label{spectralcog}}

  \end{figure}

To measure spectral COG in Praat:

\begin{enumerate}
\def\labelenumi{\arabic{enumi}.}
\tightlist
\item
  Open the sound in an Editor window
\item
  Complete the steps in Section \ref{sub:gettingaslice} to create a
  spectral slice in Praat at the point you'd like to measure
\item
  Select the spectral slice in the Objects window, then \emph{Objects
  -\textgreater{} Query -\textgreater{} Get Centre of Gravity\ldots{}}
\end{enumerate}

An info window will then pop up, presenting you with the spectral COG
for the point represented by the spectrum.

For further information about Spectral Center of Gravity and an example
of its use in fricative description and cross-linguistic comparison
(showing the relationship between articulation and spectral COG), refer
to Gordon et al 2002 \emph{A cross-linguistic acoustic study of
voiceless fricatives}.
\cite{Gordon:2002vv}.\footnote{Paul Boersma has stated in correspondence that this paper incorrectly implements Spectral COG, leading to inconsistent results, and thus, should not be emulated.  For more information, see Boersma \& Hamann 2008 (from http://www.fon.hum.uva.nl/paul/papers/BoersmaHamannPhonology2008.pdf), Footnote 7}.

\hypertarget{creating-and-manipulating-sound-files-in-praat}{%
\section{Creating and manipulating sound Files in
Praat}\label{creating-and-manipulating-sound-files-in-praat}}

\hypertarget{creating-sounds-from-formula}{%
\subsection{Creating sounds from
Formula}\label{creating-sounds-from-formula}}

\label{creatingsounds}

To create a sound from formula (a pure or complex tone) , you'll want to
use:

\begin{quote}
\emph{Objects -\textgreater{} New -\textgreater{} Sound -\textgreater{}
Create sound from formula\ldots{}}
\end{quote}

There, you can plug in a formula which will generate the sound you want.
For instance:

To create a 1000 Hz puretone at 0.5 dB:

\begin{enumerate}
\def\labelenumi{\arabic{enumi}.}
\tightlist
\item
  \emph{Objects -\textgreater{} New -\textgreater{} Sound
  -\textgreater{} Create sound from formula\ldots{}}
\item
  Under formula, enter \texttt{1/2 * sin(2*pi*1000*x)}
\item
  Specify the time, sampling frequency, number of channels and such as
  you desire
\end{enumerate}

To create a 250 Hz puretone at 1 dB:

\begin{enumerate}
\def\labelenumi{\arabic{enumi}.}
\tightlist
\item
  \emph{Objects -\textgreater{} New -\textgreater{} Sound
  -\textgreater{} Create sound from formula\ldots{}}
\item
  Under formula, enter \texttt{1 * sin(2*pi*250*x)}
\item
  Specify the time, sampling frequency, number of channels and such as
  you desire
\end{enumerate}

To create a complex sound with 1 dB components at 250 and 1000 Hz:

\begin{enumerate}
\def\labelenumi{\arabic{enumi}.}
\tightlist
\item
  \emph{Objects -\textgreater{} New -\textgreater{} Sound
  -\textgreater{} Create sound from formula\ldots{}}
\item
  Under formula, enter
  \texttt{(1* sin(2*pi*250*x)) + (1 *sin(2*pi*1000*x))}
\item
  Specify the time, sampling frequency, number of channels and such as
  you desire
\end{enumerate}

For more information, see the Praat Help guide's excellent Formulas
Tutorial. To create a sound with harmonics, use \emph{Objects
-\textgreater{} New -\textgreater{} Sound -\textgreater{} Create sound
from tone complex\ldots{}}

Note as well that you can use `if' statements in formulae, as well as
`else if' statements (but not `elif' or `elsif'), so long as you include
the relevant `endif'. See Section \ref{sec:soundbarrier} for more
information.

\hypertarget{working-with-stereo-files-converting-combining-and-extracting-channels}{%
\subsection{Working with Stereo Files (Converting, Combining, and
Extracting
channels)}\label{working-with-stereo-files-converting-combining-and-extracting-channels}}

\label{sec:stereomanip}

Recording as stereo files can be excellent for synchronizing and
combining two independent data-streams (two speakers with two
microphones, oral vs.~nasal airflow, etc), but there are a few
specialized commands needed when working with stereo files, each with
specific use cases.

\hypertarget{converting-a-single-stereo-file-into-a-single-mono-file}{%
\subsubsection{Converting a single stereo file into a single mono
file}\label{converting-a-single-stereo-file-into-a-single-mono-file}}

Sometimes, you have a stereo file which doesn't need to be Stereo.
Perhaps it was recorded with a stereo microphone in a non-stereo
setting, or the two tracks are (nearly) identical. In this case, because
you want to preserve both channels' information, you'll want to
use\ldots{}

\begin{quote}
\emph{Objects -\textgreater{} Convert -\textgreater{} Convert to mono}
\end{quote}

This command will take the Stereo track and then \emph{mix the two
channels together to create a single mono track}. This incorporates the
data from both tracks into the output mono file, so relatively little is
lost. But this is \emph{not} the best choice if you only need/want one
of the two stereo channels. In that case, you'll want to\ldots{}

\hypertarget{extracting-a-stereo-files-two-channels-into-separate-mono-files}{%
\subsubsection{Extracting a stereo file's two channels into separate
mono
files}\label{extracting-a-stereo-files-two-channels-into-separate-mono-files}}

If you have a stereo file where both channels need to be preserved and
analyzed individually, you'll need to use\ldots{}

\begin{quote}
\emph{Objects -\textgreater{} Convert -\textgreater{} Extract all
channels}
\end{quote}

This command will take the Stereo track and then extract each channel
into two separate files (\_ch1 and \_ch2). These are then just boring
mono files, with the same start and end times, but each containing only
one channel's information. If you only need one channel (e.g.~a mono
source was mistakenly recorded as stereo), you can identify the channel
in use (1 or 2), and then use\ldots{}

\begin{quote}
\emph{Objects -\textgreater{} Convert -\textgreater{} Extract one
channel\ldots{}}
\end{quote}

This will allow you to specify a channel and will output only that
channel which you specified. But it is functionally identical to using
`Extract all channels' and then ignoring the other channel's output.

\hypertarget{combining-two-mono-sounds-into-one-stereo-sound}{%
\subsubsection{Combining two mono sounds into one stereo
sound}\label{combining-two-mono-sounds-into-one-stereo-sound}}

Sometimes, you might want to create (or re-create) a stereo sound from
two component mono sounds. Perhaps the goal is to add noise to one
channel (but not the other) for binaural perception experiments, or to
modify only one channel of a stereo sound (which you've previously
extracted). To combine two mono sounds to stereo\ldots{}

\begin{enumerate}
\def\labelenumi{\arabic{enumi}.}
\tightlist
\item
  Load the two mono sounds you wish to combine into the objects window
  of Praat

  \begin{itemize}
  \tightlist
  \item
    Make sure they have the same sampling frequency (see Section
    \ref{samplingrate}) and length
  \end{itemize}
\item
  Select both sounds
\item
  \emph{Objects -\textgreater{} Combine -\textgreater{} Combine to
  Stereo}
\item
  The resulting stereo sound will be placed into your objects window
  with the name `combined'.
\end{enumerate}

Note that Praat will allow you to combine two signals of unequal length
to stereo. In this case, the files will be aligned at the start time.
But this means that a subtle change in duration (an extra few ms of
pasted pause early in the file) could result in a very unpleasant
desynchronization, which will happen without warning or notification.

Chaining this ``Combine to Stereo'' process with a ``Convert to mono''
step is one (somewhat roundabout) method of combining two sound files
into one signal. See Section \label{sub:formulas} for a much more
graceful approach to doing the same thing.

\hypertarget{cropping-copying-splicing-and-pasting}{%
\subsection{Cropping, Copying, Splicing and
Pasting}\label{cropping-copying-splicing-and-pasting}}

\label{sec:cropcopy}

It's not unusual to need to move, remove, or copy sound within and
across different sound files, and luckily, Praat makes that relatively
easy. Most file editing is done with a combination of selection, copying
and pasting.

However, producing quality splices is not as straightforward as copy and
pasting within a word document. Because sound waveforms are continuous,
you need to make sure that all the cuts you're making occur at the same
point in the cycle, namely, at the zero crossing. Failure to do so will
result in loud pops or clicks in the resulting file. To ensure that
you're being zero-crossing friendly, following the below steps
(substituting ``Ctrl'' (the ``control'' key) for ``Cmd'' (the
``command'' key) if you're using a Windows or Linux machine):

To \textbf{copy/paste} a portion of the soundfile:

\begin{enumerate}
\def\labelenumi{\arabic{enumi}.}
\tightlist
\item
  Select the portion of your soundfile that you'd like to copy
\item
  \emph{Editor -\textgreater{} Select -\textgreater{} Move start of
  selection to nearest zero crossing} or Cmd + ,
\item
  \emph{Editor -\textgreater{} Select -\textgreater{} Move end of
  selection to nearest zero crossing} or Cmd + .
\item
  \emph{Editor -\textgreater{} Edit -\textgreater{} Copy} or Cmd + c
\item
  Put the cursor where you'd like the portion to go
\item
  \emph{Editor -\textgreater{} Select -\textgreater{} Move cursor to
  nearest zero crossing} or Cmd + 0
\item
  Paste using \emph{Editor -\textgreater{} Edit -\textgreater{} Paste}
  or Cmd + p
\end{enumerate}

Following these steps will cleanly insert the snippet into the word.
Given time, you'll develop muscle memory and find yourself quickly
typing ``Cmd + , Cmd + . Cmd + c'' to copy and ``Cmd + 0 Cmd + p'' to
paste. Copy-pasting can be done either within the same file, or between
two different files in Praat.

This can also be done from the objects window using \emph{Objects
-\textgreater{} Convert -\textgreater{} Extract Part}.

To delete a portion of a file or to remove silence, you'll select, again
attending to zero crossings, and use the ``Cut'' command (without
pasting anywhere else):

To \textbf{delete} a portion of the soundfile:

\begin{enumerate}
\def\labelenumi{\arabic{enumi}.}
\tightlist
\item
  Select the portion of your soundfile that you'd like to disappear
\item
  \emph{Editor -\textgreater{} Select -\textgreater{} Move start of
  selection to nearest zero crossing} Cmd + ,
\item
  \emph{Editor -\textgreater{} Select -\textgreater{} Move end of
  selection to nearest zero crossing} or Cmd + .
\item
  \emph{Editor -\textgreater{} Edit -\textgreater{} Cut} or Cmd + X
\end{enumerate}

Unfortunately, Praat doesn't include an easy way to trim, splice or cut
portions of the file from the objects window, meaning that any scripts
will have to use GUI scripting (the computer controlling the
mouse/selection tool) on the Editor window, telling the Editor what to
select, then to cut, etc. This works well, but is slightly less
efficient than is desirable. Some progress can be made from the Objects
window alone by using a combination of \emph{Objects -\textgreater{}
Convert -\textgreater{} Extract Part} and \emph{Objects -\textgreater{}
Combine -\textgreater{} Concatenate} to, effectively, create new sounds
which resemble a cut or trimmed sound, but this can be quite
counterintuitive.

If Praat won't let you copy and paste a chunk between two files, the
files may need to be \textbf{resampled} to match.

\hypertarget{sampling-rates-and-resampling}{%
\subsection{Sampling rates and
Resampling}\label{sampling-rates-and-resampling}}

\label{samplingrate}

Praat is very picky about the sampling rates of the files it works with,
that is, the number of times per second the audio signal's power is
captured in the file. When combining sounds or copy/pasting from one
file to another, both sounds will need to have the same sampling rate.

To get the sampling rate of an existing file:

\begin{enumerate}
\def\labelenumi{\arabic{enumi}.}
\tightlist
\item
  Load the sound into the objects window of Praat
\item
  \emph{Objects -\textgreater{} Query -\textgreater{} Query Time
  Sampling -\textgreater{} Get Sampling Frequency}
\item
  An info window will pop up displaying the sampling rate.
\end{enumerate}

To \textbf{resample} a soundfile (e.g.~change the sampling rate from
44,100 Hz to 22,050 Hz):

\begin{enumerate}
\def\labelenumi{\arabic{enumi}.}
\tightlist
\item
  Load the sound into the objects window of Praat
\item
  \emph{Objects -\textgreater{} Convert -\textgreater{}
  Resample\ldots{}}

  \begin{itemize}
  \tightlist
  \item
    \emph{New Sampling Frequency} = 22050 (or whatever sampling rate
    you'd like)
  \item
    \emph{Precision (Samples)} = 50 (don't change this number)
  \end{itemize}
\item
  The filtered sound will be placed into your Objects window as ``Sound
  soundname\_(new sampling rate)''
\end{enumerate}

\hypertarget{filtering-sounds}{%
\subsection{Filtering Sounds}\label{filtering-sounds}}

Not all changes you'll want to make involve parts of files. Sometimes
(for perception experiments and otherwise), you'll find it necessary to
filter your sound files.

Often, sounds files will have extraneous background noise, or in the
case of particularly low-frequency-sensitive microphones, room noise not
blocked by sound attenuation. In those cases, you'll want to filter the
sound file to remove it:

\hypertarget{low-pass-filtering}{%
\subsubsection{Low-pass filtering}\label{low-pass-filtering}}

\textbf{Low-pass filters} are useful for simulating high-frequency
hearing loss and cellular phone speech, among other things. They remove
all signal above a given frequency.

To \textbf{low-pass filter} a soundfile (e.g.~removing all sound
\underline{above} 2000 Hz):

\begin{enumerate}
\def\labelenumi{\arabic{enumi}.}
\tightlist
\item
  Load the sound into the objects window of Praat
\item
  \emph{Objects -\textgreater{} Filter -\textgreater{} Filter (Pass Hann
  Band)}

  \begin{itemize}
  \tightlist
  \item
    \emph{From Frequency} = 0
  \item
    \emph{To Frequency} = 2000 (or whatever you'd like your highest
    frequency sound to be)
  \item
    \emph{Smoothing} = 20 Hz (20 Hz is a good baseline. This controls
    how ``soft'' the cutoff is. 5 Hz is as low as you'll ever want to go
    for this setting.)
  \end{itemize}
\item
  The filtered sound will be placed into your Objects window as ``Sound
  soundname\_band''
\end{enumerate}

\hypertarget{high-pass-filtering}{%
\subsubsection{High-pass filtering}\label{high-pass-filtering}}

\textbf{High-pass filters} are useful for removing low frequency noise
from recordings (which might seep through a sound-booth). They remove
all signal below a given frequency.

To \textbf{high-pass filter} a soundfile (e.g.~removing all sound
\underline{below} 2000 Hz):

\begin{enumerate}
\def\labelenumi{\arabic{enumi}.}
\tightlist
\item
  Load the sound into the objects window of Praat
\item
  \emph{Objects -\textgreater{} Filter -\textgreater{} Filter (Stop Hann
  Band)}

  \begin{itemize}
  \tightlist
  \item
    \emph{From Frequency} = 0
  \item
    \emph{To Frequency} = 2000 (or whatever you'd like your highest
    frequency sound to be)
  \item
    \emph{Smoothing} = 20 Hz (20 Hz is a good baseline. This controls
    how ``soft'' the cutoff is. 5 Hz is as low as you'll ever want to go
    for this setting.)
  \end{itemize}
\item
  The filtered sound will be placed into your Objects window as ``Sound
  soundname\_band''
\end{enumerate}

\hypertarget{band-pass-notch-filtering}{%
\subsubsection{Band-pass (notch)
filtering}\label{band-pass-notch-filtering}}

\textbf{Band-pass filters} (also called ``\textbf{Notch filters}'') are
useful for removing sound in a very specific band of the spectrum. Notch
filters are best suited for removing particular background noises with
specific frequencies (e.g.~computer fans, mains hum, or chair
squeaking). Your bands will be wider or narrower depending on the signal
you're working to cut out.

To \textbf{band-pass filter} a soundfile (e.g.~removing all sound
between 1500 and 3500 Hz):

\begin{enumerate}
\def\labelenumi{\arabic{enumi}.}
\tightlist
\item
  Load the sound into the objects window of Praat
\item
  \emph{Objects -\textgreater{} Filter -\textgreater{} Filter (Stop Hann
  Band)}

  \begin{itemize}
  \tightlist
  \item
    \emph{From Frequency} = 1500
  \item
    \emph{To Frequency} = 3500 (or whatever you'd like your highest
    frequency sound to be)
  \item
    \emph{Smoothing} = 20 Hz (20 Hz is a good baseline. This controls
    how ``soft'' the cutoff is. 5 Hz is as low as you'll ever want to go
    for this setting.)
  \end{itemize}
\item
  The filtered sound will be placed into your Objects window as ``Sound
  soundname\_band''
\end{enumerate}

Finally, note the \emph{Objects -\textgreater{} Filter -\textgreater{}
Filter (Formula)} option, which lets you filter sounds in a much more
specific way than the two options above provide.

\hypertarget{manipulating-spectral-tilt}{%
\subsection{Manipulating Spectral
Tilt}\label{manipulating-spectral-tilt}}

Although you could band-pass out many frequency bands, alter their
amplitudes, and then re-combine, the most efficient way is to use:

\emph{Objects -\textgreater{} Filter -\textgreater{} Filter (Formula)}

Then, you can use a formula like:

\begin{verbatim}
    self / (1 + x/100) ^ 0.2
\end{verbatim}

This
formula\footnote{Thanks to Paul Boersma and Holger Mitterer on the Praat-Users mailing list for posting and revising the formula.}
reduces the amplitude of the signal increasingly as the frequency
increases, and can be modified to further increase the amount of
spectral tilt by increasing the exponent, or decrease it by decreasing
the exponent.

\hypertarget{pitch-manipulation-to-manipulation}{%
\subsection{Pitch Manipulation (To
Manipulation\ldots)}\label{pitch-manipulation-to-manipulation}}

Praat does allow you to manipulate the speaker's pitch in
already-recorded sound files.

To create a manipulation object (which allows you to change a sound's
pitch and duration):

\begin{enumerate}
\def\labelenumi{\arabic{enumi}.}
\tightlist
\item
  Load the sound into the objects window of Praat
\item
  \emph{Objects -\textgreater{} To Manipulation\ldots{}}

  \begin{itemize}
  \tightlist
  \item
    Leave \emph{Time Step} unchanged
  \item
    Set the pitch range to 75-600 Hz
  \end{itemize}
\item
  Select the newly created ``Manipulation (Soundname)'', then
  \emph{Objects -\textgreater{} View \& Edit}
\end{enumerate}

This will open a manipulation window (like the one shown in Figure
\ref{manipulation}). It shows you the pitch track (in the center), as
well as the waveform. The blue lines on top of the waveform represent
pulses. This window allows you to modify the duration and pitch of the
sound by creating and moving pitch points. A good first step is
stylizing the pitch contour which will change the detailed contour into
something more manageable.

\emph{Manipulation -\textgreater{} Pitch -\textgreater{} Stylize Pitch}

If you still have too many points, select a few (by selecting a part of
the sound) and go to \emph{Manipulation -\textgreater{} Pitch
-\textgreater{} Remove pitch point(s)}. If you need a different pitch
point, place your cursor where you want a point and \emph{Manipulation
-\textgreater{} Pitch -\textgreater{} Add pitch point at cursor}.

You can now drag the individual green pitch points around to raise and
lower the speaker's pitch at different points throughout the word, to
great phonetic (and comedic) effect.

To save the result of your manipulations, use \emph{Manipulation
-\textgreater{} File -\textgreater{} Publish Resynthesis}, and a
pitch-modified copy of the sound will be placed in the Objects window to
be saved as usual.

\begin{figure}
  \centerline{
    \mbox{\includegraphics[width=6.0in]{manipulation.png}}
  }
  \caption{The Praat Manipulation Window\label{manipulation}}

  \end{figure}

Note as well that Praat can use either LPC resynthesis (discussed later)
or ``overlap-add'' resynthesis to create the file. ``Overlap-add'' is
actually \textbf{PSOLA (Pitch Synchronous Overlap Add) resynthesis}, and
will almost always produce more natural results.

\hypertarget{matching-the-pitch-tracks-of-two-sounds}{%
\subsection{Matching the pitch tracks of two
sounds}\label{matching-the-pitch-tracks-of-two-sounds}}

When combining sounds, or creating perception experiment stimuli, it can
be useful to match the pitch contours of two sounds. Although one can
attempt this in the Manipulation window (as described above), it's far
easier to use the below procedure to match the pitch track
automatically.

To give Sound A the same pitch pattern as Sound B:

\begin{enumerate}
\def\labelenumi{\arabic{enumi}.}
\tightlist
\item
  Load both Sound A and Sound B into the objects window of Praat
\item
  Trim either Sound A or Sound B so that they have \textbf{exactly the
  same duration}

  \begin{itemize}
  \tightlist
  \item
    Praat will not allow you to swap the pitch tiers of sound files
    which are not the same length.
  \item
    You could also use PSOLA to manipulate the duration, as described
    below.
  \end{itemize}
\item
  Select Sound B, \emph{Objects -\textgreater{} To Manipulation\ldots{}}

  \begin{itemize}
  \tightlist
  \item
    Leave \emph{Time Step} unchanged
  \item
    Set the pitch range to whatever is reasonable for that speaker.
    Usually 75-300 Hz works fine.
  \end{itemize}
\item
  Select the newly created ``Manipulation B'', then \emph{Objects
  -\textgreater{} View \& Edit}
\item
  Select Sound B, \emph{Objects -\textgreater{} Extract Pitch Tier}

  \begin{itemize}
  \tightlist
  \item
    This will give you ``PitchTier B''
  \end{itemize}
\item
  Now select Sound A, \emph{Objects -\textgreater{} To
  Manipulation\ldots{}}

  \begin{itemize}
  \tightlist
  \item
    Leave \emph{Time Step} unchanged
  \item
    Set the pitch range to whatever is reasonable for that speaker.
    Usually 75-300 Hz works fine.
  \end{itemize}
\item
  Select the newly created ``Manipulation A'' \textbf{as well as}
  ``PitchTier B''
\item
  \emph{Objects -\textgreater{} Replace Pitch Tier}
\item
  Select Manipulation A (which you just combined with PitchTier B), then
  \emph{Objects -\textgreater{} View \& Edit}
\item
  In the Manipulation Window, \emph{Manipulation -\textgreater{} File
  -\textgreater{} Publish Resynthesis}
\item
  This will export a copy of the finished version into the Objects
  window. Save it from there.
\end{enumerate}

\vspace{0.5cm}
\begin{tabular}[h]{ p{0.6in} p{12cm}}
\includegraphics[width=0.5in]{danger.png} \newline \textbf{Danger!} & \raisebox{2mm}{\parbox{13cm}{\textit{Praat’s pitch matching feature is only as effective as its pitch tracking feature, which means that both require careful manual review of the results.  Although the two sounds’ pitch tracks will be significantly closer following this step, they will not be identical, and there may be artifacts and odd jumps left over.  If you require the sounds to be exactly matched, match them both to a completely flat pitch generated by formula. (see Section \ref{creatingsounds})}}}
\end{tabular}
\vspace{0.5cm}

\hypertarget{manipulating-duration-slowing-down-and-speeding-up-sounds}{%
\subsection{Manipulating Duration (Slowing Down and Speeding Up
Sounds)}\label{manipulating-duration-slowing-down-and-speeding-up-sounds}}

Similar to modifying pitch in existing sound files, Praat allows you to
modify durations, resulting in sped up or slowed down sections of
existing files. First, again, you'll need to create a manipulation
object:

To create a manipulation object (which allows you to change a sound's
pitch and duration):

\begin{enumerate}
\def\labelenumi{\arabic{enumi}.}
\tightlist
\item
  Load the sound into the objects window of Praat
\item
  \emph{Objects -\textgreater{} To Manipulation\ldots{}} * Leave
  \emph{Time Step} unchanged * Set the pitch range to whatever is
  reasonable for that speaker. Usually 75-300 Hz works fine.
\item
  Select the newly created ``Manipulation (Soundname)'', then
  \emph{Objects -\textgreater{} View \& Edit}
\end{enumerate}

This will open a manipulation window (like the one shown in Figure
\ref{manipulation}) You'll notice that underneath the pitch manipulation
area, there's a small bar labeled ``Duration Manip'' which starts off
saying ``(no duration points)''. To speed up or slow down a sound file:

\begin{enumerate}
\def\labelenumi{\arabic{enumi}.}
\tightlist
\item
  \emph{Dur -\textgreater{} Add Duration Point at Cursor}

  \begin{itemize}
  \tightlist
  \item
    Although it says ``at cursor'', adding just one point allows you to
    manipulate duration for the whole file
  \end{itemize}
\item
  Now drag that single duration point up or down in the ``Duration
  Manip'' area to change the speed of the file

  \begin{itemize}
  \tightlist
  \item
    Dragging the point up increases the duration, slowing the sound
    down, dragging it down decreases duration and speeds the sound up.
  \end{itemize}
\end{enumerate}

You can add multiple points and selectively speed or slow certain parts
of the recording (to change the length of a vowel, for instance). In
addition, as before, Praat can export the file to a wav file, which is
again best done using ``overlap-add'' (\emph{Objects Window
-\textgreater{} Get Resynthesis (overlap-add)}).

\hypertarget{modifying-duration-by-script}{%
\subsubsection{Modifying Duration by
Script}\label{modifying-duration-by-script}}

To modify duration in a large number of files, you do \textbf{not} want
to use Praat's manipulation interface. Instead, just TextGrid your
files, and use code like the below, which would add 35 ms to a marked
vowel interval's duration:

\begin{verbatim}
    # Get vowel_end and vowel_start from a textgrid first!
    vowel_durationms = (vowel_end - vowel_start) * 1000
    duration_change = 35
    finallength = duration_change + vowel_durationms
    durfactor = finallength/vowel_durationms
    select Sound ‘soundname$’
    Lengthen (overlap-add): 60, 300, ‘durfactor’
\end{verbatim}

You could also set \texttt{durfactor} to 1.5 if you wanted to make the
vowel's new duration 1.5 times larger. Also know that `Lengthen' will
modify the sound in-situ, so you'll want to make a copy of the file.

\vspace{0.5cm}
\begin{tabular}[c c]{ p{0.7in} p{12cm}}
\includegraphics[width=0.7in]{hal.png} \newline \textbf{Script Tip!} & \raisebox{5mm}{\parbox{12cm}{\textit{Don’t be afraid to modify Pitch and Duration. Because Praat uses PSOLA, modifying duration and pitch is exceptionally clean, and won’t warp the spectral properties of the sound, so long as Praat can find and keep a good pitch track!}}}
\end{tabular}
\vspace{0.5cm}

\hypertarget{scaling-and-matching-intensity}{%
\subsection{Scaling and Matching
Intensity}\label{scaling-and-matching-intensity}}

\label{sub:matchingintensity}

The `Scale Intensity' command scales the entire file's amplitude to a
certain average level. This can be helpful if you have a file which is
too quiet to play back or when preparing stimuli for a perception
experiment. This is not absolutely precise, but for most purposes, this
will produce a good amplitude match, provided you attend carefully to
the input files.

To scale Sound A to 75 dB average intensity:

\begin{enumerate}
\def\labelenumi{\arabic{enumi}.}
\tightlist
\item
  Load Sound A into the objects window of Praat
\item
  Select Sound A, \emph{Objects -\textgreater{} Modify -\textgreater{}
  Scale intensity\ldots{}}

  \begin{itemize}
  \tightlist
  \item
    Fill in the desired value (`75') for \emph{New Average Intensity
    (dB)}
  \end{itemize}
\item
  Sound A will be modified in place, overwriting the prior version, and
  can then be saved.
\end{enumerate}

Note that this is scaling the sound's \emph{average intensity}, over the
entire file. This is crucial, as it means that the actual intensity of,
say, the vowel within a word will vary depending on the surrounding
context within the file. If a file includes several seconds of silence
to either side of the word, when set to 75dB average amplitude, the word
itself will be higher in amplitude relative to the same word in a file
trimmed to the edges of the word. Put differently, if a large proportion
of the word has an amplitude near zero, the non-zero portions will need
to be relatively louder to offset this silence. This can also present
serious problems when scaling very different words to the same setting.
Even if the silences around the words are identical, the vowel in
``faith'' will be relatively louder than the vowel in ``rail'' because
of the need to scale higher to offset the surrounding, quiet fricatives
in `faith' (where there is no such need for the liquids in
`rail')\footnote{To overcome this and ensure that the vowel is exactly 75dB, consider separating the vowel, scaling it to 75dB, and then re-combining with the surrounding word}.
So, ensure that you're conscious of context when scaling amplitude, and
realize that if perfect matches within a given segment are what you're
after, you'll need a more detailed approach.

It can also be useful, especially when splicing or sound combination is
occurring, to be able to ensure that two sounds are of the same overall
intensity.

To scale Sound A to the same average intensity as Sound B:

\begin{enumerate}
\def\labelenumi{\arabic{enumi}.}
\tightlist
\item
  Load both Sound A and Sound B into the objects window of Praat
\item
  Ensure that both are closely cropped to the boundaries of the word (or
  that there is an equal silence surrounding both)
\item
  Select Sound B, \emph{Objects -\textgreater{} Query -\textgreater{}
  Get intensity (dB)}

  \begin{itemize}
  \tightlist
  \item
    \label{note}Make note of this number, it's the average intensity of
    B
  \end{itemize}
\item
  Select Sound A, \emph{Objects -\textgreater{} Modify -\textgreater{}
  Scale intensity\ldots{}}

  \begin{itemize}
  \tightlist
  \item
    Fill in the value obtained in step \ref{note} for \emph{New Average
    Intensity (dB)}
  \end{itemize}
\item
  Sound A will be modified in place, overwriting the prior (unmatched)
  version, and can then be saved.
\end{enumerate}

\vspace{0.5cm}
\begin{tabular}[h]{ p{0.6in} p{12cm}}
\includegraphics[width=0.5in]{danger.png} \newline \textbf{Danger!} & \raisebox{2mm}{\parbox{13cm}{\textit{Again, you must ensure that both stimuli are surrounded by similar amounts of silence to ensure that the resulting words are actually roughly matched in amplitude.  This has forced me to re-make stimuli at great personal cost.  Learn from my pain.}}}
\end{tabular}
\vspace{0.5cm}

Finally, remember that scaling file amplitude to a given value does not
ensure that it plays back at that same amplitude during experiments. To
claim an exact playback amplitude, you'll need to use a calibrated
headset and precise, non-listener-controlled amplitude control.

\hypertarget{concatenating-sounds}{%
\subsection{Concatenating Sounds}\label{concatenating-sounds}}

\label{sub:concatenation}

Although you can paste the contents of one file at the beginning or end
of another within the editor window, this is terribly inefficient and
error-prone for more than one or two tokens. To directly concatenate two
or more sounds sounds, that is, to place them within a single sound file
such that one plays directly after the other, Praat offers the
\emph{concatenate} command (\emph{Objects -\textgreater{} Combine
-\textgreater{} Concatenate}), whose practical use can be rather
confusing.

This command is tricky because \emph{Concatenate} combines all currently
selected sounds \textbf{in the order which they appear in the in the
objects window}. No matter the order in which you select the sounds,
whether there are intervening non-selected sounds, the sounds' order of
creation, or their filenames, \emph{Concatenate} will simply look at the
selected Sounds, find their ordering in the objects window, and stitch
them together in that order.

To concatenate Sound A and Sound B into one file, with Sound A first:

\begin{enumerate}
\def\labelenumi{\arabic{enumi}.}
\tightlist
\item
  Load both Sound A and Sound B into the objects window of Praat
  \textbf{such that Sound A is imported first}

  \begin{itemize}
  \tightlist
  \item
    If the sounds are already loaded such that B is first, use
    \emph{Copy\ldots{}} to make a new Sound B lower in the objects
    window
  \end{itemize}
\item
  Ensure that the files contain the desired amount of empty space to
  either side of each word (as the entire files, silence and all, will
  be stitched together).
\item
  Select Sound A
\item
  Select Sound B
\item
  \emph{Objects -\textgreater{} Combine -\textgreater{} Concatenate}
\item
  A new sound, named ``Sound chain'', will be created, containing Sound
  A directly followed by Sound B.
\end{enumerate}

\vspace{0.5cm}
\begin{tabular}[c c]{ p{0.7in} p{12cm}}
\includegraphics[width=0.7in]{hal.png} \newline \textbf{Script Tip!} & \raisebox{5mm}{\parbox{12cm}{\textit{This strict 'ordering within objects window' limitation applies to scripting as well.  If, for instance, you've split the onset and coda away from the vowel, and want to re-combine them after vowel manipulation, you'll want to use 'Copy' to create a new version of the coda (which will then necessarily be the newest and last item in Objects), then Select Sound onset (which is earliest in the objects window), Plus Sound modified\_vowel, Plus Sound coda (which you've just created), then Concatenate.  This 'copy to move to bottom of objects window' hack is one of the ugliest, most just-hold-your-nose-and-code scripting tricks I regularly use.}}}
\end{tabular}
\vspace{0.5cm}

Note that you also have the option to use \emph{Objects -\textgreater{}
Combine -\textgreater{} Concatenate recoverably}, which works
identically to \emph{Concatenate}, but also creates a `TextGrid chain'
annotation showing the extent of each file within the chain file, which
can be useful to later split the files back up. \emph{Objects
-\textgreater{} Combine -\textgreater{} Concatenate with
overlap\ldots{}} performs the concatenation, specifying that the last N
seconds of the first file should overlap the first N seconds of the
second (which can be useful if, for instance, both words have 50ms of
silence, but you want a 25ms inter-stimulus interval). But if you're
attempting to combine the sounds, you should instead refer to\ldots{}

\hypertarget{combining-sounds}{%
\subsection{Combining Sounds}\label{combining-sounds}}

There are two ways to combine two sounds using Praat. The first will
sometimes work well, and works best when both sounds are to be added
equally and have the exact same file length:

To combine (overlay) Sound A and Sound B:

\begin{enumerate}
\def\labelenumi{\arabic{enumi}.}
\tightlist
\item
  Load both Sound A and Sound B into the objects window of Praat
\item
  Select Sound A and Sound B together
\item
  \emph{Objects -\textgreater{} Combine -\textgreater{} Combine to
  Stereo}

  \begin{itemize}
  \tightlist
  \item
    This will create a new sound which has Sound A in one track, and
    Sound B in the other
  \end{itemize}
\item
  Select the sound created by the last step
\item
  \emph{Objects -\textgreater{} Convert -\textgreater{} Convert to Mono}

  \begin{itemize}
  \tightlist
  \item
    This will then collapse both tracks back into a single mono track
  \end{itemize}
\end{enumerate}

This will often work, and if it does, that's wonderful. If you end up
with odd or undesirable results from this, move onto the next section
and combine the two sounds using a formula.

\hypertarget{formula-modification-waveform-addition-subtraction-and-so-much-more}{%
\subsection{Formula Modification: Waveform addition, subtraction and so
much
more}\label{formula-modification-waveform-addition-subtraction-and-so-much-more}}

\label{sub:formulas}

Praat is quite capable of doing sample-by-sample mathematical operations
on both individual files and on pairs or groups of files. This is done
using one of the most obtuse yet most powerful functions available in
Praat, \emph{Objects -\textgreater{} Modify -\textgreater{} Formula}
function.

Formula modification is best visualized by thinking about digitized
sound. Remember that when sound is digitized, the waveform itself isn't
saved, but instead, the waveform is sampled (at the sampling rate),
leaving you with a series of times and the amplitude of the wave at that
moment.

So, to add two digitized waveforms, you simply move through the file
sample-by-sample and compare each sample. During waveform addition, if,
at the 31st sample (e.g.), the amplitude of Waveform A is at 28 and
Waveform B is at -5, you add those two timepoints together (28 + (-5)),
and in your resulting sound file, the 31st sample will have an amplitude
of 23. Because this moves sample-by-sample, both sounds will need to
have the same sampling rate, and need to be of the same length.

Unfortunately, formula combination of sounds is among the least polished
of Praat's features, but don't let that scare you off. To do any formula
modification:

\begin{enumerate}
\def\labelenumi{\arabic{enumi}.}
\tightlist
\item
  Load the sound(s) you'd like to modify into the objects window of
  Praat
\item
  Select the sound you'd like to modify

  \begin{itemize}
  \tightlist
  \item
    Remember, this will modify the existing sound, not create a modified
    copy, so make sure to make a copy of the sound to work on.
  \end{itemize}
\item
  \emph{Objects -\textgreater{} Modify -\textgreater{} Formula}
\end{enumerate}

This will then pull up a very technical looking input box, pictured in
Figure \ref{formulabox}. To actually make the combination, you'll need
to put into that box the formula which will be \textbf{applied to the
amplitude value} of each individual sample.

\begin{figure}
  \centerline{
    \mbox{\includegraphics[width=4.0in]{formulabox.png}}
  }
  \caption{The Praat Formula Modification Window\label{formulabox}}

  \end{figure}

To make this process a bit more clear, let's imagine that we've got a
file called sounda and a file called soundb in our Objects window.
You've gone through, made a copy of sounda to work on, selected that
copy, and then \emph{Objects -\textgreater{} Modify -\textgreater{}
Formula}. Here are several formulas you could put into that window, and
the effect that each of them would have:

\begin{itemize}
\tightlist
\item
  \texttt{0}

  \begin{itemize}
  \tightlist
  \item
    If you just put a ``0'' into the formula box, Praat would read this
    as ``the amplitude of the sample in question in the selected sound
    equals 0'' (self {[}col{]} = 0). Every sample would then be set to
    an amplitude of zero, and the resulting file would be completely
    silent.
  \end{itemize}
\item
  \texttt{4}

  \begin{itemize}
  \tightlist
  \item
    If you just put a ``4'' into the formula box, Praat would read this
    as ``the amplitude of the sample in question in the selected sound
    equals 4'' (self {[}col{]} = 4). Every sample would then be set to
    an amplitude of 4, and the resulting file would be constantly at
    four, \emph{and would likely stress or damage your headphones or
    speakers if played loudly}.
  \end{itemize}
\item
  \texttt{self [col]}

  \begin{itemize}
  \tightlist
  \item
    Praat would read this as ``the amplitude of the sample in question
    in the selected sound equals the amplitude of the sample in question
    in the selected sound'' (self{[}col{]} = self {[}col{]}). As such,
    it would not modify the sound in any way, and would be a rather
    complete waste of your time.

    \begin{itemize}
    \tightlist
    \item
      *In formulas, you'll refer to (sound) {[}col{]} frequently, which
      just means ``each individual sample in the sound''. ``self'' is
      the sound selected when you opened the formula window\}.
    \end{itemize}
  \end{itemize}
\item
  \texttt{self [col]*2}

  \begin{itemize}
  \tightlist
  \item
    Praat would read this as ``the amplitude of the sample in question
    in the selected sound equals two times the amplitude of the sample
    in question in the selected sound'' (self{[}col{]} = self
    {[}col{]}*2). This formula would double the amplitude of every
    sample, making the sound file twice as loud.
  \end{itemize}
\item
  \texttt{Sound\_soundb[col]}

  \begin{itemize}
  \tightlist
  \item
    Praat would read this as ``the amplitude of the sample in question
    in the selected sound equals the amplitude of the sample in question
    in the sound file called ``soundb''. This formula would turn sounda
    into a sample-by-sample copy of soundb.

    \begin{itemize}
    \tightlist
    \item
      \emph{Note that in formulas, other sounds in the object window
      need to be referred to as Sound\_soundname, not just soundname}.
    \end{itemize}
  \end{itemize}
\item
  \texttt{self [col] + Sound\_soundb [col]}

  \begin{itemize}
  \tightlist
  \item
    Praat would read this as ``the amplitude of the sample in question
    in the selected sound equals the amplitude of the sample in question
    in the selected sound \textbf{plus} the amplitude of the sample in
    question in the sound file called ``soundb''''. This formula would
    \textbf{add sounda and soundb using waveform addition}.
  \item
    \emph{When adding, subtracting, or multiplying sounds, you'll
    usually want to match the resulting sound's amplitude to that of the
    original sound afterwards (see section
    \ref{sub:matchingintensity})}.
  \end{itemize}
\item
  \texttt{self [col] - Sound\_soundb [col]}

  \begin{itemize}
  \tightlist
  \item
    Praat would read this as ``the amplitude of the sample in question
    in the selected sound equals the amplitude of the sample in question
    in the selected sound \textbf{minus} the amplitude of the sample in
    question in the sound file called ``soundb''''. This formula would
    \textbf{subtract soundb from sounda using waveform subtraction}.
  \end{itemize}
\item
  \texttt{self [col] + (2* Sound\_soundb [col])}

  \begin{itemize}
  \tightlist
  \item
    This formula would \textbf{add twice the amplitude of soundb to
    sounda using waveform addition}. Such a formula would be useful for
    combining two sounds when you want the acoustical features of one
    sound to slightly overwhelm those of the other.
  \end{itemize}
\end{itemize}

Also note that `x' can be used to represent the time (in seconds) within
the formula, allowing formulas like:

\begin{itemize}
\tightlist
\item
  \texttt{self [col] + ((x)*Sound\_soundb [col])}

  \begin{itemize}
  \tightlist
  \item
    This formula will add sounda and soundb \textbf{such that soundb's
    amplitude is directly linked to the time in seconds}. This is very
    niche, but would allow you to mix sounds in such a way that one
    `fades in' or `fades out' over time.
  \end{itemize}
\item
  \texttt{self [col] + (Sound\_soundb [col] * (x - xmin) / (xmax - xmin))}

  \begin{itemize}
  \tightlist
  \item
    This formula will directly link Sound B's amplitude to time, such
    that it is absent at the start of the sounds, and at amplitude 1 at
    the end.
  \end{itemize}
\end{itemize}

Using the above examples as a template, you can create a formula to do
nearly anything you'd like to your sound waveform, both through the user
interface or using Praat scripting.

\hypertarget{one-bit-requantization}{%
\subsubsection{One-Bit Requantization}\label{one-bit-requantization}}

Some people use one-bit Requantization to modify sounds such that
temporal information is preserved, but no frequency or amplitude
information is (resulting in something which is a bit like a very low
low pass filtered signal, but without permitting the remaining signal to
vary in amplitude). The basic idea is that any portion of the sound with
an amplitude greater than zero is set equal to zero, and any portion
less than 0 is set equal to negative one. Thus, the signal has a bit
depth of one (zero or one, on or off) and there are only two possible
amplitude states. This can be done in Praat using the formula:

\texttt{if self>0 then 0 else -1 fi}

This can be used for stimulus preparation (although the resulting
stimuli are acoustically horrifying), but also provides a nice
demonstration of the usage of the usage of if/then statements in
formulas. You'll want to scale amplitudes afterwards, ideally even
before listening, but this is an effective way of producing an
awful-sounding yet interesting token.

\hypertarget{synthesizing-sounds-from-scratch}{%
\subsection{Synthesizing Sounds from
scratch}\label{synthesizing-sounds-from-scratch}}

Praat offers the ability to synthesize sounds using a Klatt Synthesizer,
an articulatory synthesizer, the Vowel Editor and more. The use of these
synthesizers is outside the scope of this class, but is well explained
in the Praat Documentation.

\hypertarget{source-filter-vowel-resynthesis}{%
\subsection{Source-Filter Vowel
Resynthesis}\label{source-filter-vowel-resynthesis}}

When creating stimuli for vowel perception experiments (as well as many
other times), it can be useful to generate vowels of a controlled
quality. Although Praat includes both articulatory and cascade (Klatt)
synthesis, as well as the vowel editor, sometimes, it's important to
maintain the vocal characteristics of the recorded speaker. In these
situations, you'll want to use Source-Filter vowel resynthesis to alter
the vowel's formant qualities without altering or replacing other
significant aspects of the signal. Source-filter resynthesis is most
efficiently done by script, but can be done step-by-step by hand if
needed.

To understand both the process and the workings of source-filter
resynthesis (SFR), it's important to understand the source-filter theory
of vowel production. This understanding of vowel production holds that a
given vowel is composed of two element: the source (the voicing coming
from the larynx), and the filter (the articulations and anatomy of the
vocal tract above the larynx). When the source signal passes through the
vocal tract, the resonances in the mouth heighten some frequencies and
damp others. In this way, a relatively unremarkable voicing spectrum,
passed through a vocal tract with an /i/ tongue shape, ends up with
formants at 250 Hz, 2500 Hz, and 3000 Hz (your resonances may vary).

Source filter resynthesis takes advantage of this idea to modify vowel
qualities by taking the following steps, in the abstract:

\begin{enumerate}
\def\labelenumi{\arabic{enumi}.}
\tightlist
\item
  Take a recorded vowel and locate the overall peaks and valleys in the
  spectrum (the formants) by using an LPC (linear predictive coding)
  algorithm

  \begin{itemize}
  \tightlist
  \item
    These peaks and valleys, at least theoretically, should represent
    the resonances in the mouth caused by a given tongue shape
  \end{itemize}
\item
  Use this information to reconstruct the voicing signal (the source)
  without those peaks and valleys

  \begin{itemize}
  \tightlist
  \item
    This is accomplished by inverse-filtering the signal with the LPC,
    raising the parts of the spectrum which the LPC says are low, and
    lowering the parts which the LPC says are high. The end result,
    ideally, will be the source signal as if the person had no vocal
    tract at all.
  \end{itemize}
\item
  Alter the LPC to change the positions or bandwidths of the formants to
  your desired characteristics

  \begin{itemize}
  \tightlist
  \item
    By doing this, you modify the ``filter'', effectively changing the
    tongue-shape and associated resonances used to initially produce the
    vowel
  \end{itemize}
\item
  Filter the reconstructed source (created in Step 2) using the altered
  LPC (from Step 3)
\end{enumerate}

Performing \textbf{Source-Filter Resynthesis of a vowel} in Praat:

\begin{enumerate}
\def\labelenumi{\arabic{enumi}.}
\tightlist
\item
  Isolate a vowel in a single sound file (we'll call it ``vowela'')

  \begin{itemize}
  \tightlist
  \item
    Any vowel will do, but you'll get better results if you downsample
    the vowel first.
  \end{itemize}
\item
  Select that vowel, then \emph{Objects -\textgreater{} Formants \& LPC
  -\textgreater{} To LPC \ldots{}}

  \begin{itemize}
  \tightlist
  \item
    Choose whatever algorithm you prefer. Burg is my personal favorite.
  \end{itemize}
\item
  Select both the vowel and the LPC object that's been created
  (\emph{Sound vowela} and \emph{LPC vowela}), then \emph{Objects
  -\textgreater{} Filter (inverse)}

  \begin{itemize}
  \tightlist
  \item
    This will create the ``neutral'' source voicing from that vowel
  \end{itemize}
\item
  Select the LPC, then \emph{Objects -\textgreater{} To Formant}

  \begin{itemize}
  \tightlist
  \item
    Praat won't let you edit an LPC object, but you can easily edit a
    formant object
  \end{itemize}
\item
  To change formant frequencies, select the Formant object, then
  \emph{Objects -\textgreater{} Modify -\textgreater{} Formula
  (frequencies)\ldots{}}
\end{enumerate}

\begin{itemize}
\tightlist
\item
  You'll edit this by formula (see Section \ref{sub:formulas}), but it
  will always have the form
  \texttt{if row = [formant number] then self [modification] else self fi}.
  So, \texttt{if row = 1 then self + 100 else self fi} would raise the
  first formant by 100 Hz.

  \begin{itemize}
  \tightlist
  \item
    \emph{Changes less than 20 Hz are tough to spot and measure, and may
    not be particularly precise.}
  \end{itemize}
\item
  To change formant bandwidths, select the Formant object and the
  ``source'' generated in step 3 and \emph{Objects -\textgreater{}
  Filter}

  \begin{itemize}
  \tightlist
  \item
    It's \emph{very} easy to over-thin bandwidths, especially when doing
    this by script on tokens with variable bandwidths. When this
    happens, you'll see no formant, and just get a sort of
    chirpy-sounding prominence. Be cautious that you're not accidentally
    calling for a 5 Hz wide formant!
  \end{itemize}
\end{itemize}

This will output your new vowel, ideally, with the new formants. That
said, this is a very finicky process with lots of room for error, and it
will require some work to get working cleanly. This is also a situation
where using a good Praat script can speed things up significantly, and
allow the rapid repetition of small changes to improve output.

Given some work, though, and some time, this can be an excellent way to
modify vowel qualities. For examples of this in use for stimulus
preparation, contact the author.

\vspace{0.5cm}
\begin{tabular}[c c]{ p{0.7in} p{12cm}}
\includegraphics[width=0.7in]{hal.png} \newline \textbf{Script Tip!} & \raisebox{5mm}{\parbox{12cm}{\textit{Resynthesis works best with downsampled sounds, and introduces some artifacts.  You will almost certaintly want to modify only the bottom 3500 Hz of the vowel, and then re-combine it with the unmodified higher frequencies, so that you change what you need to, but keep the rest pristine. }}}
\end{tabular}
\vspace{0.5cm}

\hypertarget{tips-for-source-filter-vowel-resynthesis}{%
\subsubsection{Tips for Source-Filter Vowel
Resynthesis}\label{tips-for-source-filter-vowel-resynthesis}}

\begin{enumerate}
\def\labelenumi{\arabic{enumi})}
\item
  \textbf{The best SFVR is none at all!} If you can find or make tokens
  in other ways, they're often better, and recording a variety of tokens
  and cherry picking ones with the formant values expected may be more
  productive for many designs. This is, in many ways, a technique of
  last resort.
\item
  \textbf{Use many small steps.} You'll generally get better results
  from 5 50Hz steps than one 250Hz step. It's counterintuitive as you'd
  think processing errors would stack, but for whatever reason, it's
  been the case that you want to move slowly, at least last time I spent
  much time with this.
\item
  \textbf{Downsampling is key.} Use the instructions above for
  downsampling and your life will be better.
\item
  \textbf{If the F$_{0}$ track is bad, the token is too}. To deconvolve
  source and filter, you'll need a strong F$_{0}$ track. Strange results
  might be because Praat's struggling to isolate F$_{0}$, or worse still,
  struggling to accurately track it. You might be able to adjust F$_{0}$
  settings some to get closer, but you'll want nice modal tokens with
  clean-ish pitch tracks as input.
\end{enumerate}

\hypertarget{exporting-images-for-use-and-publication}{%
\section{Exporting images for use and
publication}\label{exporting-images-for-use-and-publication}}

\label{sec:pictures}

To be completely honest, the fastest (and usually sufficient) means of
getting images from Praat for documents or presentation is arranging the
windows to display what you'd like, and then taking a screenshot, which
can later be cropped.

However, with a bit more work, Praat can be used to create and annotate
publication quality graphs. The \textbf{Praat Picture window} (Figure
\ref{picture}) is used to create and display images.

\begin{figure}
  \centerline{
    \mbox{\includegraphics[width=4.00in]{picture.png}}
  }
  \caption{The Praat Picture Window \label{picture}}
  
  \end{figure}

Using the Praat picture window can be thought of as a five step process:
Create an object, choose your size, draw your object into the picture
window, garnish, and export. To give a brief example, let's say you'd
like to export a spectrogram of sounda:

\begin{enumerate}
\def\labelenumi{\arabic{enumi}.}
\tightlist
\item
  Select sounda, \emph{Objects -\textgreater{} Spectrum -\textgreater{}
  To Spectrogram\ldots{}}

  \begin{itemize}
  \tightlist
  \item
    Set the Spectrogram settings using the same settings you would in
    the Editor window (see Section \ref{subsec:spectrogramsettings})
  \end{itemize}
\item
  Click and drag in the picture window to set the size you'd like.

  \begin{itemize}
  \tightlist
  \item
    Realize that these images scale nicely, so the default usually
    suffices. Also, you can create more than one graphic per picture
    window, just change the selection area.
  \end{itemize}
\item
  Select the newly created Spectrogram object, then \emph{Objects
  -\textgreater{} Draw -\textgreater{} Paint\ldots{}}

  \begin{itemize}
  \tightlist
  \item
    Set these parameters as you'd like.
  \end{itemize}
\item
  Use the \emph{Picture -\textgreater{} Margins} menu to add labels,
  text, and other garnishes
\item
  Use \emph{Picture -\textgreater{} File -\textgreater{} Save as PDF
  file\ldots{}} to export your picture
\end{enumerate}

\begin{itemize}
\tightlist
\item
  Be warned, the resulting PDFs will have very large file sizes as they
  capture a great deal of detail. As such, they can be expanded and
  shrunken quite gracefully, but will make for some massive slideshow
  files.
\end{itemize}

Note that there's no need to include a spectrogram here, you can just as
readily create a plot featuring only a pitch contour or formant trace.
But often, a spectrogram is useful for delimiting . There is obviously
much more complexity possible, but repeating (and adapting) the steps
will yield wonderful graphs for any publication. This entire process is
quite easily scriptable, as well.

For additional information about using Praat pictures, consult Jennifer
Smith's wonderful \emph{Printing and copying/pasting Praat Images}
handout, available at
\url{http://www.unc.edu/~jlsmith/ling120/pdf/5images.pdf}.

\begin{figure}
  \centerline{
    \mbox{\includegraphics[width=6.00in]{400c.pdf}}
  }
  \caption{An example of an exported, garnished waveform plot of a 400 Hz pure tone \label{400c}}
  
  \end{figure}

\vspace{0.5cm}
\begin{tabular}[h]{ p{0.6in} p{12cm}}
\includegraphics[width=0.5in]{happyretroflex.png} \newline \textbf{Bonus!} & \raisebox{5mm}{\parbox{13cm}{\textit{By creating sounds from formula (see Section \ref{creatingsounds}) and combining them together using waveform addition and subtraction (see Section \ref{sub:formulas}), it’s a breeze to create the sorts of lightweight-yet-accurate waveform graphs needed when teaching and testing students on the fundamentals of acoustics and waveform addition.}}}
\end{tabular}
\vspace{0.5cm}

\hypertarget{creating-complex-displays}{%
\subsection{Creating Complex Displays}\label{creating-complex-displays}}

Using the Praat picture window as a canvas, you can add more detail to a
graph than is available from a single command by overlaying plots and
combining multiple plots into a larger layout.

\hypertarget{overlaying-plots}{%
\subsubsection{Overlaying Plots}\label{overlaying-plots}}

You can use the Praat picture window to generate plots with multiple
data types on them at once. For instance, if you wanted to overlay a
pitch trace onto a spectrogram, you would simply draw a spectrogram (as
above), and then, without changing your selection in the window, create
a pitch object and draw it, with the same ``Time'' boundaries as for the
spectrogram. You can also overlay a TextGrid's content to label your
data.

Although it may not always make sense to do so, you use a series of draw
commands to overlay \textbf{any} graphics. To do this, complete the same
steps above and then repeat the ``draw'' process for each element,
changing the color and width of lines using \emph{Picture
-\textgreater{} Pen} before each draw to help multiple plots stay
distinguishable.

Doing this, you can represent as many different types of data on a
single plot as you'd like, up to and well beyond the possibility of any
readability.

\hypertarget{multiple-plots-in-the-picture-window}{%
\subsubsection{Multiple Plots in the Picture
Window}\label{multiple-plots-in-the-picture-window}}

Similarly, if you wanted to draw a waveform above a spectrogram in the
picture window (similar to the editor display), you would:

\begin{enumerate}
\def\labelenumi{\arabic{enumi}.}
\tightlist
\item
  Select the portion of the picture window which will have your waveform
\item
  Select a sound, then \emph{Objects -\textgreater{} Draw
  -\textgreater{} Draw\ldots{}}

  \begin{itemize}
  \tightlist
  \item
    Edit the parameters to ``zoom'' to the portion you need
  \end{itemize}
\item
  Use the \emph{Picture -\textgreater{} Margins} menu to add labels,
  text, and other garnishes
\item
  Select the still-blank portion of the picture window which you want to
  contain the spectrogram
\item
  Select the sound, \emph{Objects -\textgreater{} Spectrum
  -\textgreater{} To Spectrogram\ldots{}}

  \begin{itemize}
  \tightlist
  \item
    Set the Spectrogram settings using the same settings you would in
    the Editor window (see Section \ref{subsec:spectrogramsettings})
  \end{itemize}
\item
  Select the newly created Spectrogram object, then \emph{Objects
  -\textgreater{} Draw -\textgreater{} Paint\ldots{}}

  \begin{itemize}
  \tightlist
  \item
    Set these parameters as you'd like.
  \end{itemize}
\item
  Use the \emph{Picture -\textgreater{} Margins} menu to add labels,
  text, and other garnishes
\item
  Use the Picture window selection to surround \emph{everything} in the
  window for export
\item
  Use \emph{Picture -\textgreater{} File -\textgreater{} Save as PDF
  file\ldots{}} to export your picture
\end{enumerate}

By doing this, you can create, entirely within Praat, a whole matrix of
plots which show all the various facets of a given sound, all without
resorting to external photo editors.

\begin{figure}
  \centerline{
    \mbox{\includegraphics[width=6.00in]{excess.pdf}}
  }
  \caption{You can now combine, overlay, and juxtapose plots to excess, as in the above single Praat picture}
  
  \end{figure}

\hypertarget{annotating-sound-files-praat-textgrids}{%
\section{Annotating Sound Files (Praat
TextGrids)}\label{annotating-sound-files-praat-textgrids}}

\label{sec:TextGridding}

Praat can't reliably tell where one word starts and where another ends.
Nor can it find the specific segment you're looking for, nor identify
the vowel in the word. As such, you'll often need to segment sound files
with that information when using any sort of automated measurements. In
Praat, this is done by creating \textbf{TextGrid} annotations in a
TextGrid file, which is saved separately from the sound itself.

TextGrid annotations are composed of different tiers which mark either
intervals or specific points within the sound file. Interval tiers are
designed to mark elements of a file with a distinct span, like a vowel,
word, or segment. Point tiers mark single points, like release bursts,
turn changes, or glottal openings. Both intervals and points can (and
should) be labeled. The names and relative position of these tiers are
specified when the TextGrid is created.

It's worth noting that Praat counts EVERY interval in the file, not just
human-created or labeled ones, so if you mark and label a single vowel
interval in a file, Praat will consider the file to have three
intervals: the section before the start of the vowel, the marked vowel,
and the section after the vowel. The marked vowel, in this situation,
would be considered interval number 2.

To create a TextGrid for a sound file\ldots{}

\begin{enumerate}
\def\labelenumi{\arabic{enumi}.}
\tightlist
\item
  Open and then select the sound that you want to TextGrid
\item
  \emph{Objects -\textgreater{} Annotate -\textgreater{} To
  TextGrid\ldots{}}
\item
  Name your tiers, specifying on the following line which tiers, if any,
  are point tiers. Tiers will be ordered in the order named.
\item
  Select both the sound you'd like to annotate and the associated
  TextGrid, and click \emph{Objects -\textgreater{} View \& Edit}
\end{enumerate}

Once your TextGrid is created, you'll be presented with the
\textbf{TextGrid Editor
Window}\footnote{Although the window which is opened when a sound and TextGrid are both selected and \textit{View \& Edit}ed looks a lot like the Editor window, Praat considers it to be a very different sort of window.  Any scripts added to the Editor window (see Section \ref{sec:scripting}) will have to be added separately to the TextGrid editor window, and some menu options present in the Editor window are not present in the TextGrid editor window.}
(Figure \ref{grideditor}). To annotate the file\ldots{}

\begin{enumerate}
\def\labelenumi{\arabic{enumi}.}
\setcounter{enumi}{4}
\tightlist
\item
  Click on the tier you'd like to make an interval on
\item
  Using your mouse, select the part of the word you'd like the interval
  to contain
\item
  Hit the key
\item
  Click on the interval you've just created and name it

  \begin{itemize}
  \tightlist
  \item
    You can use IPA characters in TextGrid labels (provided you have the
    proper Unicode IPA fonts installed), but when exporting the labels
    by script or elsewhere, all programs used for analysis must be
    Unicode aware. Most modern programs are, but many command-line
    programs (such as SPSS or Python 2.X) are not natively or
    straightforwardly happy in Unicode, so if you're planning to use any
    non-Unicode aware programs, you're best served using SAMPA or some
    other means of transliterating IPA characters into ASCII text.
  \end{itemize}
\item
  Repeat for other intervals on the same tier, as well as any other
  tiers

  \begin{itemize}
  \tightlist
  \item
    Points are created by clicking on a point tier, placing the cursor
    where you'd like the point, and hitting .
  \end{itemize}
\end{enumerate}

Once your file has been TextGridded as above (looking something like
Figure \ref{completedgrid}), you'll want to save the TextGrid file
(either from the Objects window (\emph{Objects -\textgreater{} Save
-\textgreater{} Save as Text File\ldots{}}) or within the TextGrid
editor (\emph{TextGrid Editor -\textgreater{} File -\textgreater{} Save
TextGrid as Text File\ldots{}}).

TextGridded files can then be read in by Praat scripts which measure
only certain parts of the word, can be split and labeled according to a
given tier by script (see the \texttt{file\_segmenter.praat} script), or
can simply be examined with the benefit of the labels. Once all your
files have been TextGridded, you'll be in a much better position to
start automatically measuring data and manipulating your sound files.

\begin{figure}
  \centerline{
    \mbox{\includegraphics[width=5.00in]{grideditor.png}}
  }
    \caption{The Praat TextGrid Editor Window \label{grideditor}}
  
  \end{figure}

\begin{figure}
  \centerline{
    \mbox{\includegraphics[width=5.00in]{completedgrid.png}}
  }
    \caption{A completed TextGrid annotation showing “Vowel” and “Word” tiers\label{completedgrid}}
  
  \end{figure}

\hypertarget{using-log-files}{%
\section{Using Log Files}\label{using-log-files}}

So far, all discussion of measurement has assumed you would be taking
individual hand measurements and writing them down on your own. These
last two sections describe two ways to more efficiently capture data
from your measurements.

Praat has the ability to easily capture the measurements you take as you
move through your data using log files, which are then easily exported
into Excel (or the statistics program of your choosing). This is useful
when doing a series of repetitive measures which still require human
intervention, but in many cases, the use of Praat scripting can speed
this process up further and in some cases, eliminate the need for human
intervention altogether.

To create and use a log
file:\footnote{Large parts of this section are adapted from a handout by Dr. Rebecca Scarborough for a previous LSA institute.  All credit belongs to her.}

\begin{enumerate}
\def\labelenumi{\arabic{enumi}.}
\item
  Open a sound in the Editor window
\item
  \emph{Editor -\textgreater{} Query -\textgreater{} Log
  Settings\ldots{}}
\item
  Choose a location and file name for \emph{Log file 1} or \emph{Log
  file 2}

  \begin{itemize}
  \tightlist
  \item
    On a Windows machine, this will look like C:\textbackslash Documents
    and Settings\textbackslash All
    Users\textbackslash Desktop\textbackslash Log.txt
  \item
    On a Mac, this will look like /Users/yourname/Desktop/log.txt
  \end{itemize}
\item
  Fill in \emph{Log 1 Format} and/or \emph{Log 2 Format} according to
  what you'd like to measure:

  \begin{itemize}
  \tightlist
  \item
    \textbf{For duration} (include the single quotes):

    \begin{verbatim}
                  ‘t1:4’ ‘tab$’ ‘t2:4’ ‘tab$’ ‘dur:6’
              \end{verbatim}

    \begin{itemize}
    \tightlist
    \item
      This will give you the start time (t1) and end time (t2) of a
      selection and its duration (dur). (The numbers after the colons
      specify the number of decimal places.)
    \end{itemize}
  \item
    \textbf{For formants} (include the single quotes):

    \begin{verbatim}
                  ‘t1:4’ ‘tab$’ ‘f1:0’ ‘tab$’ ‘f2:0’’tab$’ ‘f3:0’
              \end{verbatim}

    \begin{itemize}
    \tightlist
    \item
      This will give you the start time (t1) and first three formant
      frequencies (F1, F2, F3) at the cursor point (like a formant
      listing for 3 formants).
    \end{itemize}
  \item
    \textbf{For pitch} (include the single quotes):

    \begin{verbatim}
                  ‘time:4’ ‘tab$’ ‘f0:2’
              \end{verbatim}

    \begin{itemize}
    \tightlist
    \item
      This will give you the time (time) and fundamental frequency
      (\(F_0\)) at the cursor point (like a pitch listing).
    \end{itemize}
  \end{itemize}
\item
  Close the Log Settings window
\item
  You can now go back to the editor window and record duration, formant
  frequencies, or F$_{0}$ by simply selecting the relevant point or selection
  in the waveform (or spectrogram) and hitting F12 (for log 1) or
  Shift-F12 (for log 2).
\item
  By default, each measurement will display in an Info window as well as
  write to the file you set up. If you want, you can log to the log file
  or the Info window only by changing the settings for \emph{Write log 1
  to:} or \emph{Write log 2 to:} in the \emph{Editor -\textgreater{}
  Query -\textgreater{} Log Settings\ldots{}} window.
\item
  You can modify logs to include any compatible bits of information
  using the following variables: * \texttt{time} : Time at cursor or at
  midpoint of the current selection * \texttt{t1} and \texttt{t2} : Time
  at beginning and end of selection * \texttt{dur} : Duration of
  selection * \texttt{f0} : F$_{0}$ at cursor or average F$_{0}$ of selection *
  \texttt{f1}-\texttt{f5} : Formant at cursor or average for selection *
  \texttt{b1}-\texttt{b5} : Formant bandwidth at cursor or average for
  selection * \texttt{intensity} : Intensity at cursor or average for
  selection * \texttt{tab\$} : tab
\end{enumerate}

\hypertarget{scripting-in-praat}{%
\section{Scripting in Praat}\label{scripting-in-praat}}

\label{sec:scripting}

\hypertarget{what-is-praat-scripting}{%
\subsection{What is Praat scripting?}\label{what-is-praat-scripting}}

Much like a movie script tells a set of actors exactly what to say and
what actions to take, a Praat script is a text file which tells Praat
exactly what to do, and walks the program through a series of steps.
Anything that you can do through the GUI of the program (GUI is
``graphical user interface'', the menus and buttons you use to interact
with Praat), you can command Praat to do in a script.

Praat scripting is a wonderful tool, then, for situations where you find
yourself repetitively doing the same tasks over and over again, and with
increased sophistication, you can have Praat make simple decisions based
on its measurements and changes. Simply put, Praat scripting is a way to
use the computer as a co-pilot, handling the boring, repetitive tasks
independently and allowing you to do your job more efficiently.

Praat scripts can be as simple as a single line which tells Praat to
adjust the Spectrogram settings to display a narrowband spectrogram, or
as complicated as 500+ lines of code which allow Praat to go through a
folder of files, measuring 33 acoustical correlates of nasality for each
file and presenting any suspect measurements for human confirmation.

Here's a very basic Praat script, which shows a series of commands to
Praat, with commented lines (prefaced with \#) to explain what it's
doing:

\begin{verbatim}
# This is a comment, Praat ignores lines that start with #

select Sound untitled
# Plays the sound
Play
# Gets the duration
Get total duration
# Gets the amplitude
Get intensity (dB)
# Renames it
Rename... My_awesome_sound
# Prints a message into the Praat information window
print “Script Finished”
\end{verbatim}

Note, though, that Praat scripting isn't the same as writing a new
computer program. Praat scripting uses the same commands as you use in
the GUI, and only works within Praat, whereas coding (say, in C) is much
more opaque, but programs in C can be compiled to run on many different
devices and don't need specific programs.

That said, there are a few things that Praat scripting cannot do:

\begin{itemize}
\tightlist
\item
  Praat scripts can't label your data, mark linguistic units, or
  recognize speech

  \begin{itemize}
  \tightlist
  \item
    Praat doesn't know what's being said, so you need to TextGrid
    annotate your files such that Praat knows where to look for the
    vowel(s) or consonant(s) or syllable(s) that you're interested in.
  \end{itemize}
\item
  Praat scripts only run in Praat (although the \emph{system} command
  can control the OS and some other programs)

  \begin{itemize}
  \tightlist
  \item
    Praat scripting only affects Praat, so you can't incorporate a step
    which, say, opens another program to run it.
  \end{itemize}
\item
  Praat scripts can't generate measurements as consistent as hand
  measurements

  \begin{itemize}
  \tightlist
  \item
    Individual words can be odd. Humans can make small adjustments
    (taking the pitch measurement just to the right of that bit of
    creak), but Praat doesn't know enough to do that.
  \end{itemize}
\item
  Praat scripts can't sanity-check their data

  \begin{itemize}
  \tightlist
  \item
    You know fully well that F1 for /i/ is unlikely to be at 8,000 Hz.
    Praat doesn't. You can build in safeguards, but Praat will allow all
    sorts of craziness if left to its own devices.
  \end{itemize}
\item
  Praat scripts can't do anything that you couldn't eventually do
  through the user interface of Praat.

  \begin{itemize}
  \tightlist
  \item
    Praat scripting depends on Praat's built-in commands, so you can't
    do something that the program was not designed to do. That said, you
    can string many commands together to create new commands and
    processes, and you can certainly make things easier for yourself.
  \end{itemize}
\end{itemize}

To do anything vaguely scripty in Praat, you'll use the Praat menu (in
the menubar on a Mac, at the top of the objects window on a PC), and
choose \emph{New Praat Script} to open the scripting editor.

\hypertarget{praat-scripting-alternatives-for-common-tasks}{%
\subsubsection{Praat Scripting Alternatives for common
tasks}\label{praat-scripting-alternatives-for-common-tasks}}

\label{scriptingalternatives}

Praat script is a wonderful thing, and combines the speed of scripting
with the turned accuracy of measurement that the Praat GUI can get you.
However, for some batch tasks, you should consider alternatives to Praat
scripting which may be more powerful, easier, or more practical.

Based on the author's own work and experience, some tasks are better
conducted in other (free) programs. So, if you're planning to\ldots{}

\begin{itemize}
\tightlist
\item
  Batch convert files to and from multiple file formats

  \begin{itemize}
  \tightlist
  \item
    iTunes (Cost-free, but not free-as-in-freedom) can accomplish many
    conversion tasks, using \emph{iTunes -\textgreater{} Preferencces
    -\textgreater{} General Preferences -\textgreater{} Import Settings}
    and then using \emph{File -\textgreater{} Convert}.

    \begin{itemize}
    \tightlist
    \item
      Remember, friends don't let friends save phonetic data in lossy
      formats (e.g.~.mp3, .AAC, .wma)!
    \end{itemize}
  \item
    SoX (\url{http://sox.sourceforge.net}) is a free command-line
    utility which can convert and modify a variety of file formats,
    including many not readable by Praat, particularly once combined
    with ffmpeg.
  \end{itemize}
\item
  Run statistics within the script itself

  \begin{itemize}
  \tightlist
  \item
    Use PraatR (\url{http://www.aaronalbin.com/praatr/index.htm}), an
    implementation of Praat scripting commands within the R Statistics
    environment, that allows you to write R scripts which call commands
    within Praat. It's a wonderful thing, and for many uses, can replace
    Praat scripting altogether.

    \begin{itemize}
    \tightlist
    \item
      With this kind of power, you can run real-time regressions on
      formant data, and sanity check measures that way, as the author of
      PraatR does. It's brilliant.
    \end{itemize}
  \end{itemize}
\item
  Bulk find-and-replace or delete within Textgrid files (e.g.~``Unlabel
  all intervals with the label ``V'''' or ``Change every instance of
  ``spot'' to ``spat'''')

  \begin{itemize}
  \tightlist
  \item
    Use a plaintext editor, and open the Textgrid files directly,
    editing using find-and-replace. Some bulk editors like Textmate
    allow you to modify all files in a folder in the same way. Textgrids
    are just text files with lots of numbers, and can be edited easily.
  \end{itemize}
\item
  Create vowel charts based on vowel formant data

  \begin{itemize}
  \tightlist
  \item
    Use the \texttt{vowels} package in the R Statistics Environment. It
    makes pretty vowel charts easily (once the data is properly
    formatted), and takes care of all the little things (reversing axes,
    etc). It also incorporates various optional algorithmic vowel
    normalization methods, if you choose to implement them.

    \begin{itemize}
    \tightlist
    \item
      Friends don't let friends normalize vowels algorithmically without
      a firm understanding of what such methods can and cannot do. With
      great power comes great responsibility.
    \end{itemize}
  \end{itemize}
\item
  Manipulate files on your computer

  \begin{itemize}
  \tightlist
  \item
    Praat \emph{can} create, delete, and modify files on your drive. But
    by and large, your best option is to use some sort of machine-native
    scripting language (like shell script) or Python to conduct these
    tasks. Praat is more than capable of working with files, but you'll
    experience pain reading in files with multiple ``.'\,' characters
    per name, spaces in names, etc if you do this in Praat.
  \end{itemize}
\item
  Run a perception or psychology experiment

  \begin{itemize}
  \tightlist
  \item
    Praat has a workable system for running experiments that can be
    adapted to many workflows. However, it's simply not as robust as
    other options like PsychoPy (\url{http://www.psychopy.org}) or
    OpenSesame (\url{http://osdoc.cogsci.nl}), especially when working
    with reaction times or external hardware (button boxes, voice-keys,
    etc). Learn to use one of these tools, and you'll have more
    flexibility and power when the time comes (and you'll be able to
    avoid expensive, non-free experiment packages like ePrime or
    SuperLab)
  \end{itemize}
\item
  Forced Alignment and automatic speech labeling/text-grid generation

  \begin{itemize}
  \tightlist
  \item
    Forced Alignment programs like FAVE-align
    (\url{http://fave.ling.upenn.edu/usingFAAValign.html}), Penn
    Phonetics Lab Forced Aligner (P2FA,
    \url{http://www.ling.upenn.edu/phonetics/p2fa/}), or Easy-Align
    (\url{http://latlcui.unige.ch/phonetique/easyalign.php}) are all
    reasonable choices for alignment.

    \begin{itemize}
    \tightlist
    \item
      Always remember that Forced Align speeds up manual annotation (by
      providing reasonable boundaries to correct). It does \emph{not}
      replace it!
    \item
      Don't forget that you can edit the resulting Praat Textgrids in
      your favorite text editor, and automatically un-label every word
      in your carrier sentence, or remove all non-target phonemes. This
      can save huge amounts of time.
    \end{itemize}
  \item
    As of Praat 6.0.2x, Paul Boersma appears to be working on including
    a forced-align component. Keep an eye on that space!
  \end{itemize}
\end{itemize}

\hypertarget{praats-scripting-tutorials}{%
\subsubsection{Praat's scripting
tutorials}\label{praats-scripting-tutorials}}

Perhaps the most useful part of Praat's documentation is the Scripting
tutorial, accessible through \emph{Objects -\textgreater{} Help
-\textgreater{} Praat Intro -\textgreater{} Scripting} or through the
Script Editor window. Because these tutorials are so excellent at
explaining the peculiarities and structures of Praat's scripting
language, this document will skip many of the topics discussed there,
focusing instead on overall usage.

\hypertarget{praat-scripts-vs.-editor-scripts}{%
\subsubsection{Praat scripts vs.~Editor
scripts}\label{praat-scripts-vs.-editor-scripts}}

\label{editorscripts}

One small but worthwhile note is that Praat actually draws a division
between scripts and ``editor scripts''. They use the same syntax, the
same commands, and are stored in the same way, but some scripts are
meant to run within the Editor window (because they use the functions in
that window), and some are meant to run from the Objects window.

Simply put, if there's a script whose function is only useful once
you've already opened the sound you care about in the editor window,
it's going to be run as an Editor script, but if a script needs access
to more than one sound, or to the functions in the Objects window, it'll
run from the main windows. There is some flexibility here, though, as
you can always use the \texttt{Edit} and \texttt{Editor} commands to
work within the editor in a Praat script, and using \emph{endeditor} can
bring you back to the Objects window from an editor script.

The primary way in which this distinction matters is that you'll open
Editor scripts using the \emph{Editor -\textgreater{} File
-\textgreater{} Open Editor Script\ldots{}} option, whereas you'll open
regular Praat scripts with \emph{Praat Menu -\textgreater{} Open Praat
Script\ldots{}}.

\vspace{0.5cm}
\begin{tabular}[h]{ p{0.6in} p{12cm}}
\includegraphics[width=0.5in]{danger.png} \newline \textbf{Danger!} & \raisebox{2mm}{\parbox{13cm}{\textit{Sometimes you’ll download a script from the internet or from a friend and it just won’t run, or it’ll fail in odd ways.  In those situations, try running it as an Editor script (or as a Praat script, if you’re trying it in the Editor).  This is a simple step that, if it works, can save you hours of frustration and troubleshooting, as there’s often no immediate indication how a given chunk of code was meant to be run.}}}
\end{tabular}
\vspace{0.5cm}

\hypertarget{working-with-scripts}{%
\subsection{Working with Scripts}\label{working-with-scripts}}

Because a Praat script is literally just a text file with a series of
commands to Praat, they can be found all over the internet, and can be
saved as a plaintext file (it's best to give it the .praat suffix, if
nothing else, for your own reference). Praat scripts can be run in two
distinct ways.

\hypertarget{opening-and-running-a-praat-script}{%
\subsubsection{Opening and running a Praat
script}\label{opening-and-running-a-praat-script}}

To open a Praat script in Praat and then run it:

\begin{enumerate}
\def\labelenumi{\arabic{enumi}.}
\tightlist
\item
  \emph{Praat Menu -\textgreater{} Open Praat Script\ldots{}}

  \begin{itemize}
  \tightlist
  \item
    This will open the script in a Script Editor window, where it can be
    viewed or edited.
  \end{itemize}
\end{enumerate}

To open an Editor script in Praat and then run it:

\begin{enumerate}
\def\labelenumi{\arabic{enumi}.}
\tightlist
\item
  Open the sound you're interested in in the Editor window.
\item
  \emph{Editor -\textgreater{} File -\textgreater{} Open Editor
  Script\ldots{}}

  \begin{itemize}
  \tightlist
  \item
    This will open the script in a Script Editor window, where it can be
    viewed or edited.
  \end{itemize}
\end{enumerate}

To run a script (either Editor or Praat scripts): 1. \emph{Script Editor
-\textgreater{} Run -\textgreater{} Run} (or Cmd+r, Ctrl+r on Windows)

Note that if you just want to run a small part of the script (useful
when cannibalizing scripts for certain functions), you can select the
part in question and use \emph{Script Editor -\textgreater{} Run
-\textgreater{} Run Selection}.

\hypertarget{making-and-removing-menu-shortcuts-for-scripts}{%
\subsubsection{Making (and removing) Menu Shortcuts for
scripts}\label{making-and-removing-menu-shortcuts-for-scripts}}

\label{menushortcuts}

Sometimes, you'll want frequent access to a Script without having to
find it, open it, then click ``Run''. To do this, you can add the script
to a menu. There are two options on how to do
this\footnote{Thanks, once again, to Dr. Rebecca Scarborough, from whose writing parts of this section are adapted.}:

First, you can add the script as a button on the right-hand side of the
Object window (called a dynamic menu). This will only work for
Non-Editor scripts. To do this:

\begin{enumerate}
\def\labelenumi{\arabic{enumi}.}
\tightlist
\item
  Open the Script, such that it's available in a Script Editor window.
\item
  \emph{Script Editor -\textgreater{} File -\textgreater{} Add to
  dynamic menu\ldots{}}
\item
  Fill in the settings as below in the \emph{Add to dynamic menu''}
  dialog box

  \begin{itemize}
  \tightlist
  \item
    \emph{Class} and \emph{Number} refer to which object types and how
    many must be selected for the new script button to appear. (For
    example, if you had a script that required a sound and a text grid,
    you would put \emph{Class 1}: Sound, \emph{Number 1}: 1, \emph{Class
    2}: TextGrid, \emph{Number 2}: 1. You would leave \emph{Class 3}
    blank.)
  \item
    \emph{Command} specifies the text that you want to appear on the
    button.
  \item
    \emph{After command} and \emph{Depth} pertain to the placement of
    the button. If you leave them blank, the button will appear at the
    bottom of the dynamic menu.
  \item
    \emph{Script file} should be filled in automatically, if you are
    working from a saved Praat script.
  \end{itemize}
\end{enumerate}

The other option is to add your script to a pull-down menu in the Object
window. (You can follow the same directions to add an Editor script to
an Editor window menu.) To do this:

\begin{enumerate}
\def\labelenumi{\arabic{enumi}.}
\tightlist
\item
  Open the Script, such that it's available in a Script Editor window.
\item
  \emph{Script Editor -\textgreater{} File -\textgreater{} Add to fixed
  menu\ldots{}}
\item
  Fill in the settings as below in the \emph{''Add to fixed menu''}
  dialog box

  \begin{itemize}
  \tightlist
  \item
    \emph{Window} specifies whether the command will appear in an Object
    window menu or a Picture window menu.
  \item
    \emph{Menu} specifies which pull-down menu the command will appear
    in (from the top of the window).
  \item
    \emph{Command} specifies the text that will appear in the menu.
  \item
    \emph{After command} and \emph{Depth} pertain to the location of the
    new command in the menu. Leaving the defaults will place the new
    command at the bottom of the menu.
  \item
    \emph{Script file} should be filled in automatically, if you are
    working from a saved Praat script.
  \end{itemize}
\end{enumerate}

To remove a script from a menu, take the below
steps\footnote{Thanks to Kathleen Currie Hall for pointing this out during the LSA Praat workshop!}

\begin{enumerate}
\def\labelenumi{\arabic{enumi}.}
\tightlist
\item
  \emph{Praat Menu -\textgreater{} Preferences -\textgreater{}
  Buttons\ldots{}}
\item
  Choose the proper category (``Objects'' for scripts in the objects
  window menus, ``Editors'' for scripts in the Editors window menus,
  etc)
\item
  Find the script you'd like to remove
\item
  Click on \emph{ADDED} such that it turns to \emph{REMOVED}
\item
  Close that window, and if needed, restart Praat
\end{enumerate}

Editor scripts can also be associated with one of the Script Logs (Log 3
or 4) in the ``Query'' menu of the Editor window. To do this:

\begin{enumerate}
\def\labelenumi{\arabic{enumi}.}
\tightlist
\item
  Open a sound in the Editor window
\item
  \emph{Editor -\textgreater{} Query -\textgreater{} Log
  Settings\ldots{}}
\item
  Type the path to the script in the Log Script 3 (or 4) box.
\end{enumerate}

Once this is completed, you can use a keyboard shortcut to run your
script.

\vspace{0.5cm}
\begin{tabular}[h]{ p{0.6in} p{12cm}}
\includegraphics[width=0.5in]{happyretroflex.png} \newline \textbf{Bonus!} & \raisebox{5mm}{\parbox{13cm}{\textit{Adding scripts to the Log Script slots is particularly useful for scripts which you might run extraordinarily often, as you can then, with some additional software, set a button on a Multi-button mouse to one of your log script keystrokes.  For instance, if you have a script that will take the current selection, create two TextGrid boundaries, label it “vowel”, then resize the view to the next interval on another tier, you could bind it to a log script, then just make a selection, click a mouse button, rinse and repeat, saving countless hours.}}}
\end{tabular}
\vspace{0.5cm}

\hypertarget{the-praat-script-i-downloaded-wont-run}{%
\subsubsection{``The Praat script I downloaded won't
run!'\,'}\label{the-praat-script-i-downloaded-wont-run}}

\label{scriptwontrun}

Unfortunately, there are a number of issues which mean that even for an
experienced user, downloading a script from the internet doesn't always
work immediately. Although it's hard to say exactly why your script
didn't run, here are some of the most common issues people run into
which cause scripts to fail:

\begin{enumerate}
\def\labelenumi{\arabic{enumi}.}
\tightlist
\item
  \textbf{You need to download the latest version of Praat}

  \begin{itemize}
  \tightlist
  \item
    This is the \#1 reason for script failure, right here. There are new
    improvements to the scripting language every few months. If the
    script was written by somebody running version 6 and you're running
    5.3, the script may include commands which the old version doesn't
    know how to run. So, go download the latest version from
    \url{http://praat.org}.
  \item
    Praat is wonderful in that old scripts are universally supported.
    Newer versions of Praat always run older scripts, so don't worry
    about ``breaking'' other scripts by upgrading.
  \end{itemize}
\item
  \textbf{The file paths are formatted incorrectly}

  \begin{itemize}
  \tightlist
  \item
    If you use a Mac and the script was written on a Windows machine (or
    vice versa), you'll need to change the file paths. See Section
    \ref{filenames} for more details.
  \item
    Also check to make sure they've not hard-coded paths (say, reading
    the data from the specific folder
    ``/Users/will/schwalapalooza/inputdata''), and change those paths to
    point to your data where needed.
  \end{itemize}
\item
  \textbf{You're using a different sound file type than the script
  wants}

  \begin{itemize}
  \tightlist
  \item
    Make sure the script isn't wanting .aiff files when you're using
    .wav (or vice versa). This is easy to change by changing the
    file-names hard-coded in.
  \item
    Because there's no practical difference between the two anymore, I
    recommend using .wav for your files, as it's more common and you'll
    face this issue less.
  \item
    Using .mp3 files breaks most scripts, and friends don't let friends
    save phonetic data in lossy formats (e.g.~.mp3, AAC, .wmv).
  \end{itemize}
\item
  \textbf{It's an editor script}

  \begin{itemize}
  \tightlist
  \item
    You need to make sure you know whether your script is an editor
    script or not, and run it appropriately. See \ref{editorscripts} for
    more information.
  \end{itemize}
\item
  \textbf{Your data isn't in the format it wants}

  \begin{itemize}
  \tightlist
  \item
    More complex scripts need very particular formats. Some require
    textgrid annotated individual words, split apart and in a folder.
    Some require a single long wav file, with one single textgrid
    labeling all of the words. Or maybe the script needs the files to
    named in a certain way. Or maybe you need to load all of the sounds
    into the objects window.
  \item
    Good scripts should have a read-me at the top of the file (or
    nearby) including this information. For bad scripts, you can usually
    figure this out by looking at the code. But complex scripts should
    have good documentation.
  \end{itemize}
\item
  \textbf{The Textgrids aren't what it expects}

  \begin{itemize}
  \tightlist
  \item
    Often, scripts use Textgrid files to let you specify where to
    measure, but there's no particular standard as to how those are
    made. Maybe the script is looking at Tier 1 for the vowel, but your
    Tier 1 has words labeled, and Tier 2 is the vowel. Or maybe it wants
    point tiers, but you've used intervals. Or maybe it needs a specific
    label (``V'' for vowels) to intervals to measure.
  \item
    Script makers will often allow you to specify tiers or labels for
    the relevant measures. But failing that, you can always go through
    and hard-code your tier number, changing the tier number in any
    commands that read from the textgrid.
  \item
    Remember too that Praat TextGrid files are just text files. If the
    script wants every measured vowel to be labeled ``V'', and you want
    to measure everything labeled ``Vwla'', ``Vwlu'', or ``Vwli'', you
    can always open a copy of the grid file in a text editor and
    find-and-replace the labels to relabel them.
  \end{itemize}
\item
  \textbf{You're doomed}

  \begin{itemize}
  \tightlist
  \item
    Sometimes, you'll never get a script to run. Maybe it never ran, or
    there's a bug that the author didn't catch. Or maybe the author was
    doing something very unusual, or was relying on a second script to
    feed it data. But regardless, it may not be your fault. Try and find
    another script which does the same thing, or write your own.
  \item
    Despite working with Praat for 10 years, just last week, I
    downloaded a script for automatically generating .TextGrid files for
    a folder full of .wav files that I just couldn't make run. After
    around 20 minutes of battling, I just gave up, and wrote my own.
  \end{itemize}
\end{enumerate}

Finally, a note: When you've found a script online in somebody's
repository, it's very tempting to contact the author of the script and
ask for help in running it. If it's a short script you can find
elsewhere, don't bother, but this might be an option in cases where the
script is designed just for what you're doing and would save you years
of work. But \textbf{contacting the author should be your last resort},
as they're likely just as busy as you are, if not more so, and many
don't have time to help people debug work that they've posted online.

So, before you even consider contacting the author, I'd recommend you
put 5-10 hours into trying to understand the code and debug it yourself.
You'll learn a great deal from this process, and there's a pretty good
chance you'll fix the problem yourself. You might also bug somebody else
in your life who works with Praat, if such a person exists, because
troubleshooting is always easier when you've got the machine and code in
front of you and can just ``try something.''

But if it's still not working, despite putting in a lot of work to fix
it, and there's no other script that does what you need, send the author
an email. As somebody who maintains a large repository of scripts and
gets a \emph{lot} of emails from people, here are a few things you can
do to give yourself the best chance of a real response:

\begin{itemize}
\tightlist
\item
  Download the latest version of Praat first.

  \begin{itemize}
  \tightlist
  \item
    This is my default response to people contacting me, and around 70\%
    of the time, when people upgrade from the antique version of Praat
    they've had on their machine since grad school, the errors abruptly
    disappear.
  \end{itemize}
\item
  Include lots of information (``I'm running on Windows 10, Praat
  Version 6.0.23, using a folder full of .wav files and textgrids, and
  using the version of the script found at {[}URL{]}. I've attached a
  screenshot of the error message I'm getting when I run the script, and
  here's a Dropbox link to the folder full of data I'm trying to run
  on.'')
\item
  Check all of the above issues, and mention what you've already tried
  in your initial email. I'm more likely to respond if I know that
  you're not just facing Windows filename glitches or an outdated
  version of Praat.
\item
  Mention what you're working on! It's silly, but if you're doing
  interesting things, or using the script for your dissertation, you
  become more human, and the time spent responding seems more
  reasonable.
\end{itemize}

But most importantly, be kind. I know that I feel terrible ignoring
emails from people asking help, but some days, I'm just barely able to
finish my own work, let alone help with others. So, if you don't get a
response within a few weeks, try again, but don't take it personally if
it still doesn't work.

And, if all else fails, and you just can't get somebody else's script
running, maybe it's time to think about\ldots{}

\hypertarget{creating-a-new-script}{%
\subsection{Creating a new script}\label{creating-a-new-script}}

At its core, a Praat script is just a plaintext file with commands meant
to be read by Praat. As such, creating a script is as easy as saving a
text file, then editing the contents. The easiest way to do this is
using Praat's built-in Script Editor.

To create a new Praat script:

\begin{enumerate}
\def\labelenumi{\arabic{enumi}.}
\tightlist
\item
  \emph{Praat Menu -\textgreater{} New Praat Script\ldots{}}
\item
  \emph{Script Editor -\textgreater{} File -\textgreater{} Save}

  \begin{itemize}
  \tightlist
  \item
    Save it wherever you'd like on your machine, and use .praat as an
    extension (not because it's required, but because it can't hurt,
    will help you remember what the file is, and allows Praat to open
    the file as a script if dragged onto the Praat icon)
  \end{itemize}
\item
  Begin writing, saving often
\end{enumerate}

\hypertarget{using-other-text-editors}{%
\subsubsection{Using other text
editors}\label{using-other-text-editors}}

Although Praat has an editor built in which is quite capable, praat
scripts can be created from within any plaintext editor. Using an
external editor allows you features like syntax highlighting (Praat
script is similar to Perl in enough ways to make Perl syntax highlights
vaguely useful), always-on line numbers (because Praat gives you line
numbers where your script crashed and it's less of a pain to just scroll
than to use \emph{Script Editor -\textgreater{} Search -\textgreater{}
Go to line\ldots{}}), and the sorts of advanced find/replace tools
present in more robust editors.

Some good editor choices include:

\textbf{For Mac OS X}:

\begin{itemize}
\tightlist
\item
  TextMate 2 - (Version 2 is free and open source) -
  \url{http://macromates.com/}
\item
  MacVim - \url{https://code.google.com/p/macvim/}
\item
  SublimeText - \$\$\$ -\url{http://www.sublimetext.com}
\end{itemize}

\textbf{For Windows}:

\begin{itemize}
\tightlist
\item
  Notepad ++ - \url{http://notepad-plus-plus.org/} - (also Scott
  Sadowsky's Notepad ++ Syntax Highlighting package for Praat at
  \url{http://sadowsky.cl/praat.html#syntax})
\item
  TextPad - \$\$\$ - \url{http://www.textpad.com/}
\item
  SublimeText - \$\$\$ -\url{http://www.sublimetext.com}
\end{itemize}

\textbf{For Linux}:

\begin{itemize}
\tightlist
\item
  Kate - \url{http://kate-editor.org} - (See José Joaquín Atria's Kate
  syntax highlighting package for Praat at
  \url{https://github.com/jjatria/praatKateSyntax})
\item
  SublimeText - \$\$\$ -\url{http://www.sublimetext.com}
\end{itemize}

\hypertarget{filenames}{%
\subsubsection{Filenames}\label{filenames}}

\label{filenames}

Praat has a few odd quirks involving filenames which you'll need to work
around in your scripting. First, Praat chokes on filenames with decimals
in them. Be careful.

Also, absolute file paths in Praat scripts will need to be referred to
differently on Windows machines vs.~on Macs:

\begin{itemize}
\tightlist
\item
  directory\$ = ``c:\textbackslash Documents\textbackslash test
  data\textbackslash'' (Windows)
\item
  directory\$ = ``/Users/will/Documents/test data/`` (Mac)
\end{itemize}

On Unix-like systems, \texttt{\~} can be used to refer to the home
directory (/Users/will/).

Relative file path names (relative to where the script is) do not need
to change across platforms. For instance, so if you included the below
directory assignments in your script:

\begin{itemize}
\tightlist
\item
  directory\$ = ``test data/``
\end{itemize}

(or)

\begin{itemize}
\tightlist
\item
  directory\$ = ``./test data/``
\end{itemize}

Praat would always look for data in a folder called ``test data'', in
the same folder as the script itself. This script would function well on
any operating system, provided the data is put in the proper
folder.\footnote{Thanks to Paul Boersma for pointing out this particular tip}

\vspace{0.5cm}
\begin{tabular}[h]{ p{0.6in} p{12cm}}
\includegraphics[width=0.5in]{danger.png} \newline \textbf{Danger!} & \raisebox{3mm}{\parbox{13cm}{\textit{Incorrectly formatted absolute path names are the most common issue which prevents you from running scripts you download from the internet.  If your script won’t run, check the file paths throughout the script and update them to match your operating system.}}}
\end{tabular}
\vspace{0.5cm}

\hypertarget{a-note-on-praat-script-commands}{%
\subsubsection{A Note on Praat Script
Commands}\label{a-note-on-praat-script-commands}}

\label{newcommands} In the Summer and Fall of 2013, Paul Boersma has
updated the Praat script language, changing the syntax of some of the
commands to be more familiar to people working with other programming
languages. Although the old commands (as shown in the manual) do still
work, new commands generated by the the process will look like:

\texttt{do (“To Manipulation...”, 0.01, 75, 600)}.

\ldots{} versus the old command format \ldots{}

\texttt{To Manipulation... 0.01 75 600}

Don't be alarmed if you see this new format. It's fairly straightforward
to convert between the two formats mentally, and old commands do still
work. I have not had the time to update all of the code in the manual
for the new format, but anything you see here still works if dropped
into a script.

That said, \emph{new-style commands will cause scripts to crash in older
versions of Praat}. This is yet another reason to download the newest
version of Praat every time you start a new project.

\hypertarget{how-to-magically-write-a-praat-script-using-the-praat-history-function}{%
\subsubsection{How to magically write a Praat script (using the Praat
``history''
function)}\label{how-to-magically-write-a-praat-script-using-the-praat-history-function}}

Praat scripting relies on Commands, which are, effectively, verbs which
describe the actions Praat should take. These commands can take a
variety of forms, but they usually start with an uppercase letter (or,
in newer versions, ``Do ()''). Here are a bunch of Praat commands,
selected randomly from several scripts:

\begin{verbatim} 
Rename...
Extract part...
Get starting point...
select
Get high index...
To Manipulation... 0.01 75 600
Get first formant
Get frequency of maximum...
Erase all
(...and many, many more)
\end{verbatim}

The most beautiful part of Praat scripting comes with the realization
that if you can do something through the GUI, Praat will magically give
you the exact commands to do it by script. Let's examine this wonderful
reality by unintentionally writing a Praat script, using Praat's history
function.

First, let's do a little bit of work in Praat, to simulate doing
phonetic research:

\begin{enumerate}
\def\labelenumi{\arabic{enumi}.}
\tightlist
\item
  Open a sound, and extract the vowel
\item
  \emph{Praat Menu -\textgreater{} New Praat Script\ldots{}}
\item
  \emph{Script Editor -\textgreater{} Edit -\textgreater{} Clear
  history}
\end{enumerate}

\textbf{Now, let's get the duration of the sound}

\begin{enumerate}
\def\labelenumi{\arabic{enumi}.}
\setcounter{enumi}{3}
\tightlist
\item
  Select the sound
\item
  \emph{Objects -\textgreater{} Query -\textgreater{} Query Time Domain
  -\textgreater{} Get Total Duration}
\end{enumerate}

\textbf{Now, let's get the average height of F1 and F2 of the sound}

\begin{enumerate}
\def\labelenumi{\arabic{enumi}.}
\setcounter{enumi}{5}
\tightlist
\item
  Select the sound
\item
  \emph{Objects -\textgreater{} Formants \& LPC -\textgreater{} To
  Formant (burg)\ldots{}}
\item
  This creates a formant object from which we can get the mean formant
  values
\item
  Select the new formant object
\item
  \emph{Objects -\textgreater{} Query -\textgreater{} To mean\ldots{}}

  \begin{itemize}
  \tightlist
  \item
    This measures the mean for a given formant. Repeat this for Formants
    1 and 2
  \end{itemize}
\end{enumerate}

Now, let's reap what we've sown:

\begin{enumerate}
\def\labelenumi{\arabic{enumi}.}
\setcounter{enumi}{10}
\tightlist
\item
  Go back to your Script Editor window
\item
  \emph{Script Editor -\textgreater{} Edit -\textgreater{} Paste
  history}
\end{enumerate}

If all went to plan, your Script Editor window will have something like
the code pictured below:

\textbf{Code generated from the above process:}

\begin{verbatim} 
Edit
Extract selected sound (time from 0)
Close
Get total duration
To Formant (burg)... 0 5 5500 0.025 50
Get mean... 1 0 0 Hertz
Get mean... 2 0 0 Hertz
\end{verbatim}

Each of those lines are the Praat script commands to do each of the
tasks. \texttt{To Formant (burg)... 0 5 5500 0.025 50} turns a sound
into a Formant object, with the settings specified in the trailing
numbers. \texttt{Get mean... 1 0 0 Hertz}, intuitively enough, gets the
mean of the formant specified in the first number, in the time range
specified by the next two (0 = ``all''), and gives it in Hertz (as
opposed to Bark). Note, though, that because you used the mouse to do
the selection (instead of using the commands in the \emph{Editor
-\textgreater{} Select} menu), the exact definition of the selection
doesn't show up.

In short, this is the kernel of a Praat script that does everything you
just did, and this is one of the most effective ways of figuring out
what commands to use in your Praat scripting. The process is always the
same:

\begin{enumerate}
\def\labelenumi{\arabic{enumi}.}
\tightlist
\item
  Open the Script Editor
\item
  \emph{Script Editor -\textgreater{} Edit -\textgreater{} Clear
  history}
\item
  Do, using the interface, whatever actions you'd like to get the script
  commands for
\item
  \emph{Script Editor -\textgreater{} Edit -\textgreater{} Paste
  history}
\end{enumerate}

By doing this, you'll be able to see how to command Praat to do anything
that you can do in the GUI.

\hypertarget{writing-your-first-single-command-script}{%
\subsubsection{Writing your first single-command
script}\label{writing-your-first-single-command-script}}

Praat scripts can run the gamut between massive, 800+ line programs and
single-function, single line editor scripts. Although the massive
programs are often flashiest, in many cases, the single-item scripts are
the ones you can't live without.

First, let's ``write'' a script which will turn the Spectrogram
displayed in the editor window into a Narrowband spectrogram (as
described in Section \ref{subsec:broadnarrow}), using the history
function. To do this:

\begin{enumerate}
\def\labelenumi{\arabic{enumi}.}
\tightlist
\item
  Open a sound in the Editor window
\item
  Open the Script Editor and create a new script
\item
  \emph{Script Editor -\textgreater{} Edit -\textgreater{} Clear
  history}
\item
  \emph{Editor -\textgreater{} Spectrum -\textgreater{} Spectrogram
  Settings}
\item
  Set the \emph{Window Length} to 0.025 (or the narrowband window length
  of your choosing)
\item
  Go back to the Script Editor window, then \emph{Script Editor
  -\textgreater{} Edit -\textgreater{} Paste history}
\item
  Save the script wherever you'd like using \emph{Script Editor
  -\textgreater{} File -\textgreater{} Save}
\end{enumerate}

The resulting ``script'' will look like:

\begin{verbatim} 
Spectrogram settings... 0 5000 0.025 50
\end{verbatim}

That's it.

Because this code works entirely with the Editor window, you'll want to
open it as an Editor script. You can use the \emph{Editor
-\textgreater{} File -\textgreater{} Open Editor Script\ldots{}} command
to open it and run it from there, or you can add it to a menu (likely
\emph{Spectrum}) using the procedure in Section \ref{menushortcuts}.

Once you've done that, if you want to change the spectrogram type,
you're only a click away. You'll likely end up with a few of these
scripts in that menu, to change the spectrogram into broadband,
narrowband, 0-10000 Hz broadband, and so forth. But sometimes, you need
to get a bit more complicated.

\hypertarget{scripts-with-variables}{%
\subsubsection{Scripts with Variables}\label{scripts-with-variables}}

Variables are constructs in programming which are just labels which
store pieces of information, either numeric or string (text). In Praat,
they have a few characteristics worth
noting\footnote{Thanks, yet again, to Dr. Rebecca Scarborough for portions of this text}:

\begin{itemize}
\tightlist
\item
  Variable names have to start with a lower case letter (so as not to be
  interpreted as commands to the program) and must not contain spaces or
  start with a punctuation mark.
\item
  String variables (containing text) must end in the character \$.
\item
  The ``=`` sign is used to assign values to variables (e.g.,
  \texttt{f0 = 120} stores 120 in the variable \texttt{f0}).
\item
  Values for string variables must be placed in double quotes. *

  \begin{itemize}
  \tightlist
  \item
    So, to specify a file's name:
    \texttt{soundname\$ = “recording\_1\_suzanne”}
  \end{itemize}
\end{itemize}

For a more concrete demonstration, let's take another task. In
\emph{Acoustic and Auditory Phonetics} (\cite{Johnson:1997aa}), Keith
Johnson provides a formula to calculate the length of person's vocal
tract based on any formant of a neutral vowel. He gives this formula as:

\begin{quote}
\texttt{L = $(2n-1)c/4F_n$}
\end{quote}

\begin{quote}
\emph{L = Length in Meters , n = Formant Num., c = the speed of sound in
air (343 m/s)}
\end{quote}

Let's say you want to calculate this more often, and grow tired of
manually calculating it. Instead, you'd like to be able to record a
neutral vowel, then calculate the length based on F3 at the cursor's
location.

First, you'll need to capture the F3 value at the cursor. By using the
history command, you'll find that the command for getting F3 at the
cursor is \emph{Get third formant}.

However, this normally just pops the output of the command (the F3
height) into an info window. Because we need that information to be
accessible to the script, we need to capture it as a variable. Although
Praat's tutorial on Scripting has a wonderful section on Variables, and
you should really read through that to get a full idea, in Praat,
capturing the output of a command as a variable is easy.

To assign a variable to the output of a command, you just name the
variable, then put an equals sign, then the command whose output you
want to capture, like so:

\begin{quote}
\texttt{variable = commandtogetoutput}
\end{quote}

Or, in our case:

\begin{quote}
\texttt{f3height = Get third formant}
\end{quote}

The above line will then capture the output of ``Get third formant'' and
store it. So, if F3 = 2600 Hz at the cursor, once this command is run,
the variable \texttt{f3height} will be equal to \texttt{2600}.

Then, you'll use that number in the formula above to calculate a new
variable, the length of the vocal tract. Remember, having run the first
command, f3height will now be equal to the height of F3. Praat is
perfectly capable of doing basic math, so we can hard-code the other
parts of the equation into the script:

\begin{quote}
\texttt{length = ((2*3 -1) * 343/(4 * f3height))}
\end{quote}

This will assign the output of \texttt{((2*3 -1) * 343/(4 * f3height))}
to the variable \texttt{length}.

Finally, you'll want to convert the output (\texttt{length}), which is
currently in meters, to a more useful unit, namely, centimeters:

\begin{quote}
\texttt{lengthcm = length * 100}
\end{quote}

Then, you'll need to display this information in an information window.
To do that, use the \texttt{print} command.

\begin{quote}
\texttt{print Your vocal tract length is  ‘lengthcm:1’  cm}
\end{quote}

This command will print the sentence ``Your vocal tract length is'',
followed by the \texttt{lengthcm} variable's contents, followed by
``cm''. The \texttt{:1} attached to \texttt{lengthcm} rounds the output
down to one decimal point.

Put together, this is your entire script:

\begin{verbatim} 
f3height = Get third formant
length = ((2*3 -1) * 343/(4 * f3height))
lengthcm = length * 100
print Your vocal tract length is  ‘lengthcm:1’  cm
\end{verbatim}

You would then save the script and either open it as an Editor script
(because it relies on the mouse cursor within the editor window when
getting the third formant), or add to a menu in the editor window. So,
here, three steps of measurement and calculation are condensed into one
easy menu selection.

\hypertarget{about-the-praat-scripting-language}{%
\subsection{About the Praat Scripting
Language}\label{about-the-praat-scripting-language}}

We've already talked about variables, but there are several other
programming constructs used in Praat scripting which you'll need to know
about to successfully script.

\hypertarget{for-loops}{%
\subsubsection{`for' loops}\label{for-loops}}

\label{forloops}

In Praat, sometimes, you'll want to do the same thing over and over
again, for, e.g.~each item on a list, or for each vowel in a TextGrid.
To do this, you'll use a \texttt{for} loop. For loops iterate through
large amounts of data, doing the same thing many times over and over
again.

\texttt{for} loops have the form:

\begin{verbatim}
for [variable] from 1 to [another variable]
    Take this action for each of them
endfor
\end{verbatim}

In Praat, \texttt{for} loops usually have the format
\texttt{for [var] from 1 to [other var]}, followed by an indented block,
ended with an \texttt{endfor} (to tell Praat when to stop).

Here's a sample script:

\begin{verbatim} 
select TextGrid ‘sounda’
number_intervals = Get number of intervals... 2
for k from 1 to number_intervals
    Set interval text... 2 k Vowel
endfor
\end{verbatim}

This script selects a TextGrid (named `sounda'), gets the number of
intervals in the TextGrid, then goes through and changes the text for
each interval in tier two of that TextGrid to ``Vowel'')

Really, though, the best way to understand \texttt{for} loops is to see
them in action and to look through other scripts.

It's worth noting that, if you always plan to start from the first
iteration, you can leave off the ``from 1'\,' from the statement. So,
the above could be written as:

\begin{verbatim} 
select TextGrid ‘sounda’
number_intervals = Get number of intervals... 2
for k to number_intervals
    Set interval text... 2 k Vowel
endfor
\end{verbatim}

and it would function identically.

If you need to exit or restart a \texttt{for} loop, say, based on a
conditional, the best approach is to use a \texttt{goto} and
\texttt{label} (see Section \ref{goto}):

\begin{verbatim} 
select TextGrid ‘sounda’
number_intervals = Get number of intervals... 2
for k to number_intervals
    Set interval text... 2 k Vowel
    if k > 10
        print "10 intervals already? I'm just going to call this done."
        goto end
    endif
endfor
label end
\end{verbatim}

\hypertarget{if-statements}{%
\subsubsection{`if' statements}\label{if-statements}}

\texttt{if} statements tell Praat to take a certain set of actions
\textbf{only if} the requested conditions are met. This condition is
usually checking whether a variable is equal to (or greater, lesser
than) a certain value, but it can get more complicated than that.

\texttt{if} statements have the form:

\begin{verbatim}
if variable = something
    Take this action
endif
\end{verbatim}

Here are a few example if statements that you might see in a script:

\begin{verbatim} 
vowel_label$ = Get label of interval... 1 2
if vowel_label = “i”
    Set interval text... 2 2 HighFrontVowel
endif
\end{verbatim}

The above code will change the TextGrid's label to ``HighFrontVowel''
only if the vowel's label is ``i''.

\begin{verbatim}
vowel_label$ = Get label of interval... 1 2
if vowel_label != “i”
    Set interval text... 2 2 AnotherVowel
endif
\end{verbatim}

The above code will change the TextGrid's label to
\texttt{AnotherVowel’’\ only\ if\ the\ vowel’s\ label\ is\ **not**}i'\,'.
As you can see, \texttt{if} statements are incredibly useful, and will
show up constantly to control the flow of your scripts. You'll learn to
love them very quickly.

You can also use \texttt{else} and \texttt{elsif} to allow yourself
multiple choices:

\begin{verbatim}
vowel_label$ = Get label of interval... 1 2
if vowel_label = “i”
    Set interval text... 2 2 HighFrontVowel
elsif vowel_label = “u”
    Set interval text... 2 2 HighBackVowel
else
    Set interval text... 2 2 SomeOtherVowel
endif
\end{verbatim}

\vspace{0.5cm}
\begin{tabular}[h]{ p{0.6in} p{12cm}}
\includegraphics[width=0.5in]{danger.png} \newline \textbf{Danger!} & \raisebox{3mm}{\parbox{13cm}{\textit{For much of its life, Praat script used \texttt{<>} to mean ``not equal to’’ (rather than the usual \texttt{!=} or \texttt{/=} or \texttt{=/=}). Although recent versions (mercifully) allow the far more common \texttt{!=} to mean the same thing, be aware that older scripts will still include this odd \texttt{<>} notation.}}}
\end{tabular}
\vspace{0.5cm}

\hypertarget{while-loops}{%
\subsubsection{`while' loops}\label{while-loops}}

\label{while} \texttt{while} loops are meant to tell Praat to continue
doing something until a certain condition is met. Let's say you wanted
to add a second sound to a first sound over and over again until it was
3 seconds long:

\begin{verbatim}
select Sound ‘soundname$’\_original 
while chunk_duration < 3
    plus Sound ‘soundname$’\_addition
    Concatenate
    chunk\_duration = Get total duration
endwhile
\end{verbatim}

Here, a while loop is ideal, as it just keeps going until the criterion
is reached with no further code, muss, or fuss.

\vspace{0.5cm}
\begin{tabular}[h]{ p{0.6in} p{12cm}}
\includegraphics[width=0.5in]{danger.png} \newline \textbf{Danger!} & \raisebox{3mm}{\parbox{13cm}{\textit{While loops are great, but Praat (especially on OS X) acts funny if it has to do more than a certain number of iterations, as they can quickly fill up your memory. Especially if your while loop deals with something complex which might not always happen, if Praat starts crashing on certain tokens, your while loops are a good place to start looking.}}}
\end{tabular}
\vspace{0.5cm}

\hypertarget{forms}{%
\subsubsection{Forms}\label{forms}}

Forms can run either at the start of the script, or during the script
itself, and are your way of getting information from the user. Forms can
be used to tell the script how to run, specify options, or have the
users input starting information (say, pitch ranges or the number of
measures per vowel). These let more complex scripts get user feedback
before and while they do what they do best. For an example starting
form, please see Figure \ref{form}.

\begin{figure}
  \centerline{
    \mbox{\includegraphics[width=3.50in]{form.png}}
  }
    \caption{An example of a startup form from the CU Nasality Measurement Script\label{form}}
  
  \end{figure}

Forms are created with blocks of code like the below:

\begin{verbatim}
form [Label for the form]
    comment [what you’re asking the user for]
        text directory [your suggestion for what they type in]
    comment [Another request, but this lets them assign a number for each choice]
        integer vowel 1
        integer word 2
    comment [Yet another request from the user, but here, with a choice.  2 is default]
        choice mode 2
        button Mode1
        button Mode2
    comment [This choice is boolean, Yes vs. No, and defaults to yes]
        boolean ScriptIsAwesome Yes
endform
\end{verbatim}

This code would create a form like below, in Figure \ref{demoform}.

\begin{figure}
  \centerline{
    \mbox{\includegraphics[width=4.00in]{demoform.png}}
  }
    \caption{A form generated with the demo form code above\label{demoform}}
  
  \end{figure}

You could then use that form's choices to make decisions later:

\begin{verbatim}
if mode = 1
    print “Your mode is set to 1”
elsif mode = 2
    print “Your mode is set to 2”
endif
if ScriptIsAwesome
    print “You’re awesome!”
else
    print “Someday, you’ll be awesome”
endif
\end{verbatim} 
\pagebreak

\hypertarget{goto-and-label}{%
\subsubsection{goto and label}\label{goto-and-label}}

\label{goto}

The \texttt{goto} command allows you to jump around within your script,
and is generally used to exit loops, to re-try a procedure, or otherwise
cause mischief. To use a \texttt{goto}, you must first specify a point
in the script using the \texttt{label} command, and then \texttt{goto}
that label:

\begin{verbatim}

select Sound ‘soundname$’\_original 
while chunk_duration < 3
    plus Sound ‘soundname$’\_addition
    Concatenate
    chunk\_duration = Get total duration
    if chunk_duration > 10
        print "You overshot that bigtime.  Check your chunks."
        goto endscript
    endif
endwhile

label endscript
\end{verbatim}

\vspace{0.5cm}
\begin{tabular}[h]{ p{0.6in} p{12cm}}
\includegraphics[width=0.5in]{danger.png} \newline \textbf{Danger!} & \raisebox{3mm}{\parbox{13cm}{\textit{For the most part, you shouldn't use gotos except to exit loops. For most other uses, an if statement or while loop will accomplish the same thing, but in a more easy-to-debug way.  Much like sticks of dynamite, gotos are very, very seldom the right tool for the job, and they're just as likely to blow up in your face as they are to fix your problem.  Think long and hard before you light that fuse.}}}
\end{tabular}
\vspace{0.5cm}

\hypertarget{commented-lines}{%
\subsubsection{Commented lines (\#)}\label{commented-lines}}

Not every part of a Praat script is designed to be read by a computer.
Lines which start with a \# symbol are called \textbf{Commented Lines},
and will be ignored by Praat. They're usually written in by human
programmers to explain exactly what the code is doing, or in some cases,
as a header. Do yourself a favor and comment the code that you write, as
sometimes, it's the only way you'll understand your code when you look
at it days/months/years later.

Here's an example of some commented code. Remember, Praat will only
``see'\,' the lines which don't start with \#:

\begin{verbatim} 
# Now, this next chunk starts a for loop that just goes through 
# the above directory and keeps reading in files each time it iterates
for j from 1 to number_files
    select Strings list
    filename$ = Get string... ‘j’
    Read from file... ‘directory$’’filename$’
    
    # At this point, we now have the variable “soundname” 
    # which refers to the file being worked on
    soundname$ = selected$ (“Sound”)
    
    ...
\end{verbatim}

\hypertarget{scaling-scripts-up-nowarn-noprogress-and-avoiding-the-editor-window}{%
\subsubsection{Scaling scripts up: nowarn, noprogress, and avoiding the
editor
window}\label{scaling-scripts-up-nowarn-noprogress-and-avoiding-the-editor-window}}

\label{scaling}

Lengthy praat scripts, meant to run on a great many sounds in a highly
automated way, benefit greatly in terms of speed and efficiency by
reducing the number of dialogs which pop up.

If you're doing something which modifies a sound in a way that
occasionally produces a tiny bit of clipping (yielding an error message
like ``Advice: 2 of 34583 samples are clipped.'\,' upon saving), you can
use the \texttt{nowarn} directive to save without warning. For instance:

\begin{verbatim} 
select Sound ‘soundname$’_mod
nowarn Write to WAV file... ‘directory$’output/‘soundname$’_modified.wav
select TextGrid ‘soundname$’
Write to text file... ‘directory$’output/‘soundname$’_modified.TextGrid
\end{verbatim}

This will save the file as a .wav file, and completely ignore any errors
resulting from clipping. Mind you, this means that there will be some
clipping in your file, which is a Bad Thing. One or two samples may be
fine, but in a production script, you'll want to Scale Amplitude or
Scale Peaks to make that stop.

If you're doing the same process over and over again in a highly
automated fashion on a fast machine, you'll notice that Praat spends
more time rendering the ``Progress: To Pitch\ldots'\,' type of dialogs
than it does making the pitch file. To suppress the generation of these
dialogs (letting Praat run more silently in the background), use the
\texttt{noprogress} directive:

\begin{verbatim} 
select Sound ‘soundname$’_mod
noprogress To Formant (burg)... 0 formnum formrange 0.0256 50
noprogress To Pitch... 0 60 ‘crazyhighh1’
\end{verbatim}

This will generate the formant and pitch objects without displaying a
dialog, allowing the process to run much more quickly than it otherwise
would have. This will have the side effect of effectively preventing the
user from stopping the script mid-run (without force-quitting Praat),
but the gain in speed may offset this inconvenience.

Finally, it's important to realize that as you're scripting, anything
that actually displays on the screen will be more resource- and
time-intensive than a background task, especially if a spectrogram is
being generated. As such, if you have the choice of doing a given
process using the editor window (Open Sound in Editor, Move cursor to
point X, Get First Formant, Get Second formant, close editor window) or
creating and querying an object (To Formant\ldots, Get Value at time 1
timepoint, Get Value at time 2 timepoint), it will certainly be faster
to use an object and leave the editor closed.

Editor scripting is great, especially as you're just starting off, but
if you're planning to run on a large number of items, removing Editor
Scripting from the mix will almost certainly result in a speed increase
over huge datasets.

\vspace{0.5cm}
\begin{tabular}[h]{ p{0.6in} p{12cm}}
\includegraphics[width=0.5in]{danger.png} \newline \textbf{Danger!} & \raisebox{3mm}{\parbox{13cm}{\textit{When you have purged all user-interface-generating elements from your script, at least on macOS, once you start your script running, Praat will appear to "crash", with the user interface becoming unresponsive, even to the extent that the file chooser dialog can't be moved.  This is simiply a consequence of your script's efficiency coupled with Praat's coding.  You'll see that data and files are saving happily to the designated directories, and checking Activity Monitor will reveal heavy CPU usage, but Praat will appear dead-to-the-world.  When the script finishes (or errors out), Praat will return to life.}}}
\end{tabular}
\vspace{0.5cm}

\hypertarget{useful-tips}{%
\subsubsection{Useful tips}\label{useful-tips}}

\label{scriptingtips}

These are a few tips and tricks that I learned well after they would've
saved me hours upon hours of time. I'm passing them along in hopes that
they'll save you the time more quickly.

\begin{itemize}
\item
  To refer to an object in the Objects window, it's best to give both
  its type (Sound, TextGrid, etc) and its name. So, instead of coding
  \texttt{select sounda}, you'll want to use
  \texttt{select Sound sounda}.
\item
  If your code is failing, sometimes the line numbers in the crash
  messages won't tell you exactly how far the script is getting before
  it crashes. To test this, insert the word ``fail'\,' (or anything else
  that isn't a real command) into your script at a certain point. The
  moment it hits that word, the script will crash, and you'll know
  you've gotten that far. Then repeat until you crash for some other
  reason. At that point, you'll know exactly where your script goes off
  the rails.
\item
  Scripts will sometimes fail talking about a certain variable being
  \texttt{ - -undefined - -}. This usually means that you're looking for
  a pitch or pulse in a place where Praat's pitch tracking can't find
  voicing. To help prevent these failures (which will crash the entire
  run of a script), you may considering adding a small block of code
  like the following, which will only take a certain action if the
  answer is defined:
\end{itemize}

\begin{verbatim} 
if pulse_begin_time <> undefined
    Get pitch...
endif
\end{verbatim}

\begin{itemize}
\item
  Once again, comment your code. If I had a nickel for every time I've
  opened up a script I wrote, only to find myself unable to understand
  what I had written, I'd have a lot more nickels.
\item
  A great many issues in Praat scripting can be resolved by looking at a
  timepoint just 5-10 ms to either direction. Adding in a loop which, if
  an unreasonable measure (or an \texttt{ - -undefined - -}) is
  encountered, tries again at a nearby second timepoint, can really
  increase the robustness of your scripting.
\item
  If your script is automatically creating lots of objects, you'll want
  to clean house periodically. Although it's ideal to hard-code the
  deletion of every item manually (select Sound sounda, Remove),
  sometimes you wind up generating a large number of no-longer-needed
  files. A little chunk of code along the lines of the below will do the
  trick nicely:
\end{itemize}

\begin{verbatim} 
select all
minus Sound ‘soundname$’
minus TextGrid ‘soundname$’
minus (whatever other files you care about)
Remove
\end{verbatim}

\begin{itemize}
\tightlist
\item
  To check if a number is even or odd in a Praat script, use the ``mod''
  command:

  \begin{verbatim}
  oddness = number mod 2
  if oddness = 0
    print “Number is even”
  elsif oddness = 1
    print “Number is odd”
  endif
  \end{verbatim}
\item
  If you're using editor scripting, Praat will not naturally close the
  editor window, and will fail after 5 copies of the same window are
  open. Make sure to include a `Close' command. So, for instance, in my
  formant script, after pausing for human confirmation, I have the
  following, which grabs control of the open window and then closes
  it:\\

  \begin{verbatim}
    Edit
  editor Sound 'soundname$'_word
            Close
    endeditor
  \end{verbatim}
\end{itemize}

\hypertarget{everything-else}{%
\subsubsection{Everything Else}\label{everything-else}}

Praat scripting takes a long time to really get your mind around, and
there are all sorts of other constructs which can be useful in working.
You would be well served, once you've got your mind wrapped around the
basics, to examine the Praat Scripting tutorial pages (\emph{Objects
-\textgreater{} Help -\textgreater{} Praat Intro -\textgreater{}
Scripting}) for some of the following concepts, which will improve your
scripting efficiency even more.

\begin{itemize}
\tightlist
\item
  Scripting 5.3. Jumps

  \begin{itemize}
  \tightlist
  \item
    This section talks about using \texttt{elsif} and \texttt{else} to
    improve your script's flow
  \end{itemize}
\item
  Scripting 5.5. Procedures

  \begin{itemize}
  \tightlist
  \item
    This section talks about using Procedures to streamline your code
    and minimize repetition
  \end{itemize}
\item
  Scripting 6.6. Controlling the User

  \begin{itemize}
  \tightlist
  \item
    This section talks about using Pause forms to ask the user questions
    and to confirm measurements.
  \end{itemize}
\item
  Scripting 7.7. Scripting the Editor

  \begin{itemize}
  \tightlist
  \item
    This section will teach you how to work within the editor window in
    Praat scripts.
  \end{itemize}
\end{itemize}

\hypertarget{in-defense-of-code-cannibalism}{%
\subsection{In defense of Code
Cannibalism}\label{in-defense-of-code-cannibalism}}

There is very seldom any need to reinvent the wheel, and there is no
shortage whatsoever of Praat scripts available online and floating
around departments. Although it's unlikely that you'll happen to find a
script which does \emph{exactly} what you need, chances are very good
that what you need can be cobbled together from several scripts. So long
as you give attribution to the original author(s) of the script(s) you
borrow code from, there's no shame in reusing parts of other scripts as
you work on your own.

Also, it's worth noting that one of the best ways to learn how to code
is to find a script, see how it works (in terms of what it does) by
running it, and then looking at the script's code to find out how it
does what it does. Once you've started collecting scripts, you'll find
yourself reaching for certain chunks of code every time you, say, want
to open and analyze every file in a given folder, or need to generate a
plot showing nasality in a given word.

To that end, I've published a collection of my most favorite scripts at
\url{https://github.com/stylerw/styler_praat_scripts}. Among them is
demo\_formant\_script.praat, a very basic formant measurement script
that takes in TextGridded sound files and spits out the formant readings
for those files. This script was written specifically to be used for
learning and cannibalism, and is commented heavily. There's much to
learn from these scripts, although many have been written using an older
version of the Praat scripting language.

In short, the more time you spend nosing around scripts, the better.

\hypertarget{closing-remarks-on-praat-scripting}{%
\subsection{Closing Remarks on Praat
scripting}\label{closing-remarks-on-praat-scripting}}

Scripting can very quickly start to save you some time, and as such, the
payoff for even a small amount of work can be great.

However, to get really incredible results, and to trim hours off of your
measurement tasks, you'll need to put more time in. You'll need to code,
then code some more, then go steal somebody else's code, then re-code it
because they did it wrong. Then you'll need to spend lots of time with
Praat's tutorials, then spend lots of time on the Praat users list.
You'll add and remove quotes until you never want to see one again, and
add loops until you're blue in the face.

But then finally, that one script that you've been working on
\emph{forever} will work. You'll start it, fill in the form, and it will
run like a cheetah after a hyperactive gazelle. And it'll run the next
time. And the time after that. And each time you start up your script,
it'll keep running, and you'll see the time savings mounting.

Then, one day, you'll do the math and realize that a task that would
have taken you 1900 hours of click-by-click hand measurement only takes
twelve minutes to run by script, and you've just saved yourself 79 days
of
pain\footnote{As an aside, ``doing the math’’ is an excellent way to justify the principled and cautious use of scripting and automation for data collection to reviewers or a dissertation committee.}.
At this point, you'll realize that all that time, that agony, totally
paid off, and rather than spending the afternoon taking measurements,
you can just start a script running to do it for you, then go actually
enjoy your life.

That, my friends, is why we Praat script.

\hypertarget{advanced-techniques}{%
\section{Advanced Techniques}\label{advanced-techniques}}

\label{sec:advanced}

Sometimes, you will sit down and realize you need to do something
ambitious, weird, or crazy in Praat (or elsewhere) to accomplish what
you need to. This section serves to detail, with no particular
structure, some of the methods I've used for accomplishing various crazy
tasks.

\vspace{0.5cm}
\begin{tabular}[h]{ p{0.6in} p{12cm}}
\includegraphics[width=0.5in]{danger.png} \newline \textbf{Danger!} & \raisebox{3mm}{\parbox{13cm}{\textit{The techniques described here are fairly esoteric and require some deeper understanding of the nuances of Praat.  They're here not so much because you'll likely need them in your phonetic career, but so that they're available, findable, and can save some work  for those few people who actually do.  Read on if you're curious, but if you're just getting started, feel free to stop here!}}}
\end{tabular}
\vspace{0.5cm}

\hypertarget{getting-amplitude-envelopes-and-am-demodulation-in-praat}{%
\subsection{Getting Amplitude Envelopes and AM Demodulation in
Praat}\label{getting-amplitude-envelopes-and-am-demodulation-in-praat}}

\label{sec:demod}

AM (``Amplitude Modulation'') is the process of embedding a lower
frequency signal inside a higher frequency signal, and is the process
underlying AM radio. AM Demodulation, then, is the process of undoing
this, which can be useful when dealing with machine generated signals,
and the process of AM demodulation is identical to the process of
obtaining an amplitude envelope, which can be occasionally useful for
speech analysis.

Although the human voice doesn't use AM modulation meaningfully, it can
sometimes be used for transmitting low-frequency signals generated by
specialized hardware, such as airflow measurement systems or articulator
movement trackers. This is because most consumer or pro-audio level
analog-to-digital converters filter out (or capture poorly) information
under 20 Hz or so. As such, some systems which don't want to use
higher-level capture boxes modulate information below 20 Hz into the
amplitude envelope of a higher frequency sound, and then recover the
information in software using AM Demodulation.

\begin{figure}
  \centerline{
    \mbox{\includegraphics[width=6.00in]{modflow.png}}
  }
  \caption{Here's a sample modulated Oral and Nasal Airflow signal showing /mamamamamamama/ with oral flow on channel 1.  Note that the amplitude envelope seems to show meaningful information beyond the steady signal.}
  
  \end{figure}

Once you've confirmed that AM modulation is happening (look for a high
frequency ``carrier'' sound which doesn't vary in frequency, but varies
sharply in amplitude), it's very easy to demodulate. As with
demodulating any AM signal, there are two steps.

First, we'll need to rectify the signal, allowing only positive signals
to pass through. In Praat, we can accomplish this by doing either of the
following:

\begin{enumerate}
\def\labelenumi{\arabic{enumi}.}
\tightlist
\item
  Open the Sound in Praat, and Select it
\item
  \emph{Objects -\textgreater{} Modify -\textgreater{} Formula\ldots{}}
\item
  Enter `abs(self)' as the formula, and press
  ``OK''\footnote{You can also use 'if self < 0 then 0 else self endif' as the formula here for a slightly more classical rectifier feel, but in my experience, it doesn't make a meaningful difference.}
\end{enumerate}

This simply tells Praat that every point should be interpreted as
positive. Then, we filter the sound to smooth the curve:

\begin{enumerate}
\def\labelenumi{\arabic{enumi}.}
\tightlist
\item
  Select the rectified sound
\item
  \emph{Objects -\textgreater{} Filter -\textgreater{} Filter (Pass Hann
  Band)\ldots{}}
\item
  \emph{From:} 0, \emph{To:} 100, \emph{Smoothing:} 20
\end{enumerate}

\begin{itemize}
\tightlist
\item
  These parameters will vary depending on your data.
\end{itemize}

This will leave you with a positive trace representing either the
unmodulated signal or the amplitude envelope of the selected sound.

\begin{figure}
  \centerline{
    \mbox{\includegraphics[width=6.00in]{demodflow.png}}
  }
  \caption{Here's the same sample modulated Oral and Nasal Airflow signal showing /mamamamamamama/ with oral flow on channel 1, after demodulation. Here, the airflow patterns are plainly apparent, and measurable without the noise.}
  \label{fig:flow}
  \end{figure}

Note, though, that the output is not a ``Sound'' in the classical sense,
and is inaudible if listened to. These ``Sounds'' merit some discussion.

\hypertarget{breaking-the-sound-barrier-working-with-analytical-sounds-in-praat}{%
\subsection{Breaking the ``Sound'' barrier: Working with analytical
``sounds'' in
Praat}\label{breaking-the-sound-barrier-working-with-analytical-sounds-in-praat}}

\label{sec:soundbarrier}

Although we think about ``Sounds'' as being audible signals, a ``Sound''
object in Praat (and indeed, any digitized sound) is really just a
series of ``Amplitude at Time'' pairs. Although this certainly can be
listenable ``Sound'', as we saw above in \ref{sec:demod}, we can also
import, generate, and analyze inaudible signals like airflow or
articulatory traces.

Understanding this allows the ``Create Sound From Formula'' feature (and
its friend, \emph{Objects -\textgreater{} Modify -\textgreater{}
Formula\ldots{}}) to be an incredibly useful tool for more complicated
signal analysis and graphing.

Although this is applicable to any signals, since we just discussed
nasal and oral airflow, let's discuss a common (albeit problematic)
measure of nasality in airflow: \%Nasalance. \%Nasalance is designed to
capture the degree of nasality at any point in the vowel or word, by
capturing how much of the total flow at any given moment is coming from
the nose.

One common way to calculate \%Nasalance is
\(\%Nasalance=\frac{nasal flow}{(oral flow + nasal flow)}\). Although
you could export the data as a series of timepoints and easily generate
this in R or a spreadsheet program, you might want \%Nasalance over time
to act like a ``Sound'' in Praat. If it does, you can easily graph the
outout within Praat, or output all the relevant information in a single
script, or pull \%Nasalance at only a specific time.

To do this analysis and turn it into a ``Sound'' to be graphed or
queried, we can use

\begin{quote}
\emph{Objects -\textgreater{} New -\textgreater{} Sound -\textgreater{}
Create sound from formula\ldots{}}
\end{quote}

We'll assume that you've got two ``flow'' signals, similar to those
shown in Figure \ref{fig:flow} above, called `oral\_flow' and
`nasal\_flow'. We're going to generate a new ``sound'', where each point
is mathematically determined by the same points in these two other
sounds. This is best done by script (so that the duration and sampling
rates are close at hand), but can be done manually, as below:

\begin{enumerate}
\def\labelenumi{\arabic{enumi}.}
\tightlist
\item
  Extract the flow into two sounds, called `oral\_flow' and
  `nasal\_flow'.
\item
  \emph{Objects -\textgreater{} New -\textgreater{} Sound
  -\textgreater{} Create sound from formula\ldots{}}
\end{enumerate}

\begin{itemize}
\tightlist
\item
  Name is whatever you'd like
\item
  One channel
\item
  The start time is `0', the end time is the duration of the oral and
  nasal flow files (their durations should be identical)
\item
  Sampling frequency must match the flow files
\end{itemize}

The most basic formula we could use would be:

\begin{quote}
\texttt{Sound\_nasal\_flow [col] /(Sound\_nasal\_flow [col] + Sound\_oral\_flow [col])}
\end{quote}

This implements the formula exactly, and says that the amplitude value
for the new file at any given point is equal to
\(\frac{nasal value}{(oral value + nasal value)}\).

However, we might want to be a bit more robust. Nasalance doesn't make
sense when either the oral and nasal flow is almost zero (a miniscule
bit of nasal flow during an inhalation is not really the same as ``100\%
nasal flow''). A more advanced formula, which also demonstrates the use
of `else if' in a formula in Praat, would set nasalance to zero when
either oral or nasal flow is below a certain level:

\begin{quote}
\texttt{if Sound\_nasal\_flow [col] < 0.02 then 0 else if Sound\_oral\_flow [col] < 0.02 then 0 else Sound\_nasal\_flow [col] /(Sound\_nasal\_flow [col] + Sound\_oral\_flow [col]) endif endif}
\end{quote}

Put into prose, this is saying:

\begin{quote}
If nasal or oral flow is below a certain threshold (0.02 here),
nasalance is 0. Otherwise, it's
\(\frac{nasal value}{(oral value + nasal value)}\)
\end{quote}

This results in a new ``sound'' which is \emph{nothing} like a sound
(see Figure \ref{fig:nasalance}), as it contains only values from 0 to 1
over time, but which can be queried, drawn (see Figure
\ref{fig:nasalancepretty} , filtered, or extracted using Praat scripts.

\begin{figure}
  \centerline{
    \mbox{\includegraphics[width=6.00in]{nasalance.png}}
  }
  \caption{Here's the resulting \%Nasalance file resulting from the /mamamamamamama/ flow signals in Figure \ref{fig:flow}.  Below is the same sound, graphed more attractively using the Picture window.}
  \label{fig:nasalance}
  \end{figure}

Of course, this same technique of generating, analyzing, and graphing
`sounds' which aren't really sounds can be used for any sort of data.
The sole requirement is that the data can be expressed as a single
numerical value which changes over time. It's a powerful technique, and
can save a great deal of Praat-external analysis, but it also requires
considerable understanding of the mathematics of digital signals, and of
exactly what you need to do.

\vspace{0.5cm}
\begin{tabular}[h]{ p{0.6in} p{12cm}}
\includegraphics[width=0.5in]{danger.png} \newline \textbf{Danger!} & \raisebox{3mm}{\parbox{13cm}{\textit{DC signals like these (which don't move back and forth around 0) can be dangerous to some kinds of speakers and headphones (as a constant level of current is pushed through the wiring without any break or reversal for the wiring to cool down).  Although a single playback of a short DC file is unlikely to hurt anything (although it won't be audible), hitting "Play" on an hour-long, high-amplitude DC track could damage your speakers or headphones in some scenarios.  With great (DC) power comes great responsibility.}}}
\end{tabular}
\vspace{0.5cm}

\begin{figure}
  \centerline{
    \mbox{\includegraphics[width=6.00in]{nasalance_picture.png}}
  }
  \caption{This is the same \%Nasalance file resulting from /mamamamamamama/ in Figure \ref{fig:nasalance}, graphed more attractively using the 'draw' command and the Picture window.}
  \label{fig:nasalancepretty}
  \end{figure}

\pagebreak

\hypertarget{conclusion}{%
\section{Conclusion}\label{conclusion}}

\label{conclusion}

Praat is unquestionably powerful software. Although there are other
packages and tools which may offer some improvements in some specific
domains, there is no other program which can do even half of what Praat
can do without resorting to scripting. And, on top of that, unlike its
only serious contender (Matlab with other signal processing toolkits),
Praat is freely available and open source.

I hope this guide has helped you learn some of the basics of Praat in a
linguistics and speech-science context, and has shown you some of its
potential for more complex (and automated) analysis. But, you're far
from ``done'', because sooner or later, as you ask more and more
intricate questions, you'll find an analysis you need to do or a number
you need to get that Praat simply doesn't have a command for.

At this point, remember that because Praat is also a toolkit and a
scripting language, the possibilities for conducting research are
frighteningly close to endless, so you can probably do what you need to
do within Praat.

Even after 15 years, I'm regularly running into new things which I need
to do in Praat, but don't quite know how to. If you're struggling with a
new problem, you can do the same things I do. Read the Praat
documentation, experiment with other approaches, look for other people's
scripts, or even post to the Praat Users Mailing List (See Section
\ref{sec:otherresources}), where Paul Boersma himself answers questions,
even about deeply esoteric issues. But each time I run into a wall or
suspect something isn't doable, after I think about it a bit, mess
around with it for a while, or break down and ask somebody else, I
usually find a way.

In closing, I'd wish you easy analyses, but those are usually boring.
I'd wish you smooth sailing, but this is speech, so that's never going
to happen. I'd wish you good data, but speakers won't provide it, and
I'd wish you pleasant work, but sooner or later, the abstract will be
due in 1.5 hours.

So, instead of all that idealistic tripe, I leave you with one simple
wish, for your use of Praat and your life in Linguistics overall:

May you find a way.
\pagebreak

\bibliography{bibtex}                                                     
\bibliographystyle{apalike}

\end{document}
